% Options for packages loaded elsewhere
\PassOptionsToPackage{unicode}{hyperref}
\PassOptionsToPackage{hyphens}{url}
%
\documentclass[
]{book}
\usepackage{amsmath,amssymb}
\usepackage{lmodern}
\usepackage{iftex}
\ifPDFTeX
  \usepackage[T1]{fontenc}
  \usepackage[utf8]{inputenc}
  \usepackage{textcomp} % provide euro and other symbols
\else % if luatex or xetex
  \usepackage{unicode-math}
  \defaultfontfeatures{Scale=MatchLowercase}
  \defaultfontfeatures[\rmfamily]{Ligatures=TeX,Scale=1}
\fi
% Use upquote if available, for straight quotes in verbatim environments
\IfFileExists{upquote.sty}{\usepackage{upquote}}{}
\IfFileExists{microtype.sty}{% use microtype if available
  \usepackage[]{microtype}
  \UseMicrotypeSet[protrusion]{basicmath} % disable protrusion for tt fonts
}{}
\makeatletter
\@ifundefined{KOMAClassName}{% if non-KOMA class
  \IfFileExists{parskip.sty}{%
    \usepackage{parskip}
  }{% else
    \setlength{\parindent}{0pt}
    \setlength{\parskip}{6pt plus 2pt minus 1pt}}
}{% if KOMA class
  \KOMAoptions{parskip=half}}
\makeatother
\usepackage{xcolor}
\IfFileExists{xurl.sty}{\usepackage{xurl}}{} % add URL line breaks if available
\IfFileExists{bookmark.sty}{\usepackage{bookmark}}{\usepackage{hyperref}}
\hypersetup{
  pdftitle={Derecho Concursal Chileno},
  pdfauthor={Luis Aguilera Arteaga},
  hidelinks,
  pdfcreator={LaTeX via pandoc}}
\urlstyle{same} % disable monospaced font for URLs
\usepackage{longtable,booktabs,array}
\usepackage{calc} % for calculating minipage widths
% Correct order of tables after \paragraph or \subparagraph
\usepackage{etoolbox}
\makeatletter
\patchcmd\longtable{\par}{\if@noskipsec\mbox{}\fi\par}{}{}
\makeatother
% Allow footnotes in longtable head/foot
\IfFileExists{footnotehyper.sty}{\usepackage{footnotehyper}}{\usepackage{footnote}}
\makesavenoteenv{longtable}
\usepackage{graphicx}
\makeatletter
\def\maxwidth{\ifdim\Gin@nat@width>\linewidth\linewidth\else\Gin@nat@width\fi}
\def\maxheight{\ifdim\Gin@nat@height>\textheight\textheight\else\Gin@nat@height\fi}
\makeatother
% Scale images if necessary, so that they will not overflow the page
% margins by default, and it is still possible to overwrite the defaults
% using explicit options in \includegraphics[width, height, ...]{}
\setkeys{Gin}{width=\maxwidth,height=\maxheight,keepaspectratio}
% Set default figure placement to htbp
\makeatletter
\def\fps@figure{htbp}
\makeatother
\setlength{\emergencystretch}{3em} % prevent overfull lines
\providecommand{\tightlist}{%
  \setlength{\itemsep}{0pt}\setlength{\parskip}{0pt}}
\setcounter{secnumdepth}{5}
\usepackage{booktabs}
\ifLuaTeX
  \usepackage{selnolig}  % disable illegal ligatures
\fi
\usepackage[]{natbib}
\bibliographystyle{plainnat}

\title{Derecho Concursal Chileno}
\author{Luis Aguilera Arteaga}
\date{2022-06-21}

\begin{document}
\maketitle

{
\setcounter{tocdepth}{1}
\tableofcontents
}
\hypertarget{ley-20.720-sobre-insolvencia-y-reemprendimiento}{%
\chapter*{Ley 20.720 Sobre Insolvencia y Reemprendimiento}\label{ley-20.720-sobre-insolvencia-y-reemprendimiento}}
\addcontentsline{toc}{chapter}{Ley 20.720 Sobre Insolvencia y Reemprendimiento}

\hypertarget{capuxedtulo-i-disposiciones-generales}{%
\section*{CAPÍTULO I: DISPOSICIONES GENERALES}\label{capuxedtulo-i-disposiciones-generales}}
\addcontentsline{toc}{section}{CAPÍTULO I: DISPOSICIONES GENERALES}

\hypertarget{artuxedculo-1.--uxe1mbito-de-aplicaciuxf3n-de-la-ley.}{%
\paragraph*{Artículo 1°.- Ámbito de aplicación de la ley.}\label{artuxedculo-1.--uxe1mbito-de-aplicaciuxf3n-de-la-ley.}}
\addcontentsline{toc}{paragraph}{Artículo 1°.- Ámbito de aplicación de la ley.}

La presente ley establece el régimen general de los procedimientos concursales destinados a reorganizar y/o liquidar los pasivos y activos de una Empresa Deudora, y a repactar los pasivos y/o liquidar los activos de una Persona Deudora.

\hypertarget{artuxedculo-2.--definiciones.}{%
\paragraph*{Artículo 2°.- Definiciones.}\label{artuxedculo-2.--definiciones.}}
\addcontentsline{toc}{paragraph}{Artículo 2°.- Definiciones.}

Para efectos de esta ley, se entenderá, en singular o plural, por:

\textbf{1) Acuerdo de Reorganización Judicial:} aquel que se suscribe entre una Empresa Deudora y sus acreedores con el fin de reestructurar sus activos y pasivos, con sujeción al procedimiento establecido en los Títulos 1 y 2 del Capítulo III. Para los efectos de esta ley, se denominará indistintamente ``Acuerdo de Reorganización Judicial'' o ``Acuerdo''.

\textbf{2) Acuerdo de Reorganización Extrajudicial o Simplificado:} aquel que se suscribe entre una Empresa Deudora y sus acreedores con el fin de reestructurar sus activos y pasivos, y que se somete a aprobación judicial con sujeción al procedimiento establecido en el Título 3 del Capítulo III. Para los efectos de esta ley, se denominará indistintamente ``Acuerdo de Reorganización Extrajudicial o Simplificado'' o ``Acuerdo Simplificado''.

\textbf{3) Avalúo Fiscal:} el precio de los inmuebles fijado por el Servicio de Impuestos Internos para los efectos del pago del impuesto territorial.

\textbf{4) Audiencia Inicial:} aquella que se lleva a cabo en el tribunal competente con presencia del Deudor, si comparece, en un procedimiento de Liquidación Forzosa, en los términos establecidos en el artículo 120.

\textbf{5) Audiencia de Prueba:} aquella que se verifica en el marco de un juicio de oposición, en la cual se rinden las pruebas ofrecidas en la Audiencia Inicial, en los términos establecidos en el artículo 126.

\textbf{6) Audiencia de Fallo:} aquella en que se notifica la sentencia definitiva, poniéndose término al juicio de oposición, en lo términos establecidos en el artículo 127.

\textbf{7) Boletín Concursal:} plataforma electrónica a cargo de la Superintendencia de Insolvencia y Reemprendimiento, de libre acceso al público, gratuito, en la que se publicarán todas las resoluciones que se dicten y las actuaciones que se realicen en los procedimientos concursales, salvo que la ley ordene otra forma de notificación.

\textbf{8) Certificado de Nominación:} aquel emitido por la Superintendencia de Insolvencia y Reemprendimiento, en el cual consta la nominación del Veedor o Liquidador, titular y suplente.

\textbf{9) Comisión de acreedores:} aquella que puede designarse en un Procedimiento Concursal de Reorganización con el objetivo de supervigilar el cumplimiento del Acuerdo de Reorganización Judicial, con las atribuciones y deberes que dicho acuerdo señale; o aquella que puede designarse en un Procedimiento Concursal de Liquidación para adoptar los acuerdos que la Junta de Acreedores le delegue.

\textbf{10) Correo electrónico:} medio de comunicación electrónica que permite el envío y recepción de información y documentos electrónicos.

\textbf{11) Cuenta final de administración:} aquella rendición de cuentas de su gestión que debe efectuar tanto el Veedor como el Liquidador en la oportunidad prevista en la ley, ante el tribunal, en la que deberá observarse la normativa contable, tributaria y financiera aplicable, así como la de esta ley.

\textbf{12) Deudor:} toda Empresa Deudora o Persona Deudora, atendido el Procedimiento Concursal de que se trate y la naturaleza de la disposición a que se refiera.

\textbf{13) Empresa Deudora:} toda persona jurídica privada, con o sin fines de lucro, y toda persona natural contribuyente de primera categoría o del número 2) del artículo 42 del decreto ley Nº 824, del Ministerio de Hacienda, de 1974, que aprueba la ley sobre impuesto a la renta.

\textbf{14) Informe del Veedor:} aquel relativo al Acuerdo de Reorganización Judicial, regulado en el número 8) del artículo 57 de esta ley.

\textbf{15) Junta de Acreedores:} órgano concursal constituido por los acreedores de un Deudor sujeto a un Procedimiento Concursal, de conformidad a esta ley. Se denominarán, según corresponda, Junta Constitutiva, Junta Ordinaria o Junta Extraordinaria, o indistintamente ``Junta de Acreedores'' o ``Junta''.

\textbf{16) Ley:} ley de Reorganización y Liquidación de Activos de Empresas y Personas.

\textbf{17) Liquidación Forzosa:} demanda presentada por cualquier acreedor del Deudor, conforme al Párrafo 2 del Título 1 del Capítulo IV de esta ley.

\textbf{18) Liquidación Voluntaria:} aquella solicitada por el Deudor, conforme al Párrafo 1 del Título 1 del Capítulo IV de esta ley.

\textbf{19) Liquidador:} aquella persona natural sujeta a la fiscalización de la Superintendencia de Insolvencia y Reemprendimiento, cuya misión principal es realizar el activo del Deudor y propender al pago de los créditos de sus acreedores, de acuerdo a lo establecido en esta ley.

\textbf{20) Martillero Concursal:} aquel martillero público que voluntariamente se somete a la fiscalización de la Superintendencia de Insolvencia y Reemprendimiento, cuya misión principal es realizar los bienes del Deudor, en conformidad a lo encomendado por la Junta de Acreedores y de acuerdo a lo establecido en esta ley.

\textbf{21) Nómina de Veedores:} registro público integrado por las personas naturales nombradas como Veedores por la Superintendencia de Insolvencia y Reemprendimiento, en conformidad al Párrafo 1 del Título 1 del Capítulo II de esta ley.

\textbf{22) Nómina de Liquidadores:} registro público integrado por las personas naturales nombradas como Liquidadores por la Superintendencia de Insolvencia y Reemprendimiento, en conformidad al Párrafo 1 del Título 2 del Capítulo II de esta ley.

\textbf{23) Nómina de Árbitros Concursales:} registro público integrado por las personas naturales nombradas como Árbitros Concursales por la Superintendencia de Insolvencia y Reemprendimiento, en conformidad al Capítulo VII de esta ley.

\textbf{24) Nómina de Martilleros Concursales:} registro público llevado por la Superintendencia de Insolvencia y Reemprendimiento que integra a los martilleros públicos que cumplen con lo prescrito en el artículo 213 de esta ley.

\textbf{25) Persona Deudora:} toda persona natural no comprendida en la definición de Empresa Deudora.

\textbf{26) Persona Relacionada:} se considerarán Personas Relacionadas respecto de una o más personas o de sus representantes, las siguientes:

\begin{enumerate}
\def\labelenumi{\alph{enumi})}
\item
  El cónyuge, los ascendientes, descendientes y colaterales por consanguinidad o afinidad hasta el sexto grado inclusive y las sociedades en que éstos participen, con excepción de aquellas inscritas en el Registro de Valores.
\item
  Las personas que se encuentren en alguna de las situaciones a que se refiere el artículo 100 de la ley Nº 18.045, de Mercado de Valores.
\end{enumerate}

\textbf{27) Procedimiento Concursal:} cualquiera de los regulados en esta ley, denominados, indistintamente, Procedimiento Concursal de Reorganización de la Empresa Deudora, Procedimiento Concursal de Liquidación de la Empresa Deudora, Procedimiento Concursal de Renegociación de la Persona Deudora y Procedimiento Concursal de Liquidación de los Bienes de la Persona Deudora.

\textbf{28) Procedimiento Concursal de Liquidación:} aquél regulado en el Capítulo IV de esta ley.

\textbf{29) Procedimiento Concursal de Reorganización:} aquél regulado en el Capítulo III de esta ley.

\textbf{30) Procedimiento Concursal de Renegociación:} aquél regulado en el Capítulo V de esta ley.

\textbf{31) Protección Financiera Concursal:} aquel período que esta ley otorga al Deudor que se somete al Procedimiento Concursal de Reorganización, durante el cual no podrá solicitarse ni declararse su liquidación, ni podrán iniciarse en su contra juicios ejecutivos, ejecuciones de cualquier clase o restituciones en los juicios de arrendamiento. Dicho período será el comprendido entre la notificación de la Resolución de Reorganización y el Acuerdo de Reorganización Judicial, o el plazo fijado por la ley si este último no se acuerda.

\textbf{32) Quórum Especial:} el conformado por dos tercios del pasivo total con derecho a voto verificado y/o reconocido, según corresponda, en el Procedimiento Concursal respectivo.

\textbf{33) Quórum Calificado:} el conformado por la mayoría absoluta del pasivo total con derecho a voto verificado y/o reconocido, según corresponda, en el Procedimiento Concursal respectivo.

\textbf{34) Quórum Simple:} el conformado por la mayoría del pasivo verificado y/o reconocido, según corresponda, con derecho a voto, presente en la Junta de Acreedores, en el Procedimiento Concursal respectivo.

\textbf{35) Resolución de Admisibilidad:} aquella resolución administrativa dictada por la Superintendencia de Insolvencia y Reemprendimiento conforme al artículo 263, que produce los efectos del artículo 264, ambos del Capítulo V de esta ley.

\textbf{36) Resolución de Liquidación:} aquella resolución judicial dictada en un Procedimiento Concursal que produce los efectos señalados en el Párrafo 4 del Título 1 del Capítulo IV de esta ley.

\textbf{37) Resolución de Reorganización:} aquella resolución judicial dictada en un Procedimiento Concursal que produce los efectos señalados en el artículo 57 de esta ley.

\textbf{38) Servicios de Utilidad Pública:} aquéllos considerados como consumos básicos, cuyos prestadores se encuentran regulados por leyes especiales y sujetos a la fiscalización de la autoridad, tales como agua, electricidad, gas, teléfono e internet.

\textbf{39) Superintendencia:} la Superintendencia de Insolvencia y Reemprendimiento.

\textbf{40) Veedor:} aquella persona natural sujeta a la fiscalización de la Superintendencia de Insolvencia y Reemprendimiento, cuya misión principal es propiciar los acuerdos entre el Deudor y sus acreedores, facilitar la proposición de Acuerdos de Reorganización Judicial y resguardar los intereses de los acreedores, requiriendo las medidas precautorias y de conservación de los activos del Deudor, de acuerdo a lo establecido en esta ley.

\hypertarget{artuxedculo-3uxba.--competencia.}{%
\paragraph*{Artículo 3º.- Competencia.}\label{artuxedculo-3uxba.--competencia.}}
\addcontentsline{toc}{paragraph}{Artículo 3º.- Competencia.}

Los Procedimientos Concursales contemplados en esta ley serán de competencia del juzgado de letras que corresponda al domicilio del Deudor, pudiendo interponer el acreedor el incidente de incompetencia del tribunal, de acuerdo a las reglas generales.

En las ciudades asiento de Corte la distribución se regirá por un auto acordado dictado por la Corte de Apelaciones respectiva, considerando especialmente la radicación preferente de causas concursales en los tribunales que cuenten con la capacitación a que se refiere el inciso siguiente.

Los jueces titulares y secretarios de los juzgados de letras que conozcan preferentemente de asuntos concursales deberán estar capacitados en derecho concursal, en especial, sobre las disposiciones de esta ley y de las leyes especiales que rijan estas materias.

Cada Corte de Apelaciones adoptará las medidas pertinentes para garantizar la especialización a que se refiere la presente disposición.

No obstante, los demás tribunales competentes estarán habilitados para conocer de asuntos concursales en el marco de sus atribuciones si, excepcionalmente y por circunstancias derivadas del sistema de distribución de trabajo, ello fuere necesario.

El tribunal al cual corresponda conocer de un Procedimiento Concursal de aquellos contemplados en esta ley, no perderá su competencia por el hecho de existir entre los acreedores y el Deudor personas que gocen de fuero especial.

Para los efectos de lo previsto en este artículo, la Academia Judicial coordinará la dictación de los cursos necesarios para la capacitación en derecho concursal de jueces titulares y secretarios de los juzgados de letras dentro del programa de perfeccionamiento de miembros del Poder Judicial establecido en la ley N° 19.346, que crea la Academia Judicial.

\hypertarget{artuxedculo-4.--recursos.}{%
\paragraph*{Artículo 4°.- Recursos.}\label{artuxedculo-4.--recursos.}}
\addcontentsline{toc}{paragraph}{Artículo 4°.- Recursos.}

Las resoluciones judiciales que se pronuncien en los Procedimientos Concursales de Reorganización y de Liquidación establecidos en esta ley sólo serán susceptibles de los recursos que siguen:

\textbf{1) Reposición:} procederá contra aquellas resoluciones susceptibles de este recurso conforme a las reglas generales, deberá interponerse dentro del plazo de tres días contado desde la notificación de aquélla y podrá resolverse de plano o previa tramitación incidental, según determine el tribunal. Contra la resolución que resuelva la reposición no procederá recurso alguno.

\textbf{2) Apelación:} procederá contra las resoluciones que esta ley señale expresamente y deberá interponerse dentro del plazo de cinco días contado desde la notificación de aquéllas. Será concedida en el solo efecto devolutivo, salvo las excepciones que esta ley señale y, en ambos, casos gozará de preferencia para su inclusión en la tabla y para su vista y fallo.

En el caso de las resoluciones susceptibles de recurrirse de reposición y de apelación, la segunda deberá interponerse en subsidio de la primera, de acuerdo a las reglas generales.

\textbf{3) Casación:} procederá en los casos y en las formas establecidas en la ley.

\hypertarget{artuxedculo-5.--incidentes.}{%
\paragraph*{Artículo 5°.- Incidentes.}\label{artuxedculo-5.--incidentes.}}
\addcontentsline{toc}{paragraph}{Artículo 5°.- Incidentes.}

Sólo podrán promoverse incidentes en aquellas materias en que esta ley lo permita expresamente. Se tramitarán conforme a las reglas generales previstas en el Código de Procedimiento Civil y no suspenderán el Procedimiento Concursal, salvo que esta ley establezca lo contrario.

\hypertarget{artuxedculo-6uxba.--de-las-notificaciones.}{%
\paragraph*{Artículo 6º.- De las notificaciones.}\label{artuxedculo-6uxba.--de-las-notificaciones.}}
\addcontentsline{toc}{paragraph}{Artículo 6º.- De las notificaciones.}

Siempre que el tribunal ordene que una resolución se notifique por avisos, deberá realizarse mediante una publicación en el Boletín Concursal, entendiéndose notificada desde la fecha de su inserción en aquél.

Las notificaciones efectuadas en el Boletín Concursal serán de carácter público y deberán ser realizadas por el Veedor, el Liquidador o la Superintendencia, según corresponda, dentro de los dos días siguientes a la dictación de las respectivas resoluciones, salvo que la norma correspondiente disponga un plazo diferente.

Toda resolución que no tenga señalada una forma distinta de notificación, se entenderá efectuada mediante una publicación en el Boletín Concursal.

Mediante norma de carácter general, la Superintendencia establecerá la forma de efectuar las publicaciones, los requisitos técnicos de operación y seguridad del Boletín Concursal y la obligación de actualizarlo diariamente por quien corresponda.

Cada vez que se establezca que una resolución debe notificarse por Correo Electrónico, se estará a lo dispuesto en la norma de carácter general en cuanto a la forma de efectuarla. En todo caso, en la primera actuación que se realice ante el tribunal o la Superintendencia de Insolvencia y Reemprendimiento, según corresponda, en los Procedimientos Concursales, el Deudor, los acreedores y los terceros interesados señalarán una dirección de Correo Electrónico válida a la cual se deberán efectuar las notificaciones conforme a lo dispuesto precedentemente.

La notificación por Correo Electrónico enviada a la dirección señalada por el respectivo notificado será válida, aun cuando aquella no se encontrare vigente, estuviere en desuso o no permitiere su recepción por el destinatario. Se entenderá notificado el destinatario desde el envío del Correo Electrónico a la referida dirección.

En los casos en que no sea posible notificar por Correo Electrónico, se notificará por carta certificada y dicha notificación se entenderá efectuada al tercer día siguiente al de su recepción en la oficina de correos.

De todas las notificaciones que se practiquen en virtud de lo dispuesto en este artículo se dejará constancia por escrito en el expediente, sin que sea necesaria certificación alguna al respecto.

Cada vez que la ley ordene al Deudor señalar el Correo Electrónico de sus acreedores, se entenderá que debe indicar el de los representantes legales de aquéllos.

Una vez finalizados los Procedimientos Concursales, en la forma prescrita en esta ley, la Superintendencia deberá proceder a la eliminación, modificación o bloqueo de los datos del Deudor en el Boletín Concursal, en conformidad con lo establecido en la ley N° 19.628, sobre protección de la vida privada.

\hypertarget{artuxedculo-7.--cuxf3mputo-de-plazos.}{%
\paragraph*{Artículo 7°.- Cómputo de plazos.}\label{artuxedculo-7.--cuxf3mputo-de-plazos.}}
\addcontentsline{toc}{paragraph}{Artículo 7°.- Cómputo de plazos.}

Los plazos de días establecidos en esta ley son de días hábiles, entendiéndose inhábiles los días domingos y feriados, salvo que se establezca que un plazo específico es de días corridos. Los plazos se computarán desde el día siguiente a aquél en que se notifique la resolución o el acto respectivo.

Cuando esta ley establezca un plazo para actuaciones que deban realizarse antes de determinada fecha, éste se contará hacia atrás a partir del día inmediatamente anterior al de la respectiva actuación.

\hypertarget{artuxedculo-8.--exigibilidad.}{%
\paragraph*{Artículo 8°.- Exigibilidad.}\label{artuxedculo-8.--exigibilidad.}}
\addcontentsline{toc}{paragraph}{Artículo 8°.- Exigibilidad.}

Las normas contenidas en leyes especiales prevalecerán sobre las disposiciones de esta ley.

Aquellas materias que no estén reguladas expresamente por leyes especiales, se regirán supletoriamente por las disposiciones de esta ley.

\hypertarget{capuxedtulo-ii-del-veedor-y-del-liquidador}{%
\section*{CAPÍTULO II: DEL VEEDOR Y DEL LIQUIDADOR}\label{capuxedtulo-ii-del-veedor-y-del-liquidador}}
\addcontentsline{toc}{section}{CAPÍTULO II: DEL VEEDOR Y DEL LIQUIDADOR}

\hypertarget{tuxedtulo-1.-del-veedor}{%
\subsection*{Título 1. Del Veedor}\label{tuxedtulo-1.-del-veedor}}
\addcontentsline{toc}{subsection}{Título 1. Del Veedor}

\hypertarget{puxe1rrafo-1.-de-la-nuxf3mina-de-veedores}{%
\subsubsection*{Párrafo 1. De la Nómina de Veedores}\label{puxe1rrafo-1.-de-la-nuxf3mina-de-veedores}}
\addcontentsline{toc}{subsubsection}{Párrafo 1. De la Nómina de Veedores}

\hypertarget{artuxedculo-9.--estructura.}{%
\paragraph*{Artículo 9°.- Estructura.}\label{artuxedculo-9.--estructura.}}
\addcontentsline{toc}{paragraph}{Artículo 9°.- Estructura.}

La Nómina de Veedores estará integrada por las personas naturales nombradas en el cargo de Veedor por la Superintendencia, la que la mantendrá debidamente actualizada y a disposición del público a través de su página web.

\hypertarget{artuxedculo-10.--solicitud-de-inscripciuxf3n.}{%
\paragraph*{Artículo 10.- Solicitud de inscripción.}\label{artuxedculo-10.--solicitud-de-inscripciuxf3n.}}
\addcontentsline{toc}{paragraph}{Artículo 10.- Solicitud de inscripción.}

Toda persona natural interesada en ser nombrada Veedor podrá presentar su solicitud ante la Superintendencia. En ella deberá expresar si ejercería el cargo a nivel nacional o regional, acompañando los antecedentes que acrediten el cumplimiento de los requisitos señalados en el artículo 13 y una declaración jurada en que exprese no estar afecto a las prohibiciones contempladas en el artículo 17.

\hypertarget{artuxedculo-11.--inclusiuxf3n-en-la-nuxf3mina-de-veedores.}{%
\paragraph*{Artículo 11.- Inclusión en la Nómina de Veedores.}\label{artuxedculo-11.--inclusiuxf3n-en-la-nuxf3mina-de-veedores.}}
\addcontentsline{toc}{paragraph}{Artículo 11.- Inclusión en la Nómina de Veedores.}

El Veedor será incorporado a la nómina correspondiente mediante resolución dictada por la Superintendencia, una vez verificado el cumplimiento de los requisitos señalados en el artículo 13.

\hypertarget{artuxedculo-12.--menciones-de-la-nuxf3mina-de-veedores.}{%
\paragraph*{Artículo 12.- Menciones de la Nómina de Veedores.}\label{artuxedculo-12.--menciones-de-la-nuxf3mina-de-veedores.}}
\addcontentsline{toc}{paragraph}{Artículo 12.- Menciones de la Nómina de Veedores.}

La referida Nómina contendrá las siguientes menciones respecto de cada Veedor:

\begin{enumerate}
\def\labelenumi{\arabic{enumi})}
\item
  Nombre completo, profesión, domicilio, datos de contacto y regiones en que ejercerá sus funciones.
\item
  Calificaciones obtenidas durante los últimos cinco años en el examen a que se refiere el artículo 14.
\item
  Número total de Procedimientos Concursales de Reorganización en que hubiere intervenido, con mención de aquellos en que se hubiere aprobado el Acuerdo de Reorganización, de los cinco principales acreedores y el sector o rubro de los Deudores en cada uno de ellos.
\item
  Honorario promedio percibido.
\item
  Registro de las sanciones aplicadas.
\end{enumerate}

\hypertarget{puxe1rrafo-2.-del-veedor}{%
\subsubsection*{Párrafo 2. Del Veedor}\label{puxe1rrafo-2.-del-veedor}}
\addcontentsline{toc}{subsubsection}{Párrafo 2. Del Veedor}

\hypertarget{artuxedculo-13.--requisitos.}{%
\paragraph*{Artículo 13.- Requisitos.}\label{artuxedculo-13.--requisitos.}}
\addcontentsline{toc}{paragraph}{Artículo 13.- Requisitos.}

Podrá solicitar su inclusión en la Nómina de Veedores toda persona natural que cumpla con los siguientes requisitos:

\begin{enumerate}
\def\labelenumi{\arabic{enumi})}
\item
  Contar con un título profesional de contador auditor o de una profesión de a lo menos diez semestres de duración, otorgado por universidades del Estado o reconocidas por éste, o por la Corte Suprema, en su caso;
\item
  Contar con, a lo menos, cinco años de ejercicio de la profesión que haga valer;
\item
  Aprobar el examen para Veedores a que se refiere el artículo siguiente;
\item
  No estar afecto a alguna de las prohibiciones establecidas en el artículo 17, y
\item
  Otorgar, en tiempo y forma, la garantía señalada en el artículo 16.
\end{enumerate}

\hypertarget{artuxedculo-14.--del-examen-de-conocimientos.}{%
\paragraph*{Artículo 14.- Del examen de conocimientos.}\label{artuxedculo-14.--del-examen-de-conocimientos.}}
\addcontentsline{toc}{paragraph}{Artículo 14.- Del examen de conocimientos.}

La Superintendencia convocará a un examen de conocimientos a las siguientes personas:

\begin{enumerate}
\def\labelenumi{\arabic{enumi})}
\item
  Postulantes a integrar la Nómina de Veedores.
\item
  Veedores que no hubieren asumido Procedimientos Concursales de Reorganización en un período de tres años contado desde su último examen rendido y aprobado.
\item
  Veedores que hubieren reprobado el examen en conformidad con lo establecido en el presente artículo.
\end{enumerate}

El Veedor que hubiere reprobado el examen podrá rendirlo nuevamente en el período siguiente de examinación, en la fecha, hora y lugar que fije la Superintendencia. La inasistencia injustificada se entenderá como reprobación para todos los efectos legales.

El Veedor que hubiere reprobado el examen de repetición quedará suspendido de pleno derecho para asumir nuevos Procedimientos Concursales de Reorganización, aún como interventor, por un período de doce meses contado desde la notificación de su reprobación efectuada por correo electrónico, y hasta que apruebe un nuevo examen, debiendo rendirlo una vez terminado el período de suspensión, en la fecha de examinación correspondiente. Si reprueba nuevamente el examen de repetición, será excluido de la Nómina de Veedores.

El examen de conocimientos señalado en este artículo se convocará dos veces en cada año calendario y será regulado por la Superintendencia a través de normas de carácter general.

\hypertarget{artuxedculo-15.--responsabilidad.}{%
\paragraph*{Artículo 15.- Responsabilidad.}\label{artuxedculo-15.--responsabilidad.}}
\addcontentsline{toc}{paragraph}{Artículo 15.- Responsabilidad.}

La responsabilidad civil del Veedor alcanzará hasta la culpa levísima y podrá perseguirse cuando corresponda, en cuyo caso se aplicarán las reglas del juicio sumario, una vez presentada la Cuenta Final de Administración, conforme a lo dispuesto en el Párrafo 2 del Título 3 del Capítulo II de esta ley, y sin perjuicio de la responsabilidad legal en que pudiere incurrir.

Sin perjuicio de lo anterior, si el Veedor no rindiere su Cuenta Final de Administración dentro del plazo regulado en el artículo 29, su responsabilidad civil también podrá perseguirse desde el vencimiento de dicho plazo.

\hypertarget{artuxedculo-16.--garantuxeda-de-fiel-desempeuxf1o.}{%
\paragraph*{Artículo 16.- Garantía de fiel desempeño.}\label{artuxedculo-16.--garantuxeda-de-fiel-desempeuxf1o.}}
\addcontentsline{toc}{paragraph}{Artículo 16.- Garantía de fiel desempeño.}

Todo Veedor mantendrá en la Superintendencia y mientras subsista su responsabilidad, una garantía por un monto de 2.000 unidades de fomento, con una vigencia mínima de tres años, renovable por igual período. En caso de no otorgarla en tiempo y forma, el Veedor no podrá asumir en nuevos Procedimientos Concursales de Reorganización.

La garantía podrá consistir en una boleta bancaria de garantía, póliza de seguro o cualquiera otra que la Superintendencia determine mediante norma de carácter general, la cual también establecerá la forma de rendirla, sus plazos, devolución, renovación y demás especificaciones aplicables.

La garantía a que se refiere este artículo tiene por objetivo caucionar el fiel desempeño de la actividad del Veedor y asegurar el correcto y cabal cumplimiento de todas sus obligaciones, incluyendo la eventual indemnización a que sea condenado en caso de hacerse efectiva su responsabilidad civil y el pago de las multas administrativas impuestas en su contra.

La Superintendencia hará efectiva la garantía y entregará su monto a requerimiento del tribunal que hubiere declarado la responsabilidad civil del Veedor, siempre que la resolución condenatoria se encuentre firme y ejecutoriada. Tratándose de multas impuestas por la propia Superintendencia, la resolución respectiva indicará el plazo en que el Veedor deberá pagarlas, el cual no podrá ser inferior a veinte días. Dicho plazo se contará desde que esa resolución se encuentre firme y ejecutoriada. Una vez transcurrido el término anterior sin verificarse el pago, la Superintendencia hará efectiva la garantía e imputará los fondos a la multa respectiva, restituyendo el saldo al Veedor, si correspondiere.
Sin perjuicio de lo anterior, si se ejecutare la garantía del Veedor conforme al inciso anterior, y una vez que se le restituya el saldo en su caso, se entenderá suspendido para asumir nuevos Procedimientos Concursales de Reorganización, y tendrá un plazo de veinte días para constituir una nueva garantía en los términos previstos en este artículo, manteniéndose la señalada suspensión mientras no la otorgue.

\hypertarget{artuxedculo-17.--prohibiciones.}{%
\paragraph*{Artículo 17.- Prohibiciones.}\label{artuxedculo-17.--prohibiciones.}}
\addcontentsline{toc}{paragraph}{Artículo 17.- Prohibiciones.}

No podrán ser Veedores las siguientes personas:

\begin{enumerate}
\def\labelenumi{\arabic{enumi})}
\item
  Las que hayan sido condenadas por crimen o simple delito.
\item
  Los funcionarios de cualquier órgano de la Administración del Estado, los integrantes de las empresas públicas creadas por ley, los que ejerzan cargos de elección popular, y aquellos que presten cualquier tipo de servicios remunerados o no a la Superintendencia.
\end{enumerate}

No obstante, no regirá esta incompatibilidad respecto de las personas que desempeñen labores docentes en instituciones de educación superior. Sin embargo, no se considerarán labores docentes las que correspondan a la dirección superior de una entidad académica, respecto de las cuales regirá la incompatibilidad a que se refiere este numeral.

\begin{enumerate}
\def\labelenumi{\arabic{enumi})}
\setcounter{enumi}{2}
\item
  Las que tuvieren incapacidad física o mental para ejercer el cargo.
\item
  Las que hubieren dejado de integrar la Nómina de Veedores en virtud de las causales de exclusión del artículo siguiente y sin perjuicio de lo dispuesto en el inciso final del mismo.
\end{enumerate}

\hypertarget{artuxedculo-18.--causales-de-exclusiuxf3n-de-la-nuxf3mina-de-veedores.}{%
\paragraph*{Artículo 18.- Causales de exclusión de la Nómina de Veedores.}\label{artuxedculo-18.--causales-de-exclusiuxf3n-de-la-nuxf3mina-de-veedores.}}
\addcontentsline{toc}{paragraph}{Artículo 18.- Causales de exclusión de la Nómina de Veedores.}

Los Veedores serán excluidos de su respectiva Nómina en los siguientes casos:

\begin{enumerate}
\def\labelenumi{\arabic{enumi})}
\item
  Por haber sido nombrados en contravención a lo dispuesto en este Título.
\item
  Por dejar de cumplir los requisitos enumerados en el artículo 13 de este Título.
\item
  Por adquirir para sí o para terceros, ya sea como persona natural o a través de una persona jurídica en la que el Veedor sea socio o Persona Relacionada, cualquier bien u obtener para sí alguna ventaja económica en los Procedimientos Concursales en que intervengan como Veedor.
\item
  Por enajenar o autorizar la enajenación de cualquier bien en los Procedimientos Concursales en que intervenga como Veedor a:
\end{enumerate}

\begin{enumerate}
\def\labelenumi{\alph{enumi})}
\item
  Sus Personas Relacionadas.
\item
  Alguna persona jurídica en que tenga interés económico directo o indirecto.
\item
  Socios o accionistas de una sociedad en la que el Veedor forme parte, o de las sociedades en las cuales tenga participación, salvo aquellas que se encuentren inscritas en el Registro de Valores y hagan oferta pública de ellos.
\item
  Personas con las que posea bienes en comunidad, con excepción de los copropietarios a que se refiere la ley Nº 19.537, sobre copropiedad inmobiliaria.
\item
  Sus dependientes.
\item
  Profesionales o técnicos que le presten servicios, sean éstos esporádicos o permanentes, cualquiera sea la forma en que estén constituidos.
\end{enumerate}

\begin{enumerate}
\def\labelenumi{\arabic{enumi})}
\setcounter{enumi}{4}
\item
  Por haberse declarado judicialmente, mediante sentencia firme y ejecutoriada, su responsabilidad civil o penal en conformidad con el artículo 27.
\item
  Por renuncia presentada ante la Superintendencia, sin perjuicio de las obligaciones y responsabilidades por las funciones que ya hubiere asumido.
\item
  Por sentencia firme y ejecutoriada que rechace la Cuenta Final de Administración que debe presentar en conformidad a esta ley.
\item
  Por aplicación de la letra c) del artículo 339.
\item
  Por reprobación definitiva del examen de conocimientos a que se refiere el artículo 14.
\item
  Por muerte.
\end{enumerate}

Producida alguna de las circunstancias señaladas en los números precedentes, la Superintendencia dictará la resolución de exclusión respectiva.

Sin perjuicio de lo anterior, en el evento que se produzcan algunas de las circunstancias previstas en los numerales 1), 2), 3), 4) y 8) anteriores, la Superintendencia deberá previamente representarla al Veedor para que éste presente sus descargos, dentro de los cinco días siguientes. Vencido el plazo señalado sin que se presente descargo alguno, la Superintendencia dictará la correspondiente resolución de exclusión. Si el Veedor presenta sus descargos, la Superintendencia podrá acogerlos o rechazarlos dictando la correspondiente resolución.

Las personas excluidas de la Nómina de Veedores por las causales de los números 1), 2) y 6) podrán solicitar, una vez transcurridos cinco años contados desde la fecha en que quedó firme el acto administrativo de exclusión, su reincorporación en la referida nómina, estándose a lo dispuesto en el presente Título.

Las personas excluidas de la Nómina de Veedores por cualquier otra causal no podrán volver a solicitar su inscripción en ella.

Lo anterior será sin perjuicio de la responsabilidad civil y penal que pudiere corresponderles en conformidad a la ley.

\hypertarget{artuxedculo-19.--reclamo-de-exclusiuxf3n.}{%
\paragraph*{Artículo 19.- Reclamo de exclusión.}\label{artuxedculo-19.--reclamo-de-exclusiuxf3n.}}
\addcontentsline{toc}{paragraph}{Artículo 19.- Reclamo de exclusión.}

El Veedor podrá reclamar de su exclusión de la respectiva nómina ante el juzgado de letras con competencia en lo civil de su domicilio dentro del plazo de diez días contado desde la notificación por carta certificada de la resolución que decida dicha exclusión.

El tribunal competente sujetará la tramitación del reclamo a las normas del procedimiento sumario, conforme a lo establecido en el artículo 341. Mientras se encuentre pendiente el reclamo de exclusión, el Veedor no podrá asumir nuevos Procedimientos Concursales.

Excluido el Veedor de la Nómina de Veedores, subsistirá la obligación de rendir cuenta de su gestión, así como la responsabilidad legal en que pudiere haber incurrido.

\hypertarget{artuxedculo-20.--designaciuxf3n-del-veedor-en-los-procedimientos-concursales.}{%
\paragraph*{Artículo 20.- Designación del Veedor en los Procedimientos Concursales.}\label{artuxedculo-20.--designaciuxf3n-del-veedor-en-los-procedimientos-concursales.}}
\addcontentsline{toc}{paragraph}{Artículo 20.- Designación del Veedor en los Procedimientos Concursales.}

Sólo podrá designarse Veedor a quien integre la Nómina de Veedores a la época de la dictación de la Resolución de Reorganización o de la Resolución de Liquidación, según corresponda.

\hypertarget{artuxedculo-21.--inhabilidades.}{%
\paragraph*{Artículo 21.- Inhabilidades.}\label{artuxedculo-21.--inhabilidades.}}
\addcontentsline{toc}{paragraph}{Artículo 21.- Inhabilidades.}

No podrán ser nominados o designados Veedores en un Procedimiento Concursal de Reorganización:

\begin{enumerate}
\def\labelenumi{\arabic{enumi})}
\item
  Las Personas Relacionadas con el Deudor.
\item
  Los deudores y acreedores del Deudor o sus representantes, y todos los que tuvieren un interés directo o indirecto en el respectivo procedimiento.
\item
  Los que tuvieren objetada su Cuenta Final de Administración en un Procedimiento Concursal, siempre que hayan insistido en uno o más reparos.
\item
  Los que estuvieren suspendidos en conformidad a lo dispuesto en el artículo 14 o de acuerdo al número 5) del artículo 337 de esta ley.
\end{enumerate}

\hypertarget{artuxedculo-22.--nominaciuxf3n-del-veedor.}{%
\paragraph*{Artículo 22.- Nominación del Veedor.}\label{artuxedculo-22.--nominaciuxf3n-del-veedor.}}
\addcontentsline{toc}{paragraph}{Artículo 22.- Nominación del Veedor.}

Una vez que la Superintendencia reciba los antecedentes señalados en el artículo 55, notificará a los tres mayores acreedores del Deudor según la información entregada, dentro del día siguiente y por el medio más expedito. Esta notificación será certificada por el ministro de fe de la Superintendencia para todos los efectos legales.

Dentro del segundo día siguiente a la referida notificación, cada acreedor propondrá por escrito o por correo electrónico a un Veedor titular y a un Veedor suplente vigente en la Nómina de Veedores. Para estos efectos, cada acreedor será individualmente considerado sin distinción del monto de su crédito.

Dentro del día siguiente al señalado en el inciso anterior, la Superintendencia nominará como Veedor titular al que hubiere obtenido la primera mayoría de entre los propuestos para el cargo de titular por los acreedores, y como Veedor suplente a aquel que hubiere obtenido la primera mayoría de entre los propuestos para ese cargo. Si sólo respondiere un acreedor, se estará a su propuesta. Si respondieren todos o dos de ellos y la propuesta recayere en personas diversas, se estará a aquella del acreedor cuyo crédito sea superior.

En caso que no se reciban propuestas, la nominación tendrá lugar mediante sorteo ante la Superintendencia, en el que participarán aquellos Veedores que integren la terna propuesta por el Deudor en la solicitud señalada en el artículo 54 o, en su defecto, todos aquellos Veedores vigentes en la Nómina de Veedores a esa fecha. Los sorteos que efectúe la Superintendencia se regularán por medio de una norma de carácter general.

Excepcionalmente, si de los antecedentes señalados en el artículo 55, se acredita que un solo acreedor representa más del 50\% del pasivo del deudor, la Superintendencia nominará al Veedor Titular y al Veedor Suplente propuesto por ese acreedor. En caso que dicho acreedor no propusiere al Veedor Titular y al Veedor Suplente, se estará a las reglas generales establecidas en los incisos anteriores.

El Veedor titular y el Veedor suplente nominados serán inmediatamente notificados por la Superintendencia por el medio más expedito.

El Veedor titular nominado deberá manifestar ante la Superintendencia si acepta el cargo a más tardar al día siguiente a su notificación y deberá jurar o prometer desempeñarlo fielmente. Al aceptar el cargo, deberá declarar sus relaciones con el Deudor o con los acreedores de éste, si las tuviere, y que no tiene impedimento o inhabilidad alguna para desempeñar el cargo.

Aceptado el cargo, la Superintendencia emitirá el Certificado de Nominación del Veedor, el cual será remitido directamente al tribunal competente, dentro del día siguiente a su emisión, para que éste designe a un Veedor nominado en la Resolución de Reorganización.
El Veedor podrá excusarse de aceptar una nominación ante la Superintendencia, debiendo expresar fundadamente y por escrito su justificación al día siguiente de su notificación. La Superintendencia resolverá dentro de los dos días siguientes con los antecedentes aportados por el Veedor y sin ulterior recurso. Si la excusa es desestimada, el Veedor deberá asumir como tal en el Procedimiento Concursal, entendiéndose legalmente aceptado el cargo desde que se resuelva la excusa y se emita el correspondiente Certificado de Nominación. Si la excusa es aceptada, la Superintendencia nominará al Veedor suplente como titular, nominándose a un nuevo Veedor suplente mediante sorteo.

\hypertarget{artuxedculo-23.--de-la-cesaciuxf3n-en-el-cargo.}{%
\paragraph*{Artículo 23.- De la cesación en el cargo.}\label{artuxedculo-23.--de-la-cesaciuxf3n-en-el-cargo.}}
\addcontentsline{toc}{paragraph}{Artículo 23.- De la cesación en el cargo.}

El Veedor cesará en el cargo por el término del Procedimiento Concursal de Reorganización o por cese anticipado en el mismo. Sin perjuicio de lo anterior, subsistirá su responsabilidad hasta la aprobación de su Cuenta Final de Administración.

\hypertarget{artuxedculo-24.--del-cese-anticipado-en-el-cargo.}{%
\paragraph*{Artículo 24.- Del cese anticipado en el cargo.}\label{artuxedculo-24.--del-cese-anticipado-en-el-cargo.}}
\addcontentsline{toc}{paragraph}{Artículo 24.- Del cese anticipado en el cargo.}

Para los efectos de esta ley, se entenderá que el Veedor cesa anticipadamente en su cargo:
1) Por la revocación de la Junta de Acreedores.

\begin{enumerate}
\def\labelenumi{\arabic{enumi})}
\setcounter{enumi}{1}
\item
  Por remoción decretada por el tribunal.
\item
  Por renuncia aceptada por la Junta de Acreedores o, en su defecto, por el tribunal, la que deberá fundarse en una causa grave.
\item
  Por haber dejado de formar parte de la Nómina de Veedores, sin perjuicio de continuar en el cargo hasta que asuma el Veedor Suplente o el que se designe.
\item
  Por inhabilidad sobreviniente. El Veedor deberá dar cuenta al tribunal y a la Superintendencia, dentro del plazo de tres días, de la inhabilidad que le afecte. El incumplimiento de esta obligación será constitutivo de falta gravísima para los efectos de lo dispuesto en el número 8) del artículo 18.
\end{enumerate}

El Veedor suplente asumirá dentro de los dos días siguientes a la cesación en el cargo del Veedor titular, cualquiera sea la causa del cese.

El Veedor que haya cesado anticipadamente en su cargo deberá rendir cuenta de su gestión y hacer entrega de los antecedentes del Procedimiento Concursal al Veedor suplente, dentro de los diez días siguientes a la fecha en que este último haya asumido. En caso de incumplimiento, el tribunal competente, de oficio o a petición de cualquier interesado, requerirá el cumplimiento según lo previsto en el artículo 238 del Código de Procedimiento Civil, en cuyo caso la multa será de 10 a 200 unidades tributarias mensuales. Sin perjuicio de lo anterior, la Superintendencia podrá aplicar las sanciones que correspondan.

\hypertarget{artuxedculo-25.--deberes-del-veedor.}{%
\paragraph*{Artículo 25.- Deberes del Veedor.}\label{artuxedculo-25.--deberes-del-veedor.}}
\addcontentsline{toc}{paragraph}{Artículo 25.- Deberes del Veedor.}

La función principal del Veedor es propiciar los acuerdos entre el Deudor y sus acreedores, facilitando la proposición y negociación del Acuerdo. Para estos efectos, el Veedor podrá citar al Deudor y a sus acreedores en cualquier momento desde la publicación de la Resolución de Reorganización hasta la fecha en que debe acompañar al tribunal competente el informe que regula el numeral 8) del artículo 57, con el propósito de facilitar los acuerdos entre las partes y propiciar la celebración de un Acuerdo de Reorganización Judicial en los términos regulados en la presente ley.

En el ejercicio de sus funciones deberá especialmente:

\begin{enumerate}
\def\labelenumi{\arabic{enumi})}
\item
  Imponerse de los libros, documentos y operaciones del Deudor.
\item
  Incorporar y publicar en el Boletín Concursal copia de todos los antecedentes y resoluciones que esta ley le ordene.
\item
  Realizar las inscripciones y notificaciones que disponga la Resolución de Reorganización.
\item
  Realizar las labores de fiscalización y valorización que se le imponen en los artículos 72 y siguientes, referidas a la continuidad del suministro, a la venta necesaria de activos y a la obtención de nuevos recursos.
\item
  Arbitrar las medidas necesarias en el procedimiento de determinación del pasivo establecido en los artículos 70 y 71.
\item
  Realizar la calificación de los poderes para comparecer en las Juntas de Acreedores e informar al tribunal competente sobre la legalidad de éstos, cuando corresponda.
\item
  Impetrar las medidas precautorias y de conservación de los activos del Deudor que sean necesarias para resguardar los intereses de los acreedores, sin perjuicio de los acuerdos que éstos puedan adoptar.
\item
  Dar cuenta al tribunal competente y a la Superintendencia de cualquier acto o conducta del Deudor que signifique una administración negligente o dolosa de sus negocios y, con la autorización de dicho tribunal, adoptar las medidas necesarias para mantener la integridad de los activos, cuando corresponda.
\item
  Rendir mensualmente cuenta de su actuación y de los negocios del Deudor a la Superintendencia, y presentar las observaciones que le merezca la administración de aquél. Esta cuenta será enviada, además, por correo electrónico a cada uno de los acreedores.
\item
  Ejecutar todos los actos que le encomiende esta ley.
\end{enumerate}

\hypertarget{artuxedculo-26.--delegaciuxf3n-de-funciones.}{%
\paragraph*{Artículo 26.- Delegación de funciones.}\label{artuxedculo-26.--delegaciuxf3n-de-funciones.}}
\addcontentsline{toc}{paragraph}{Artículo 26.- Delegación de funciones.}

El Veedor sólo podrá delegar sus funciones, manteniendo su responsabilidad y a su costa, en otros Veedores vigentes en la Nómina de Veedores, con igual competencia territorial.
La referida delegación deberá efectuarse por instrumento público, en el que conste la aceptación del delegado, el que será agregado al expediente y notificado mediante su publicación en el Boletín Concursal.

\hypertarget{artuxedculo-27.--concierto-previo.}{%
\paragraph*{Artículo 27.- Concierto Previo.}\label{artuxedculo-27.--concierto-previo.}}
\addcontentsline{toc}{paragraph}{Artículo 27.- Concierto Previo.}

El Veedor que se concertare con el Deudor, con algún acreedor o un tercero para proporcionarle alguna ventaja indebida o para obtenerla para sí, será sancionado de conformidad a lo establecido en el Párrafo 7 del Título IX del Libro Segundo del Código Penal.

\hypertarget{artuxedculo-28.--honorarios-del-veedor.}{%
\paragraph*{Artículo 28.- Honorarios del Veedor.}\label{artuxedculo-28.--honorarios-del-veedor.}}
\addcontentsline{toc}{paragraph}{Artículo 28.- Honorarios del Veedor.}

Los honorarios del Veedor serán convenidos entre éste, los tres principales acreedores y el Deudor y serán de cargo de este último. Estos honorarios gozarán de la preferencia establecida en el número 4 del artículo 2472 del Código Civil, sin perjuicio de lo prescrito en el número 3) del artículo 118 de esta ley.

\hypertarget{artuxedculo-29.--de-la-cuenta-final.}{%
\paragraph*{Artículo 29.- De la Cuenta Final.}\label{artuxedculo-29.--de-la-cuenta-final.}}
\addcontentsline{toc}{paragraph}{Artículo 29.- De la Cuenta Final.}

El Veedor rendirá cuenta final de su gestión en el plazo de treinta días contado desde la Resolución que aprueba el Acuerdo de Reorganización Judicial o desde la Resolución de Liquidación, en su caso. Al respecto, le será plenamente aplicable lo dispuesto en el Párrafo 2 del Título 3 del Capítulo II de esta ley.

\hypertarget{tuxedtulo-2.-del-liquidador}{%
\subsection*{Título 2. Del Liquidador}\label{tuxedtulo-2.-del-liquidador}}
\addcontentsline{toc}{subsection}{Título 2. Del Liquidador}

\hypertarget{puxe1rrafo-1.-de-la-nuxf3mina-de-liquidadores}{%
\subsubsection*{Párrafo 1. De la Nómina de Liquidadores}\label{puxe1rrafo-1.-de-la-nuxf3mina-de-liquidadores}}
\addcontentsline{toc}{subsubsection}{Párrafo 1. De la Nómina de Liquidadores}

\hypertarget{artuxedculo-30.--estructura.}{%
\paragraph*{Artículo 30.- Estructura.}\label{artuxedculo-30.--estructura.}}
\addcontentsline{toc}{paragraph}{Artículo 30.- Estructura.}

La Nómina de Liquidadores estará integrada por todas las personas naturales nombradas como tales por la Superintendencia, la que deberá mantenerla debidamente actualizada y a disposición del público a través de su página web.

\hypertarget{artuxedculo-31.--norma-general.}{%
\paragraph*{Artículo 31.- Norma general.}\label{artuxedculo-31.--norma-general.}}
\addcontentsline{toc}{paragraph}{Artículo 31.- Norma general.}

Será aplicable a los Liquidadores lo dispuesto en el Título 1 del Capítulo II de la presente ley respecto de los Veedores, en todo aquello que no esté expresamente regulado en el presente Título y, en todo caso, siempre que no sea contrario a la naturaleza de la función que desempeñan.

\hypertarget{artuxedculo-32.--requisitos.}{%
\paragraph*{Artículo 32.- Requisitos.}\label{artuxedculo-32.--requisitos.}}
\addcontentsline{toc}{paragraph}{Artículo 32.- Requisitos.}

Podrá ser Liquidador y solicitar su inclusión en la Nómina de Liquidadores, toda persona natural que cumpla con los siguientes requisitos:

\begin{enumerate}
\def\labelenumi{\arabic{enumi})}
\item
  Contar con un título profesional de contador auditor o de una profesión de a lo menos diez semestres de duración, otorgado por universidades del Estado o reconocidas por éste, o por la Corte Suprema, en su caso.
\item
  Contar con, a lo menos, cinco años de ejercicio de la profesión que haga valer.
\item
  Aprobar un examen de conocimientos para Liquidadores, en los términos del artículo 14.
\item
  No estar afecto a alguna de las prohibiciones establecidas en el artículo 17.
\item
  Otorgar, en tiempo y forma, la garantía señalada en el artículo 16.
\end{enumerate}

\hypertarget{artuxedculo-33.--menciones-de-la-nuxf3mina-de-liquidadores.}{%
\paragraph*{Artículo 33.- Menciones de la Nómina de Liquidadores.}\label{artuxedculo-33.--menciones-de-la-nuxf3mina-de-liquidadores.}}
\addcontentsline{toc}{paragraph}{Artículo 33.- Menciones de la Nómina de Liquidadores.}

Además de las menciones señaladas en el artículo 12, la Nómina de Liquidadores deberá contener el régimen de descuento de honorarios ofrecido por el Liquidador y su respectiva vigencia, respecto de la tabla del artículo 40.

Asimismo, deberá señalar el número de Procedimientos Concursales de Liquidación en que cada Liquidador hubiere intervenido, la lista de los cinco principales acreedores en cada uno de ellos, el porcentaje de Procedimientos Concursales de Liquidación con Cuenta Final de Administración aprobada y el sector o rubro del Deudor en cada uno de dichos procedimientos.

\hypertarget{artuxedculo-34.--causales-de-exclusiuxf3n-de-la-nuxf3mina-de-liquidadores.}{%
\paragraph*{Artículo 34.- Causales de exclusión de la Nómina de Liquidadores.}\label{artuxedculo-34.--causales-de-exclusiuxf3n-de-la-nuxf3mina-de-liquidadores.}}
\addcontentsline{toc}{paragraph}{Artículo 34.- Causales de exclusión de la Nómina de Liquidadores.}

Además de las causales de exclusión señaladas en el artículo 18, será excluido de la Nómina de Liquidadores aquel que se negare a asumir un Procedimiento Concursal de Liquidación sin causa justificada.

Para estos efectos, se entenderá como causa justificada las señaladas en esta ley.

\hypertarget{puxe1rrafo-2.-del-liquidador}{%
\subsubsection*{Párrafo 2. Del Liquidador}\label{puxe1rrafo-2.-del-liquidador}}
\addcontentsline{toc}{subsubsection}{Párrafo 2. Del Liquidador}

\hypertarget{artuxedculo-35.--responsabilidad.}{%
\paragraph*{Artículo 35.- Responsabilidad.}\label{artuxedculo-35.--responsabilidad.}}
\addcontentsline{toc}{paragraph}{Artículo 35.- Responsabilidad.}

La responsabilidad civil de los Liquidadores alcanzará hasta la culpa levísima y se podrá perseguir, cuando corresponda, en juicio sumario una vez presentada la Cuenta Final de Administración, conforme lo dispuesto en los artículos 49 y siguientes de esta ley, y sin perjuicio de la responsabilidad legal en que pudiere incurrir.

Sin perjuicio de lo anterior, si el Liquidador no rindiere su Cuenta Final de Administración dentro del plazo regulado en el artículo 50, su responsabilidad civil también podrá perseguirse desde el vencimiento de dicho plazo.

\hypertarget{artuxedculo-36.--deberes-del-liquidador.}{%
\paragraph*{Artículo 36.- Deberes del Liquidador.}\label{artuxedculo-36.--deberes-del-liquidador.}}
\addcontentsline{toc}{paragraph}{Artículo 36.- Deberes del Liquidador.}

El Liquidador representa judicial y extrajudicialmente los intereses generales de los acreedores y los derechos del Deudor en cuanto puedan interesar a la masa, sin perjuicio de las facultades de aquéllos y de éste determinadas por esta ley.

En el ejercicio de sus funciones, el Liquidador deberá especialmente, con arreglo a esta ley:

\begin{enumerate}
\def\labelenumi{\arabic{enumi})}
\item
  Incautar e inventariar los bienes del Deudor.
\item
  Liquidar los bienes del Deudor.
\item
  Efectuar los repartos de fondos a los acreedores en la forma dispuesta en el Párrafo 3 del Título 5 del Capítulo IV de esta ley.
\item
  Cobrar los créditos del activo del Deudor.
\item
  Contratar préstamos para solventar los gastos del Procedimiento Concursal de Liquidación.
\item
  Exigir rendición de cuentas de cualquiera que haya administrado bienes del Deudor.
\item
  Reclamar del Deudor la entrega de la información necesaria para el desempeño de su cargo.
\item
  Registrar sus actuaciones y publicar las resoluciones que se dicten en el Procedimiento Concursal de Liquidación en el Boletín Concursal.
\item
  Depositar a interés en una institución financiera los fondos que perciba, en cuenta separada para cada Procedimiento Concursal de Liquidación y a nombre de éste, y abrir una cuenta corriente con los fondos para solventarlo.
\item
  Ejecutar los acuerdos legalmente adoptados por la Junta de Acreedores dentro del ámbito de su competencia.
\item
  Cerrar los libros de comercio del Deudor, quedando responsable por ello frente a terceros desde la dictación de la Resolución de Liquidación.
\item
  Transigir y conciliar los créditos laborales con el acuerdo de la Junta de Acreedores, según lo dispone el artículo 246 de esta ley.
\item
  Ejercer las demás facultades y cumplir las demás obligaciones que le encomienda la presente ley.
\end{enumerate}

\hypertarget{artuxedculo-37.--nominaciuxf3n-del-liquidador.}{%
\paragraph*{Artículo 37.- Nominación del Liquidador.}\label{artuxedculo-37.--nominaciuxf3n-del-liquidador.}}
\addcontentsline{toc}{paragraph}{Artículo 37.- Nominación del Liquidador.}

Presentada una solicitud de inicio de Procedimiento Concursal de Liquidación ante el tribunal competente, la Superintendencia nominará al Liquidador conforme al procedimiento establecido en el presente artículo, salvo en el caso previsto en el número 3 del artículo 120.

Tratándose de una solicitud de Liquidación Voluntaria, el Deudor acompañará a la Superintendencia copia de la respectiva solicitud con cargo del tribunal competente o de la Corte de Apelaciones correspondiente y copia de la nómina de acreedores y sus créditos, de acuerdo a lo establecido en el artículo 115 de esta ley.

Tratándose de una solicitud de Liquidación Forzosa, el acreedor peticionario acompañará a la Superintendencia copia de la respectiva solicitud con cargo del tribunal competente o de la Corte de Apelaciones correspondiente y copia de la nómina de acreedores y sus créditos que haya acompañado el Deudor, en su caso, de acuerdo a lo establecido en el artículo 120 de esta ley.

Acompañados los antecedentes antes señalados, la Superintendencia notificará a los tres mayores acreedores del Deudor, que no sean Personas Relacionadas de éste, según la información entregada, dentro del día siguiente y por el medio más expedito, lo que será certificado por un ministro de fe de la Superintendencia.

Dentro del segundo día siguiente a la referida notificación, cada acreedor propondrá por escrito o por correo electrónico a un Liquidador titular y a un Liquidador suplente vigentes en la Nómina de Liquidadores. Para estos efectos, cada acreedor será individualmente considerado, sin distinción del monto de su crédito.

Dentro del día siguiente al señalado en el inciso anterior, la Superintendencia nominará como Liquidador titular al que hubiere obtenido la primera mayoría de entre los propuestos para ese cargo por los acreedores, y como suplente a aquel que hubiere obtenido la primera mayoría de entre los propuestos para ese cargo. Si sólo respondiere un acreedor, se estará a su propuesta. Si respondieren todos o dos de ellos y la propuesta recayere en personas diversas, se estará a aquella del acreedor cuyo crédito sea superior. En caso que no se reciban propuestas, la nominación tendrá lugar mediante sorteo ante la Superintendencia, en el que participarán todos aquellos Liquidadores vigentes en la Nómina de Liquidadores a esa fecha.

Los sorteos que efectúe la Superintendencia se regularán por medio de una norma de carácter general.

Excepcionalmente, si de los antecedentes acompañados a la Superintendencia por el Deudor o acreedor peticionario, según corresponda, se acredita que un solo acreedor representa más del 50\% del pasivo del deudor, la Superintendencia nominará al Liquidador titular y al suplente propuesto por dicho acreedor. En caso que dicho acreedor no propusiere al Liquidador titular y al suplente, se estará a las reglas generales establecidas en los incisos anteriores.

Los Liquidadores titular y suplente nominados serán inmediatamente notificados por la Superintendencia por el medio más expedito.

El Liquidador titular nominado deberá manifestar ante la Superintendencia, a más tardar al día siguiente de su notificación, si acepta el cargo y deberá jurar o prometer desempeñarlo fielmente. Al aceptar el cargo deberá declarar sus relaciones con el Deudor y los acreedores de éste, y que no tiene impedimento o inhabilidad alguna para desempeñarlo.

El Liquidador podrá excusarse ante la Superintendencia de aceptar una nominación, debiendo expresar fundadamente y por escrito sus justificaciones, al día siguiente de su notificación. La Superintendencia resolverá dentro de los dos días siguientes con los antecedentes aportados por el Liquidador y sin ulterior recurso. Si la excusa es desestimada, el Liquidador deberá asumir como tal en el Procedimiento Concursal de Liquidación, entendiéndose legalmente aceptado el cargo desde que se resuelva la excusa y se emita el correspondiente Certificado de Nominación. Si la excusa es aceptada, la Superintendencia nominará al Liquidador suplente como titular, nominándose a un nuevo Liquidador suplente mediante sorteo.

Aceptado el cargo, la Superintendencia emitirá el Certificado de Nominación del Liquidador, el cual será remitido directamente al tribunal competente, dentro del día siguiente a su emisión, para que éste lo designe como Liquidador en carácter de provisional en la Resolución de Liquidación.

\hypertarget{artuxedculo-38.--cese-anticipado-en-el-cargo.}{%
\paragraph*{Artículo 38.- Cese anticipado en el cargo.}\label{artuxedculo-38.--cese-anticipado-en-el-cargo.}}
\addcontentsline{toc}{paragraph}{Artículo 38.- Cese anticipado en el cargo.}

El Liquidador cesará anticipadamente en el cargo por no haberse confirmado su nominación por la Junta de Acreedores; por haberse aprobado un Acuerdo de Reorganización Judicial o un Acuerdo de Reorganización Simplificado que termine con el Procedimiento Concursal de Liquidación, o por lo dispuesto en los artículos 23 y 24, que serán aplicables, en lo que corresponda, al Liquidador.

Si el Liquidador titular cesare anticipadamente en el cargo asumirá el suplente, sin perjuicio de la facultad de la Junta de Acreedores de designar uno nuevo. Si no pudiere asumir el Liquidador suplente, la Superintendencia deberá citar a Junta Extraordinaria de Acreedores con el fin de que se designe un Liquidador titular y a uno suplente, en caso que los acreedores no los hubieren designado. Si dicha junta no se celebra por falta de quórum, la Superintendencia hará la designación por sorteo.

Los Liquidadores que fueren designados de conformidad a este artículo deberán asumir aun cuando el Procedimiento Concursal de Liquidación no tuviere bienes o fondos por repartir.

\hypertarget{artuxedculo-39.--honorarios-del-liquidador.}{%
\paragraph*{Artículo 39.- Honorarios del Liquidador.}\label{artuxedculo-39.--honorarios-del-liquidador.}}
\addcontentsline{toc}{paragraph}{Artículo 39.- Honorarios del Liquidador.}

Los honorarios a percibir por los Liquidadores en los Procedimientos Concursales de Liquidación a su cargo se sujetarán a las disposiciones siguientes:

\begin{enumerate}
\def\labelenumi{\arabic{enumi})}
\item
  Se determinarán de conformidad a la tabla progresiva por tramos prevista en el artículo siguiente.
\item
  Tendrán la naturaleza de remuneración única y de gasto de administración del Procedimiento Concursal de Liquidación para todos los efectos legales a que hubiere lugar.
  Serán de cargo del Liquidador todos los gastos correspondientes al ejercicio de su cargo, así como los honorarios de todos sus asesores jurídicos, técnicos, administrativos o de cualquier otra índole que hubiere contratado para el desarrollo de su actividad.
\end{enumerate}

Si el domicilio del Deudor fuere distinto al del Liquidador, los gastos de traslado y otros necesarios para el Procedimiento Concursal de Liquidación se considerarán gastos de administración y deberán ser ratificados por la Junta o, en subsidio, por el tribunal competente.

\begin{enumerate}
\def\labelenumi{\arabic{enumi})}
\setcounter{enumi}{2}
\item
  No se incluirán aquellos honorarios que se devenguen en caso de la continuación de actividades económicas del Deudor en los términos de los artículos 232 y 233 de esta ley.
\item
  Sólo podrán pagarse honorarios adicionales si los acreedores lo acuerdan en Junta de Acreedores. El pago de este aumento será de cargo exclusivo de aquellos acreedores que lo hubieren votado favorablemente.
\item
  Los honorarios se calcularán considerando los montos reservados de conformidad a lo dispuesto en los números 2 y 3 del artículo 247, pero sólo se pagarán los correspondientes a los fondos efectivamente repartidos de acuerdo a la tabla progresiva por tramos prevista en el artículo siguiente.
\item
  El Liquidador deberá retener en instrumentos de renta fija, a nombre del Deudor sujeto a un Procedimiento Concursal de Liquidación, el 10\% del honorario que le correspondería percibir en cada reparto. Estos honorarios sólo podrán ingresar al patrimonio del Liquidador una vez presentada la Cuenta Final de Administración, conforme a lo dispuesto en los artículos 49 y siguientes. Si la señalada cuenta es rechazada por sentencia firme, estos fondos serán restituidos a la masa, debiendo ser destinados para el pago de los honorarios del nuevo Liquidador designado en caso que no hubiere fondos por repartir.
\item
  Podrán acordarse en Junta de Acreedores, con Quórum Simple, anticipos de honorarios al Liquidador, los que no podrán exceder del 10\% de los ingresos en dinero efectivo que haya producido el Procedimiento Concursal de Liquidación al momento del anticipo.
\item
  Si el Liquidador cesa anticipadamente en el cargo conforme al artículo 38, sus honorarios y los de quien lo reemplace serán acordados entre el Liquidador respectivo y la Junta de Acreedores. Faltando dicho acuerdo, resolverá el tribunal competente sin ulterior recurso.
\item
  Se prohíbe al Liquidador o a sus Personas Relacionadas recibir a cualquier título otro pago distinto de los regulados en el presente artículo, por parte de algún acreedor o de sus Personas Relacionadas.
\end{enumerate}

\hypertarget{artuxedculo-40.--tabla-de-honorarios.}{%
\paragraph*{Artículo 40.- Tabla de Honorarios.}\label{artuxedculo-40.--tabla-de-honorarios.}}
\addcontentsline{toc}{paragraph}{Artículo 40.- Tabla de Honorarios.}

El honorario único a que se refiere el artículo anterior deberá pagarse al Liquidador en su equivalente en pesos a la fecha del respectivo reparto, de conformidad a la tabla progresiva por tramos regulada a continuación:

\begin{enumerate}
\def\labelenumi{\arabic{enumi})}
\item
  Sobre la parte que exceda de 0 y no sobrepase de 2.000 unidades de fomento, 20\%.
\item
  Sobre la parte que exceda de 2.000 y no sobrepase las 4.000 unidades de fomento, 15\%.
\item
  Sobre la parte que exceda de 4.000 y no sobrepase las 8.000 unidades de fomento, 11\%.
\item
  Sobre la parte que exceda de 8.000 y no sobrepase las 16.000 unidades de fomento, 8\%.
\item
  Sobre la parte que exceda de 16.000 y no sobrepase las 32.000 unidades de fomento, 6\%.
\item
  Sobre la parte que exceda de 32.000 y no sobrepase las 64.000 unidades de fomento, 4\%.
\item
  Sobre la parte que exceda de 64.000 y no sobrepase las 130.000 unidades de fomento, 3\%.
\item
  Sobre la parte que exceda de 130.000 y no sobrepase las 260.000 unidades de fomento, 2,25\%.
\item
  Sobre la parte que exceda de 260.000 y no sobrepase las 520.000 unidades de fomento, 1,75\%.
\item
  Sobre la parte que exceda de 520.000 y no sobrepase las 1.000.000 de unidades de fomento, 1,5\%.
\item
  Sobre la parte que exceda de 1.000.000 de unidades de fomento, 1\%.
\end{enumerate}

El primer tramo se calculará sobre los ingresos del Procedimiento Concursal de Liquidación cuando no hubiere repartos o, si habiendo repartos, correspondiere al Liquidador un honorario inferior a 30 unidades de fomento y, en este caso, el honorario no podrá exceder de esa cantidad.

Para la determinación del honorario que corresponda al Liquidador en cada reparto, se deberá calcular previamente la cantidad que le corresponda por honorarios y luego aplicar la tabla precedente en la forma progresiva descrita, a partir del respectivo tramo. En consecuencia, para la aplicación de la tabla y determinación del porcentaje del honorario que le corresponde en cada reparto, deberá considerarse el monto total distribuido en repartos anteriores.

Si luego de practicada la diligencia de incautación y confección de inventario a que se refiere el numeral 2) del artículo 163, se constatare por el Liquidador que el Deudor carece de bienes, o que éstos son insuficientes para el pago de los honorarios que pudieren corresponderle, éste sólo tendrá derecho a una remuneración de 30 unidades de fomento, que serán pagadas por la Superintendencia con cargo a su presupuesto.

\hypertarget{artuxedculo-41.--contrataciones-especializadas.}{%
\paragraph*{Artículo 41.- Contrataciones especializadas.}\label{artuxedculo-41.--contrataciones-especializadas.}}
\addcontentsline{toc}{paragraph}{Artículo 41.- Contrataciones especializadas.}

No obstante lo dispuesto en el artículo anterior y previo acuerdo adoptado en Junta de Acreedores con Quórum Calificado, el Liquidador podrá contratar, con cargo a los gastos del Procedimiento Concursal de Liquidación, personas naturales o jurídicas para que efectúen actividades especializadas debidamente calificadas como tales por la Junta de Acreedores.
Con todo, podrán realizar dichas contrataciones aun antes de la Junta Constitutiva, siempre y cuando sea estrictamente necesario, previa autorización del tribunal.

Las actividades especializadas deberán referirse directamente al cuidado y mantención del activo del Deudor, a la recuperación y realización del mismo y a su entrega material. La contratación se hará previo informe del Liquidador, el cual contendrá los fundamentos de la misma, el grado y alcance de la actividad y la forma en que se beneficiarán los acreedores o se evitarán perjuicios al activo incautado.

El Liquidador, o sus Personas Relacionadas, no podrán tener participación alguna en los actos o contratos que se ejecuten o celebren en conformidad a este artículo, salvo en cuanto a sus actividades como Liquidador en el Procedimiento Concursal de Liquidación, y tampoco podrán participar como socios, accionistas, trabajadores o asesores de las personas jurídicas que sean contratadas para las actividades o informes indicados. La transgresión a esta prohibición constituirá causa gravísima para efectos de la letra c) del artículo 339.

\hypertarget{tuxedtulo-3.-de-las-disposiciones-comunes-al-veedor-y-liquidador}{%
\subsection*{Título 3. De las disposiciones comunes al Veedor y Liquidador}\label{tuxedtulo-3.-de-las-disposiciones-comunes-al-veedor-y-liquidador}}
\addcontentsline{toc}{subsection}{Título 3. De las disposiciones comunes al Veedor y Liquidador}

\hypertarget{artuxedculo-42.--regla-general.}{%
\paragraph*{Artículo 42.- Regla general.}\label{artuxedculo-42.--regla-general.}}
\addcontentsline{toc}{paragraph}{Artículo 42.- Regla general.}

Una misma persona natural no podrá estar inscrita en la Nómina de Veedores y en la Nómina de Liquidadores.

\hypertarget{artuxedculo-43.--de-la-inclusiuxf3n-en-la-nuxf3mina-de-veedores-y-en-la-nuxf3mina-de-liquidadores.}{%
\paragraph*{Artículo 43.- De la inclusión en la Nómina de Veedores y en la Nómina de Liquidadores.}\label{artuxedculo-43.--de-la-inclusiuxf3n-en-la-nuxf3mina-de-veedores-y-en-la-nuxf3mina-de-liquidadores.}}
\addcontentsline{toc}{paragraph}{Artículo 43.- De la inclusión en la Nómina de Veedores y en la Nómina de Liquidadores.}

El registro de una persona en la Nómina de Veedores, no importará su inclusión en la Nómina de Liquidadores, ni viceversa.

\hypertarget{artuxedculo-44.--prohibiciones-relativas-del-veedor-o-liquidador.}{%
\paragraph*{Artículo 44.- Prohibiciones relativas del Veedor o Liquidador.}\label{artuxedculo-44.--prohibiciones-relativas-del-veedor-o-liquidador.}}
\addcontentsline{toc}{paragraph}{Artículo 44.- Prohibiciones relativas del Veedor o Liquidador.}

Sin perjuicio de las demás prohibiciones establecidas en esta ley, los Veedores y Liquidadores no podrán intervenir en Procesos Concursales de Reorganización o Liquidación en que no hubieren sido designados, salvo las actuaciones que les correspondan como acreedor con anterioridad al Procedimiento Concursal respectivo, de representante legal en conformidad al artículo 43 del Código Civil, y de lo previsto en el artículo 26 de esta ley. La contravención a la presente prohibición constituye una infracción gravísima para los efectos del número 8) del artículo 18.

Asimismo, los Veedores y Liquidadores no podrán contratar por sí, a través de terceros o de una persona jurídica en la que sean socios o Personas Relacionadas, con cualquier Deudor sometido a un Procedimiento Concursal.

\hypertarget{artuxedculo-45.--de-la-exclusiuxf3n-de-la-nuxf3mina-veedores-y-de-la-nuxf3mina-de-liquidadores.}{%
\paragraph*{Artículo 45.- De la exclusión de la Nómina Veedores y de la Nómina de Liquidadores.}\label{artuxedculo-45.--de-la-exclusiuxf3n-de-la-nuxf3mina-veedores-y-de-la-nuxf3mina-de-liquidadores.}}
\addcontentsline{toc}{paragraph}{Artículo 45.- De la exclusión de la Nómina Veedores y de la Nómina de Liquidadores.}

La exclusión de la Nómina de Veedores supondrá necesariamente impedimento para incorporarse a la Nómina de Liquidadores, y viceversa, salvo que se funde en el número 6) del artículo 18, en cuyo caso, excepcionalmente, podrá solicitarse la incorporación a la otra nómina, antes del plazo de 5 años señalado en el inciso cuarto del referido artículo, previa autorización de la Superintendencia.

\hypertarget{puxe1rrafo-1.-de-las-cuentas-provisorias}{%
\subsubsection*{Párrafo 1. De las cuentas provisorias}\label{puxe1rrafo-1.-de-las-cuentas-provisorias}}
\addcontentsline{toc}{subsubsection}{Párrafo 1. De las cuentas provisorias}

\hypertarget{artuxedculo-46.--contenido.}{%
\paragraph*{Artículo 46.- Contenido.}\label{artuxedculo-46.--contenido.}}
\addcontentsline{toc}{paragraph}{Artículo 46.- Contenido.}

La Superintendencia, mediante norma de carácter general, fijará la forma y contenidos obligatorios de las cuentas provisorias que deba rendir el Liquidador, las que deberán incluir, a lo menos, un desglose detallado de los ingresos y gastos durante los últimos tres meses, con observancia de la normativa contable, tributaria y financiera aplicable.

\hypertarget{artuxedculo-47.--oportunidad-y-revisiuxf3n.}{%
\paragraph*{Artículo 47.- Oportunidad y revisión.}\label{artuxedculo-47.--oportunidad-y-revisiuxf3n.}}
\addcontentsline{toc}{paragraph}{Artículo 47.- Oportunidad y revisión.}

Las cuentas provisorias deberán publicarse mensualmente en el Boletín Concursal y rendirse ante la Junta de Acreedores respectiva, la que deberá aprobarlas o rechazarlas en esa misma sesión.

A partir de la publicación señalada en el inciso anterior, los acreedores podrán formular a la Superintendencia sus observaciones a la cuenta provisoria publicada, para que ésta las incluya en el Boletín Concursal dentro del plazo de cinco días contado desde la recepción de aquellas. El Liquidador deberá responder las observaciones en la próxima Junta de Acreedores que se celebre y, a continuación, se resolverá su aprobación o rechazo.

La aprobación de la cuenta provisoria por la Junta de Acreedores no impedirá, en su caso, objetar la Cuenta Final de Administración, respecto de las partidas incluidas en ella.

\hypertarget{artuxedculo-48.--no-celebraciuxf3n-de-la-junta-de-acreedores.}{%
\paragraph*{Artículo 48.- No celebración de la Junta de Acreedores.}\label{artuxedculo-48.--no-celebraciuxf3n-de-la-junta-de-acreedores.}}
\addcontentsline{toc}{paragraph}{Artículo 48.- No celebración de la Junta de Acreedores.}

Si la Junta de Acreedores no se celebra por falta de quórum, el Liquidador notificará dicha circunstancia en el Boletín Concursal dentro del plazo de dos días.

\hypertarget{puxe1rrafo-2.-de-la-cuenta-final-de-administraciuxf3n}{%
\subsubsection*{Párrafo 2. De la Cuenta Final de Administración}\label{puxe1rrafo-2.-de-la-cuenta-final-de-administraciuxf3n}}
\addcontentsline{toc}{subsubsection}{Párrafo 2. De la Cuenta Final de Administración}

\hypertarget{artuxedculo-49.--contenido.}{%
\paragraph*{Artículo 49.- Contenido.}\label{artuxedculo-49.--contenido.}}
\addcontentsline{toc}{paragraph}{Artículo 49.- Contenido.}

La Superintendencia fijará la forma y contenidos obligatorios de la Cuenta Final de Administración mediante norma de carácter general, con observancia de la normativa contable, tributaria y financiera aplicable.

\hypertarget{artuxedculo-50.--oportunidad.}{%
\paragraph*{Artículo 50.- Oportunidad.}\label{artuxedculo-50.--oportunidad.}}
\addcontentsline{toc}{paragraph}{Artículo 50.- Oportunidad.}

El Liquidador deberá acompañar al Tribunal y a la Superintendencia su Cuenta Final de Administración dentro de los treinta días siguientes a que se verifique cualquiera de las circunstancias que a continuación se señalan:

\begin{enumerate}
\def\labelenumi{\arabic{enumi})}
\item
  Vencimiento de los plazos legales de realización de bienes.
\item
  Agotamiento de los fondos o pago íntegro de los créditos reconocidos.
\item
  Cese anticipado de su cargo.
\end{enumerate}

\hypertarget{artuxedculo-51.--rendiciuxf3n-de-la-cuenta.}{%
\paragraph*{Artículo 51.- Rendición de la Cuenta.}\label{artuxedculo-51.--rendiciuxf3n-de-la-cuenta.}}
\addcontentsline{toc}{paragraph}{Artículo 51.- Rendición de la Cuenta.}

Una vez acompañada su Cuenta Final de Administración al tribunal competente y a la Superintendencia, el Liquidador deberá citar a Junta de Acreedores a efectos de rendirla, explicar su contenido, las conclusiones y acreditar la retención del porcentaje de honorarios a percibir de conformidad a lo dispuesto en el número 7) del artículo 39. La Superintendencia podrá concurrir a dicha Junta con derecho a voz.

La citación deberá publicarse en el Boletín Concursal dentro de los cinco días siguientes a la resolución que tuvo por acompañada la Cuenta Final de Administración ante el Tribunal, e incluirá el día, hora y lugar en que se celebrará la Junta de Acreedores. Entre la fecha de publicación de la citación y de celebración de la Junta de Acreedores deberán transcurrir no menos de diez ni más de veinticinco días. La citación incluirá también una copia de la Cuenta Final de Administración.

Dicha Junta se celebrará con los acreedores que asistan.

\hypertarget{artuxedculo-52.--de-la-objeciuxf3n.}{%
\paragraph*{Artículo 52.- De la objeción.}\label{artuxedculo-52.--de-la-objeciuxf3n.}}
\addcontentsline{toc}{paragraph}{Artículo 52.- De la objeción.}

Podrán objetar la Cuenta Final de Administración del Liquidador el Deudor, cualquier acreedor y la Superintendencia.

Las objeciones se presentarán ante la Superintendencia dentro de los cinco días siguientes a la fecha en que se celebró o debió celebrarse la respectiva Junta de Acreedores. Si el objetante fuese la Superintendencia, su objeción será publicada en el Boletín Concursal en el mismo plazo antes señalado.

En caso de no deducirse objeciones oportunamente, el Liquidador o la Superintendencia solicitarán al tribunal competente que tenga por aprobada la Cuenta Final de Administración para todos los efectos legales.

Si se presentaren objeciones, se observarán las normas que siguen:

\begin{enumerate}
\def\labelenumi{\arabic{enumi})}
\item
  Una vez vencido el plazo señalado en el inciso segundo, la Superintendencia requerirá informe al Liquidador de todas las objeciones presentadas o publicadas en una única resolución, la que se notificará por correo electrónico al Liquidador y se publicará en el Boletín Concursal.
\item
  El Liquidador dispondrá de diez días contados desde la notificación de la resolución antes indicada para contestar en una sola presentación todas las objeciones planteadas. En su presentación, el Liquidador podrá incluir correcciones a la Cuenta Final de Administración objetada, caso en el cual acompañará el texto definitivo que las refleje.
\item
  Si el Liquidador no efectúa presentación alguna en el plazo antes indicado, se entenderá suspendido de pleno derecho para asumir en los procedimientos regidos por esta ley mientras la o las objeciones no sean resueltas.
\item
  Transcurrido el plazo señalado en el número 2), se haya presentado o no el informe del Liquidador, los objetantes dispondrán de tres días para insistir en sus objeciones.
\item
  Si no se presentaren insistencias, se tendrá por aprobada la Cuenta Final de Administración.
\item
  En caso de insistencia, la Superintendencia remitirá al tribunal competente, dentro del plazo de diez días, un informe que contendrá las objeciones planteadas, la contestación del Liquidador si la hubiere, y su opinión en cuanto a si los hechos afectan decisivamente el patrimonio concursado, si implican un grave perjuicio para los acreedores y/o el Deudor, o si reflejan una manifiesta e inexcusable inobservancia del Liquidador a su deber de cuidado. El referido informe establecerá si el Liquidador quedará suspendido para asumir en nuevos Procedimientos Concursales.
\item
  El tribunal competente apreciará los antecedentes aportados de acuerdo a las normas de la sana crítica y pronunciará su resolución dentro de los quince días siguientes a la entrega del informe que indica el número anterior.
\item
  Si la resolución desechare en todas sus partes la o las objeciones deducidas, condenará al o los objetantes en costas, quienes responderán solidariamente de ellas, salvo que el tribunal competente estime que hubo motivo plausible para litigar. La misma regla se aplicará en caso que la resolución rechace una o más objeciones y acoja otras, respondiendo solidariamente todas las partes vencidas de la condena en costas. Tratándose del Deudor, responderán solidariamente de esas costas su abogado patrocinante y sus mandatarios judiciales.
\item
  La resolución del tribunal competente que acoja una o más objeciones insistidas, señalará las medidas que el Liquidador deberá ejecutar para subsanar, reparar o corregir los defectos advertidos. Si la resolución rechaza la Cuenta Final de Administración deberá designar al Liquidador suplente como titular, de acuerdo con lo establecido en el inciso segundo del artículo 38.
\end{enumerate}

Contra esta resolución procederá el recurso de apelación, el que se concederá en el solo efecto devolutivo. Contra la resolución que rechace una o más objeciones insistidas no procederá recurso alguno.

Una vez firme la sentencia que rechace la Cuenta Final de Administración, la Superintendencia excluirá al Liquidador de la Nómina de Liquidadores, de conformidad a lo establecido en el artículo 34 de esta ley.

\hypertarget{artuxedculo-53.--ejecuciuxf3n-de-las-resoluciones-que-rechazan-la-cuenta-final-de-administraciuxf3n.}{%
\paragraph*{Artículo 53.- Ejecución de las resoluciones que rechazan la Cuenta Final de Administración.}\label{artuxedculo-53.--ejecuciuxf3n-de-las-resoluciones-que-rechazan-la-cuenta-final-de-administraciuxf3n.}}
\addcontentsline{toc}{paragraph}{Artículo 53.- Ejecución de las resoluciones que rechazan la Cuenta Final de Administración.}

La ejecución de estas resoluciones se sujetará a las siguientes reglas:

\begin{enumerate}
\def\labelenumi{\arabic{enumi})}
\tightlist
\item
  Si la resolución ordena al Liquidador a quien se le rechazó la Cuenta restituir a la masa una suma de dinero, se procederá de la siguiente forma:
\end{enumerate}

\begin{enumerate}
\def\labelenumi{\alph{enumi})}
\item
  Tendrá el plazo de treinta días, prorrogable por igual período, desde que la resolución se encuentra firme y ejecutoriada para dar cumplimiento a lo resuelto.
\item
  Si no efectuare la restitución señalada, el tribunal competente certificará esa omisión, de oficio o a petición de parte, y comunicará tal circunstancia a la Superintendencia.
\item
  Con esa certificación, la Superintendencia hará efectiva la garantía de fiel desempeño referida en el artículo 16 de esta ley, consignando los fondos en el tribunal competente.
\end{enumerate}

\begin{enumerate}
\def\labelenumi{\arabic{enumi})}
\setcounter{enumi}{1}
\tightlist
\item
  Si la resolución ordena al Liquidador cuya Cuenta se rechazó una medida distinta a la de restituir a la masa una suma de dinero, se procederá de la siguiente manera:
\end{enumerate}

\begin{enumerate}
\def\labelenumi{\alph{enumi})}
\item
  El Liquidador cuya Cuenta se rechazó ejecutará lo resuelto dentro del mismo plazo indicado en el número anterior o en aquél que fije el tribunal en su resolución.
\item
  El honorario del nuevo Liquidador designado se determinará de común acuerdo con la Junta de Acreedores o, en su defecto, por el tribunal competente, y se pagará de acuerdo a lo establecido en el número 6 del artículo 39.
\end{enumerate}

En todos los casos señalados en este artículo, el Liquidador cuya cuenta se rechazó podrá solicitar una prórroga ante el tribunal competente, por una sola vez y por un máximo de treinta días, para dar cumplimiento a lo resuelto.

\hypertarget{capuxedtulo-iii-del-procedimiento-concursal-de-reorganizaciuxf3n}{%
\section*{CAPÍTULO III: DEL PROCEDIMIENTO CONCURSAL DE REORGANIZACIÓN}\label{capuxedtulo-iii-del-procedimiento-concursal-de-reorganizaciuxf3n}}
\addcontentsline{toc}{section}{CAPÍTULO III: DEL PROCEDIMIENTO CONCURSAL DE REORGANIZACIÓN}

\hypertarget{tuxedtulo-1.-del-inicio-del-procedimiento-concursal-de-reorganizaciuxf3n-judicial}{%
\subsection*{Título 1. Del inicio del Procedimiento Concursal de Reorganización Judicial}\label{tuxedtulo-1.-del-inicio-del-procedimiento-concursal-de-reorganizaciuxf3n-judicial}}
\addcontentsline{toc}{subsection}{Título 1. Del inicio del Procedimiento Concursal de Reorganización Judicial}

\hypertarget{artuxedculo-54.--uxe1mbito-de-aplicaciuxf3n-e-inicio-del-procedimiento-concursal-de-reorganizaciuxf3n-judicial.}{%
\paragraph*{Artículo 54.- Ámbito de aplicación e inicio del Procedimiento Concursal de Reorganización Judicial.}\label{artuxedculo-54.--uxe1mbito-de-aplicaciuxf3n-e-inicio-del-procedimiento-concursal-de-reorganizaciuxf3n-judicial.}}
\addcontentsline{toc}{paragraph}{Artículo 54.- Ámbito de aplicación e inicio del Procedimiento Concursal de Reorganización Judicial.}

El Procedimiento Concursal de Reorganización Judicial será aplicable sólo a la Empresa Deudora, que para efectos de este Capítulo se denominará indistintamente Empresa Deudora o Deudor.

El Procedimiento Concursal de Reorganización se iniciará mediante la presentación de una solicitud por la Empresa Deudora ante el tribunal correspondiente a su domicilio.

Un modelo de dicha solicitud se regulará por la Superintendencia mediante una norma de carácter general, que estará disponible en sus dependencias, en su sitio web y en las dependencias de los tribunales con competencia en Procedimientos Concursales de conformidad a lo establecido en el artículo 3º.

\hypertarget{artuxedculo-55.--antecedentes-para-la-nominaciuxf3n-del-veedor.}{%
\paragraph*{Artículo 55.- Antecedentes para la nominación del Veedor.}\label{artuxedculo-55.--antecedentes-para-la-nominaciuxf3n-del-veedor.}}
\addcontentsline{toc}{paragraph}{Artículo 55.- Antecedentes para la nominación del Veedor.}

Para los efectos de la nominación de los Veedores titular y suplente, el Deudor deberá presentar a la Superintendencia una copia del documento indicado en el artículo anterior, con el respectivo cargo del tribunal competente o de la Corte de Apelaciones correspondiente. Además, deberá acompañar un certificado emitido por un auditor independiente al Deudor, inscrito en el Registro de Auditores Externos de la Superintendencia de Valores y Seguros. Este certificado se extenderá conforme a la información disponible suministrada por el Deudor y deberá contener un estado de sus deudas, con expresión del nombre, domicilio y correo electrónico de los acreedores o de sus representantes legales, en su caso; de la naturaleza de los respectivos títulos, y del monto de sus créditos, indicando el porcentaje que cada uno representa en el total del pasivo, con expresión de los tres mayores acreedores, excluidas las Personas Relacionadas al Deudor. La nominación de los Veedores titular y suplente se realizará según el procedimiento establecido en el artículo 22 y, una vez concluido, la Superintendencia extenderá el respectivo Certificado de Nominación contemplado en dicha disposición.

\hypertarget{artuxedculo-56.--antecedentes-que-deberuxe1-acompauxf1ar-el-deudor.}{%
\paragraph*{Artículo 56.- Antecedentes que deberá acompañar el Deudor.}\label{artuxedculo-56.--antecedentes-que-deberuxe1-acompauxf1ar-el-deudor.}}
\addcontentsline{toc}{paragraph}{Artículo 56.- Antecedentes que deberá acompañar el Deudor.}

Aceptada la nominación por el Veedor titular y suplente, la Superintendencia remitirá al tribunal competente el Certificado de Nominación correspondiente. Paralelamente, el Deudor acompañará lo siguiente:

\begin{enumerate}
\def\labelenumi{\arabic{enumi})}
\item
  Relación de todos sus bienes, con expresión de su avalúo comercial, del lugar en que se encuentren y de los gravámenes que los afecten. Deberá señalar, además, cuáles de estos bienes tienen la calidad de esenciales para el giro de la Empresa Deudora;
\item
  Relación de todos aquellos bienes de terceros constituidos en garantía en favor del Deudor. Deberá señalar, además, cuáles de estos bienes tienen la calidad de esenciales para el giro de la Empresa Deudora;
\item
  Relación de todos aquellos bienes que se encuentren en poder del Deudor en una calidad distinta a la de dueño;
\item
  El certificado a que hace referencia el artículo 55, para la determinación del pasivo afecto a los Acuerdos de Reorganización Judicial. El pasivo que se establezca en este certificado deberá considerar el estado de deudas del Deudor, con una fecha de cierre no superior a cuarenta y cinco días anteriores a esta presentación, con indicación expresa de los créditos que se encuentren garantizados con prenda o hipoteca y el avalúo comercial de los bienes sobre los que recaen las garantías. Este certificado servirá de base para determinar todos los quórum de acreedores que se necesiten en la adopción de cualquier acuerdo, hasta que se confeccione la nómina de créditos reconocidos, conforme al procedimiento establecido en el Párrafo 2 del Título 1 del Capítulo III de esta ley, con sus respectivas ampliaciones o modificaciones, si existieren, y
\item
  Si el Deudor llevare contabilidad completa, presentará el balance correspondiente a su último ejercicio y un balance provisorio que contenga la información financiera y contable, con una fecha de cierre no superior a cuarenta y cinco días anteriores a esta presentación.
\end{enumerate}

Si se tratare de una persona jurídica, los documentos referidos serán firmados por sus representantes legales.

\hypertarget{artuxedculo-57.--resoluciuxf3n-de-reorganizaciuxf3n.}{%
\paragraph*{Artículo 57.- Resolución de Reorganización.}\label{artuxedculo-57.--resoluciuxf3n-de-reorganizaciuxf3n.}}
\addcontentsline{toc}{paragraph}{Artículo 57.- Resolución de Reorganización.}

Dentro del quinto día de efectuada la presentación señalada en el artículo anterior, el tribunal competente dictará una resolución designando a los Veedores titular y suplente nominados en la forma establecida en el artículo 22. En la misma resolución dispondrá lo siguiente:

\begin{enumerate}
\def\labelenumi{\arabic{enumi})}
\tightlist
\item
  Que durante el plazo de treinta días contado desde la notificación de esta resolución, prorrogable de conformidad a lo dispuesto en el artículo 58, el Deudor gozará de una Protección Financiera Concursal en virtud de la cual:
\end{enumerate}

\begin{enumerate}
\def\labelenumi{\alph{enumi})}
\item
  No podrá declararse ni iniciarse en contra del Deudor un Procedimiento Concursal de Liquidación, ni podrán iniciarse en su contra juicios ejecutivos, ejecuciones de cualquier clase o restituciones en juicios de arrendamiento. Lo anterior no se aplicará a los juicios laborales sobre obligaciones que gocen de preferencia de primera clase, suspendiéndose en este caso sólo la ejecución y realización de bienes del Deudor, salvo que se trate de juicios laborales de este tipo que el Deudor tuviere en tal carácter a favor de su cónyuge, de sus parientes, o de los gerentes, administradores, apoderados con poder general de administración u otras personas que tengan injerencia en la administración de sus negocios. Para estos efectos, se entenderá por parientes del Deudor o de sus representantes legales los ascendientes, descendientes, y los colaterales hasta el cuarto grado de consanguinidad y afinidad, inclusive.
\item
  Se suspenderá la tramitación de los procedimientos señalados en la letra a) precedente y los plazos de prescripción extintiva.
\item
  Todos los contratos suscritos por el Deudor mantendrán su vigencia y condiciones de pago. En consecuencia, no podrán terminarse anticipadamente en forma unilateral, exigirse anticipadamente su cumplimiento o hacerse efectivas las garantías contratadas, invocando como causal el inicio de un Procedimiento Concursal de Reorganización. El crédito del acreedor que contraviniere esta 52
  prohibición quedará pospuesto hasta que se pague a la totalidad de los acreedores a quienes les afectare el Acuerdo de Reorganización Judicial, incluidos los acreedores Personas Relacionadas del Deudor.
\end{enumerate}

Para hacer efectiva la postergación señalada en el inciso anterior, deberá solicitarse su declaración en forma incidental ante el tribunal que conoce del Procedimiento Concursal de Reorganización.

\begin{enumerate}
\def\labelenumi{\alph{enumi})}
\setcounter{enumi}{3}
\tightlist
\item
  Si el Deudor formare parte de algún registro público como contratista o prestador de cualquier servicio, y siempre que se encuentre al día en sus obligaciones contractuales con el respectivo mandante, no podrá ser eliminado ni se le privará de participar en procesos de licitación fundado en el inicio de un Procedimiento Concursal de Reorganización. Si la entidad pública lo elimina de sus registros o discrimina su participación, fundado en la apertura de un Procedimiento Concursal de Reorganización, a pesar de encontrarse al día en sus obligaciones con el respectivo mandante, deberá indemnizar los perjuicios que dicha discriminación o eliminación le provoquen al Deudor.
\end{enumerate}

\begin{enumerate}
\def\labelenumi{\arabic{enumi})}
\setcounter{enumi}{1}
\tightlist
\item
  Que durante la Protección Financiera Concursal se aplicarán al Deudor las siguientes medidas cautelares y de restricción:
\end{enumerate}

\begin{enumerate}
\def\labelenumi{\alph{enumi})}
\item
  Quedará sujeto a la intervención del Veedor titular designado en la misma resolución, el que tendrá los deberes contenidos en el artículo 25;
\item
  No podrá gravar o enajenar sus bienes, salvo aquellos cuya enajenación o venta sea propia de su giro o que resulten estrictamente necesarios para el normal desenvolvimiento de su actividad; y respecto de los demás bienes o activos, se estará a lo previsto en el artículo 74, y
\item
  Tratándose de personas jurídicas, éstas no podrán modificar sus pactos, estatutos sociales o régimen de poderes. La inscripción de cualquier transferencia de acciones de la Empresa Deudora en los registros sociales pertinentes requerirá la autorización del Veedor, que la extenderá en la medida que ella no altere o afecte los derechos de los acreedores. Lo anterior no regirá respecto de las sociedades anónimas abiertas que hagan oferta pública de sus valores.
\end{enumerate}

\begin{enumerate}
\def\labelenumi{\arabic{enumi})}
\setcounter{enumi}{2}
\item
  La fecha en que expirará la Protección Financiera Concursal.
\item
  La orden al Deudor para que a través del Veedor publique en el Boletín Concursal y acompañe al tribunal competente, a lo menos diez días antes de la fecha fijada para la Junta de Acreedores, su propuesta de Acuerdo de Reorganización Judicial. Si el Deudor no da cumplimiento a esta orden, el Veedor certificará esta circunstancia y el tribunal competente dictará la Resolución de Liquidación, sin más trámite.
\item
  La fecha, lugar y hora en que deberá efectuarse la Junta de Acreedores llamada a conocer y pronunciarse sobre la propuesta de Acuerdo de Reorganización Judicial que presente el Deudor. La fecha de dicha Junta será aquella en la que expire la Protección Financiera Concursal.
\item
  Que dentro de quince días contados desde la notificación de esta resolución, todos los acreedores deberán acreditar ante el tribunal competente su personería para actuar en el Procedimiento Concursal de Reorganización, con indicación expresa de la facultad que le confieren a sus apoderados para conocer, modificar y adoptar el Acuerdo de Reorganización Judicial.
\item
  La orden para que el Veedor inscriba copia de esta resolución en los conservadores de bienes raíces correspondientes al margen de la inscripción de propiedad de cada uno de los inmuebles que pertenecen al deudor.
\item
  La orden al Veedor para que acompañe al tribunal competente y publique en el Boletín Concursal su informe sobre la propuesta de Acuerdo de Reorganización Judicial, a lo menos tres días antes de la fecha fijada para la celebración de la Junta de Acreedores que votará dicho acuerdo. Este Informe del Veedor deberá contener la calificación fundada acerca de:
\end{enumerate}

\begin{enumerate}
\def\labelenumi{\alph{enumi})}
\item
  Si la propuesta es susceptible de ser cumplida, habida consideración de las condiciones del Deudor;
\item
  El monto probable de recuperación que le correspondería a cada acreedor en sus respectivas categorías, en caso de un Procedimiento Concursal de Liquidación, y
\item
  Si la propuesta de determinación de los créditos y su preferencia indicada por el Deudor se ajustan a la ley.
\end{enumerate}

Si el Veedor no presentare el referido informe dentro del plazo indicado, el Deudor, cualquiera de los acreedores o el tribunal competente informará a la Superintendencia para que se apliquen las sanciones pertinentes. En este caso, el Acuerdo de Reorganización Judicial se votará con prescindencia del Informe del Veedor.

\begin{enumerate}
\def\labelenumi{\arabic{enumi})}
\setcounter{enumi}{8}
\item
  Que dentro de quinto día de efectuada la notificación de esta resolución, deberán asistir a una audiencia el Deudor y los tres mayores acreedores indicados en la certificación del contador auditor independiente referida en el artículo 55. Esta diligencia se efectuará con los que concurran y tratará sobre la proposición de honorarios que formule el Veedor. Si en ella no se arribare a acuerdo sobre el monto de los honorarios y su forma de pago, o no asistiere ninguno de los citados, dichos honorarios se fijarán por el tribunal competente sin ulterior recurso.
\item
  La orden al Deudor para que proporcione al Veedor copia de todos los antecedentes acompañados conforme al artículo 56. Estos antecedentes y la copia de la resolución de que trata este artículo serán publicados por el Veedor en el Boletín Concursal dentro del plazo de tres días contado desde su dictación.
\end{enumerate}

\hypertarget{artuxedculo-58.--pruxf3rroga-de-la-protecciuxf3n-financiera-concursal.}{%
\paragraph*{Artículo 58.- Prórroga de la Protección Financiera Concursal.}\label{artuxedculo-58.--pruxf3rroga-de-la-protecciuxf3n-financiera-concursal.}}
\addcontentsline{toc}{paragraph}{Artículo 58.- Prórroga de la Protección Financiera Concursal.}

El plazo establecido en el número 1) del artículo anterior para la Protección Financiera Concursal podrá prorrogarse hasta por treinta días, si el Deudor obtiene el apoyo de dos o más acreedores, que representen más del 30\% del total del pasivo, excluidos los créditos de las Personas Relacionadas con el Deudor. Hasta el décimo día anterior al vencimiento del plazo antes señalado, el Deudor podrá solicitar una nueva prórroga por otros treinta días si obtiene el apoyo de dos o más acreedores que representen más del 50\% del total del pasivo, excluidos los créditos de las Personas Relacionadas con el Deudor.

Sin perjuicio de lo anterior, se podrá solicitar en un solo acto la prórroga del plazo regulado para la Protección Financiera Concursal a que se refiere el número 1) del artículo anterior hasta por sesenta días, si el Deudor obtiene el apoyo de dos o más acreedores que representen más del 50\% del total del pasivo, excluidos los créditos de las Personas Relacionadas con el Deudor.

Los acreedores hipotecarios y prendarios que presten su apoyo para la prórroga de la Protección Financiera Concursal no perderán su preferencia y podrán impetrar las medidas conservativas que procedan.

\hypertarget{artuxedculo-59.--nueva-fecha-y-hora-de-la-junta-de-acreedores-llamada-a-conocer-y-pronunciarse-sobre-la-propuesta-de-acuerdo-de-reorganizaciuxf3n-judicial.}{%
\paragraph*{Artículo 59.- Nueva fecha y hora de la Junta de Acreedores llamada a conocer y pronunciarse sobre la propuesta de Acuerdo de Reorganización Judicial.}\label{artuxedculo-59.--nueva-fecha-y-hora-de-la-junta-de-acreedores-llamada-a-conocer-y-pronunciarse-sobre-la-propuesta-de-acuerdo-de-reorganizaciuxf3n-judicial.}}
\addcontentsline{toc}{paragraph}{Artículo 59.- Nueva fecha y hora de la Junta de Acreedores llamada a conocer y pronunciarse sobre la propuesta de Acuerdo de Reorganización Judicial.}

Para lograr la prórroga regulada en el artículo anterior, el Deudor deberá presentar al tribunal competente, junto con la respectiva solicitud de prórroga, las cartas de apoyo de los acreedores autorizadas ante un ministro de fe, y un certificado extendido por un contador auditor independiente al Deudor, que indique los porcentajes del pasivo que permitan el apoyo requerido.
Acogida la prórroga de la Protección Financiera Concursal, el tribunal competente deberá fijar la nueva fecha y hora de la Junta de Acreedores llamada a conocer y pronunciarse sobre la propuesta de Acuerdo de Reorganización Judicial.

\hypertarget{puxe1rrafo-1.-del-objeto-de-la-propuesta-del-acuerdo-de-reorganizaciuxf3n-judicial}{%
\subsubsection*{Párrafo 1. Del objeto de la propuesta del Acuerdo de Reorganización Judicial}\label{puxe1rrafo-1.-del-objeto-de-la-propuesta-del-acuerdo-de-reorganizaciuxf3n-judicial}}
\addcontentsline{toc}{subsubsection}{Párrafo 1. Del objeto de la propuesta del Acuerdo de Reorganización Judicial}

\hypertarget{artuxedculo-60.--objeto-de-la-propuesta-de-acuerdo-de-reorganizaciuxf3n-judicial.}{%
\paragraph*{Artículo 60.- Objeto de la propuesta de Acuerdo de Reorganización Judicial.}\label{artuxedculo-60.--objeto-de-la-propuesta-de-acuerdo-de-reorganizaciuxf3n-judicial.}}
\addcontentsline{toc}{paragraph}{Artículo 60.- Objeto de la propuesta de Acuerdo de Reorganización Judicial.}

La propuesta podrá versar sobre cualquier objeto tendiente a reestructurar los pasivos y activos de una Empresa Deudora.

\hypertarget{artuxedculo-61.--acuerdos-de-reorganizaciuxf3n-judicial-por-clases-o-categoruxedas-de-acreedores.}{%
\paragraph*{Artículo 61.- Acuerdos de Reorganización Judicial por clases o categorías de acreedores.}\label{artuxedculo-61.--acuerdos-de-reorganizaciuxf3n-judicial-por-clases-o-categoruxedas-de-acreedores.}}
\addcontentsline{toc}{paragraph}{Artículo 61.- Acuerdos de Reorganización Judicial por clases o categorías de acreedores.}

La propuesta de Acuerdo podrá separarse en clases o categorías de acreedores y se podrá formular una propuesta para los acreedores valistas y otra para los acreedores hipotecarios y prendarios cuyos créditos se encuentren garantizados con bienes de propiedad del Deudor o de terceros. Los acreedores hipotecarios y prendarios que voten la propuesta del Acuerdo conservarán sus preferencias.

La propuesta de Acuerdo será igualitaria para todos los acreedores de una misma clase o categoría, salvo que medie acuerdo en contrario, de conformidad a lo dispuesto en los artículos 64 y siguientes.

Los acreedores hipotecarios y prendarios cuyos créditos se encuentren garantizados con bienes de propiedad del Deudor o de terceros podrán votar la propuesta de Acuerdo que se formule para acreedores valistas si renuncian a la preferencia de sus créditos y no podrán votar la propuesta de Acuerdo que se formule para la clase o categoría de los acreedores hipotecarios o prendarios, salvo que dicha renuncia sea parcial y se manifieste expresamente.

Si los acreedores hipotecarios y prendarios votan la propuesta de Acuerdo de los acreedores valistas, los montos de sus créditos preferentes se descontarán del pasivo de su clase o categoría y se incluirán en el pasivo de la clase o categoría de los acreedores valistas para efectos del cómputo a que se refiere el artículo 79 por las sumas a que hubiere alcanzado la renuncia.

\hypertarget{artuxedculo-62.--propuestas-alternativas-de-acuerdo-de-reorganizaciuxf3n-judicial.}{%
\paragraph*{Artículo 62.- Propuestas alternativas de Acuerdo de Reorganización Judicial.}\label{artuxedculo-62.--propuestas-alternativas-de-acuerdo-de-reorganizaciuxf3n-judicial.}}
\addcontentsline{toc}{paragraph}{Artículo 62.- Propuestas alternativas de Acuerdo de Reorganización Judicial.}

En cada una de sus clases o categorías, la propuesta de Acuerdo podrá contener una proposición principal y otras alternativas para todos los acreedores de la misma clase o categoría, en cuyo caso éstos deberán optar por regirse por alguna de ellas, dentro de los diez días siguientes a la fecha de la Junta de Acreedores llamada a conocer y pronunciarse sobre la propuesta de Acuerdo.

\hypertarget{artuxedculo-63.--posposiciuxf3n-del-pago-a-acreedores-personas-relacionadas.}{%
\paragraph*{Artículo 63.- Posposición del pago a acreedores Personas Relacionadas.}\label{artuxedculo-63.--posposiciuxf3n-del-pago-a-acreedores-personas-relacionadas.}}
\addcontentsline{toc}{paragraph}{Artículo 63.- Posposición del pago a acreedores Personas Relacionadas.}

Los acreedores Personas Relacionadas con el Deudor, cuyos créditos no se encuentren debidamente documentados 90 días antes del inicio del Procedimiento Concursal de Reorganización, quedarán pospuestos en el pago de sus créditos, hasta que se paguen íntegramente los créditos de los demás acreedores a los que les afectará el Acuerdo de Reorganización Judicial. Sin perjuicio de lo anterior, el Acuerdo podrá hacer aplicable la referida posposición a otros acreedores Personas Relacionadas con el Deudor, cuyos créditos se encuentren debidamente documentados, previo informe fundado del Veedor. Esta posposición no regirá respecto de los créditos que se originen en virtud de los artículos 72 y 73.

Tampoco regirá respecto de los créditos que se originen en virtud del artículo 74, en la medida que se autorice por los acreedores que representen más del 50\% del pasivo del Deudor.

\hypertarget{artuxedculo-64.--diferencias-entre-acreedores-de-igual-clase-o-categoruxeda.}{%
\paragraph*{Artículo 64.- Diferencias entre acreedores de igual clase o categoría.}\label{artuxedculo-64.--diferencias-entre-acreedores-de-igual-clase-o-categoruxeda.}}
\addcontentsline{toc}{paragraph}{Artículo 64.- Diferencias entre acreedores de igual clase o categoría.}

En las propuestas de Acuerdo de Reorganización Judicial se podrán establecer condiciones más favorables para algunos de los acreedores de una misma clase o categoría, siempre que los demás acreedores de la respectiva clase o categoría lo acuerden con Quórum Especial, el cual se calculará únicamente sobre el monto de los créditos de estos últimos.

\hypertarget{artuxedculo-65.--constituciuxf3n-de-garantuxedas-en-los-acuerdos-de-reorganizaciuxf3n-judicial.}{%
\paragraph*{Artículo 65.- Constitución de garantías en los Acuerdos de Reorganización Judicial.}\label{artuxedculo-65.--constituciuxf3n-de-garantuxedas-en-los-acuerdos-de-reorganizaciuxf3n-judicial.}}
\addcontentsline{toc}{paragraph}{Artículo 65.- Constitución de garantías en los Acuerdos de Reorganización Judicial.}

En los Acuerdos podrá estipularse la constitución de garantías para asegurar el cumplimiento de las obligaciones del Deudor. Estas garantías podrán constituirse en el mismo Acuerdo o en instrumentos separados.

Para estos efectos, los acreedores podrán designar a uno o más de ellos para que los representen en la celebración de los actos que sean necesarios para la debida constitución de las garantías.

\hypertarget{artuxedculo-66.--acreedores-comprendidos-en-los-acuerdos-de-reorganizaciuxf3n-judicial.}{%
\paragraph*{Artículo 66.- Acreedores comprendidos en los Acuerdos de Reorganización Judicial.}\label{artuxedculo-66.--acreedores-comprendidos-en-los-acuerdos-de-reorganizaciuxf3n-judicial.}}
\addcontentsline{toc}{paragraph}{Artículo 66.- Acreedores comprendidos en los Acuerdos de Reorganización Judicial.}

Los Acuerdos sólo afectarán a los acreedores cuyos créditos se originen con anterioridad a la Resolución de Reorganización regulada en el artículo 57.
Los créditos que se originen con posterioridad no serán incluidos en el Acuerdo de Reorganización Judicial.

\hypertarget{artuxedculo-67.--prohibiciuxf3n-de-repartos.}{%
\paragraph*{Artículo 67.- Prohibición de repartos.}\label{artuxedculo-67.--prohibiciuxf3n-de-repartos.}}
\addcontentsline{toc}{paragraph}{Artículo 67.- Prohibición de repartos.}

Se prohíbe a la Empresa Deudora repartir sumas a sus accionistas o socios, bajo ningún concepto, ni directa ni indirectamente, sea por la vía de reducción de capital, condonación de préstamos otorgados y/o repartos de dividendos antes de haber pagado el 100\% de las obligaciones emanadas del Acuerdo de Reorganización Judicial, salvo que los acreedores expresamente lo autoricen en la forma que lo determine el Acuerdo.

\hypertarget{artuxedculo-68.--cluxe1usula-arbitral-en-acuerdos-de-reorganizaciuxf3n-judicial.}{%
\paragraph*{Artículo 68.- Cláusula arbitral en Acuerdos de Reorganización Judicial.}\label{artuxedculo-68.--cluxe1usula-arbitral-en-acuerdos-de-reorganizaciuxf3n-judicial.}}
\addcontentsline{toc}{paragraph}{Artículo 68.- Cláusula arbitral en Acuerdos de Reorganización Judicial.}

En cualquiera de las clases o categorías de un Acuerdo de Reorganización Judicial podrá estipularse una cláusula arbitral, en cuyo caso las diferencias que se produzcan entre el Deudor y uno o más acreedores o entre éstos, con motivo de la aplicación, interpretación, cumplimiento, terminación o declaración de incumplimiento del Acuerdo, se someterán a arbitraje. Éste será obligatorio para todos los acreedores a los que afecte el referido Acuerdo.
Si el árbitro declara la terminación o el incumplimiento del Acuerdo, remitirá de inmediato el expediente al tribunal competente para que éste dicte la Resolución de Liquidación en conformidad a esta ley.

\hypertarget{artuxedculo-69.--interventor-y-comisiuxf3n-de-acreedores.}{%
\paragraph*{Artículo 69.- Interventor y Comisión de Acreedores.}\label{artuxedculo-69.--interventor-y-comisiuxf3n-de-acreedores.}}
\addcontentsline{toc}{paragraph}{Artículo 69.- Interventor y Comisión de Acreedores.}

El Acuerdo de Reorganización Judicial deberá estipular el nombramiento de un interventor por al menos un año contado desde el Acuerdo, el que recaerá en un Veedor vigente de la Nómina de Veedores. El interventor nombrado tendrá las atribuciones, deberes y remuneración que el mismo Acuerdo señale. Si ellas no se especifican, se entenderá que tendrá las señaladas en el artículo 294 del Código de Procedimiento Civil.

El Veedor tendrá la obligación de poner en conocimiento, de forma fundada, el incumplimiento del Acuerdo a la Superintendencia y a los acreedores que les afecte, mediante notificación por Correo Electrónico.

Sin perjuicio de lo anterior, en el Acuerdo de Reorganización Judicial podrá designarse a una Comisión de Acreedores para supervigilar el cumplimiento de sus estipulaciones, con las atribuciones, deberes y remuneración que, en su caso, señale el Acuerdo.

\hypertarget{puxe1rrafo-2.-de-la-determinaciuxf3n-del-pasivo}{%
\subsubsection*{Párrafo 2. De la determinación del pasivo}\label{puxe1rrafo-2.-de-la-determinaciuxf3n-del-pasivo}}
\addcontentsline{toc}{subsubsection}{Párrafo 2. De la determinación del pasivo}

\hypertarget{artuxedculo-70.--verificaciuxf3n-y-objeciuxf3n-de-los-cruxe9ditos.}{%
\paragraph*{Artículo 70.- Verificación y objeción de los créditos.}\label{artuxedculo-70.--verificaciuxf3n-y-objeciuxf3n-de-los-cruxe9ditos.}}
\addcontentsline{toc}{paragraph}{Artículo 70.- Verificación y objeción de los créditos.}

Los acreedores tendrán un plazo de ocho días contado desde la notificación de la Resolución de Reorganización a que se refiere el artículo 57 para verificar sus créditos ante el tribunal que conoce del procedimiento. Con tal propósito, deberán acompañar los títulos justificativos de éstos, señalando, en su caso, si se encuentran garantizados con prenda o hipoteca y el avalúo comercial de los bienes sobre los que recaen las garantías. No será necesaria verificación alguna si los créditos y el avalúo comercial de las garantías se encontraren señaladas, a satisfacción del acreedor, en el estado de deudas a que se refiere el número 4) del artículo 56 publicado en el Boletín Concursal.

Vencido el plazo señalado en el inciso anterior y dentro de los dos días siguientes, el Veedor publicará en el Boletín Concursal todas las verificaciones presentadas, indicando los créditos que se encuentren garantizados con prenda o hipoteca y el avalúo comercial de los bienes sobre los que recaen las garantías.

En el plazo de ocho días siguientes a la publicación indicada en el inciso precedente, el Veedor, el Deudor y los acreedores podrán deducir objeción fundada sobre la falta de títulos justificativos de los créditos, sus montos, preferencias o sobre el avalúo comercial de los bienes sobre los que recaen las garantías, que se indican en el estado de deudas que presenta el Deudor, de conformidad al número 4) del artículo 56 o en las verificaciones presentadas por los acreedores.

Los interesados presentarán sus objeciones ante el tribunal. Vencido el plazo indicado en el inciso precedente, y dentro de los dos días siguientes, el Veedor publicará en el Boletín Concursal todas las objeciones presentadas. Asimismo, expirado el plazo que se señala en el citado inciso anterior sin que se formulen objeciones, los créditos y el avalúo comercial de los bienes sobre los que recaen las garantías no objetados, quedarán reconocidos.

El Veedor confeccionará la nómina de los créditos reconocidos, la que deberá indicar los montos de los créditos, si éstos se encuentran garantizados con prenda o hipoteca y el avalúo comercial de los bienes sobre los que recaen las garantías, acompañándola al expediente dentro de quinto día de expirado el plazo para objetar y la publicará en el Boletín Concursal, sirviendo ésta como única nómina para la votación a que se refiere el artículo 78, sin perjuicio de su posterior ampliación o modificación de acuerdo al artículo siguiente.

\hypertarget{artuxedculo-71.--impugnaciuxf3n-de-cruxe9ditos.}{%
\paragraph*{Artículo 71.- Impugnación de créditos.}\label{artuxedculo-71.--impugnaciuxf3n-de-cruxe9ditos.}}
\addcontentsline{toc}{paragraph}{Artículo 71.- Impugnación de créditos.}

Si se formulan objeciones, el Veedor arbitrará las medidas necesarias para subsanarlas. Si no se subsanan, los créditos y el avalúo comercial de los bienes sobre los que recaen las garantías que fueren objeto de dichas objeciones se considerarán impugnados, y el Veedor los acumulará, emitirá un informe acerca de si existen o no fundamentos plausibles para ser considerados por el tribunal competente, y se pronunciará fundadamente sobre el avalúo comercial del bien sobre el que recae la garantía objetada.

El Veedor acompañará al tribunal competente la nómina de créditos impugnados con su respectivo informe y la nómina de créditos reconocidos indicada en el artículo 70, y las publicará en Boletín Concursal dentro de los cinco días siguientes a la expiración del plazo previsto para objetar que se señala en el inciso primero del artículo anterior.

Agregados al expediente los antecedentes que señala el inciso anterior, el tribunal citará a una audiencia única y verbal para el fallo de las impugnaciones. Dicha audiencia se celebrará dentro de tercero día contado desde la notificación de la resolución que tiene por acompañada la nómina de créditos reconocidos e impugnados.

A la audiencia podrán concurrir el Veedor, el Deudor, los impugnantes y los impugnados. En ésta deberán resolverse las incidencias que promuevan las partes en relación a las impugnaciones. El tribunal competente podrá, si fuere estrictamente necesario, suspender y continuar la referida audiencia con posterioridad. Con todo, la resolución que se pronuncie sobre las impugnaciones deberá dictarse a más tardar el segundo día anterior a la fecha de celebración de la Junta de Acreedores llamada a conocer y pronunciarse sobre la propuesta de Acuerdo.

La resolución que falle las impugnaciones ordenará la incorporación o modificación de créditos en la nómina de créditos reconocidos, o la modificación del avalúo comercial de los bienes sobre los que recaen las garantías, cuando corresponda, y será apelable en el sólo efecto devolutivo. El Veedor deberá publicar la nómina de créditos reconocidos según la resolución anterior en el Boletín Concursal, a más tardar el día anterior a la fecha de celebración de la Junta de Acreedores llamada a conocer y pronunciarse sobre la propuesta de Acuerdo.

\hypertarget{puxe1rrafo-3.-de-la-continuidad-del-suministro-de-la-venta-de-activos-y-de-los-nuevos-recursos-durante-la-protecciuxf3n-financiera-concursal}{%
\subsubsection*{Párrafo 3. De la continuidad del suministro, de la venta de activos y de los nuevos recursos durante la Protección Financiera Concursal}\label{puxe1rrafo-3.-de-la-continuidad-del-suministro-de-la-venta-de-activos-y-de-los-nuevos-recursos-durante-la-protecciuxf3n-financiera-concursal}}
\addcontentsline{toc}{subsubsection}{Párrafo 3. De la continuidad del suministro, de la venta de activos y de los nuevos recursos durante la Protección Financiera Concursal}

\hypertarget{artuxedculo-72.--continuidad-del-suministro.}{%
\paragraph*{Artículo 72.- Continuidad del suministro.}\label{artuxedculo-72.--continuidad-del-suministro.}}
\addcontentsline{toc}{paragraph}{Artículo 72.- Continuidad del suministro.}

Los proveedores de bienes y servicios que sean necesarios para el funcionamiento de la Empresa Deudora, cuyas facturas tengan como fecha de emisión no menos de ocho días anteriores a la fecha de la Resolución de Reorganización y en la medida que en su conjunto no superen el 20\% del pasivo señalado en la certificación contable referida en el artículo 55, se pagarán preferentemente en las fechas originalmente convenidas, siempre que el respectivo proveedor mantenga el suministro a la Empresa Deudora, circunstancia que deberá acreditar el Veedor.

En caso de no suscribirse el Acuerdo y, en consecuencia, se dictare la Resolución de Liquidación de la Empresa Deudora, los créditos provenientes de este suministro se pagarán con la preferencia establecida en el número 4 del artículo 2472 del Código Civil.

\hypertarget{artuxedculo-73.--operaciones-de-comercio-exterior.}{%
\paragraph*{Artículo 73.- Operaciones de comercio exterior.}\label{artuxedculo-73.--operaciones-de-comercio-exterior.}}
\addcontentsline{toc}{paragraph}{Artículo 73.- Operaciones de comercio exterior.}

Los que financien operaciones de comercio exterior de la Empresa Deudora se pagarán preferentemente en las fechas originalmente convenidas, siempre que esos acreedores mantengan las líneas de financiamiento u otorguen nuevos créditos para este tipo de operaciones, circunstancia que deberá acreditar el Veedor.

En caso que no se suscribiere el Acuerdo de Reorganización Judicial y, en consecuencia, se dictare la Resolución de Liquidación de la Empresa Deudora, los créditos provenientes de estas operaciones de comercio exterior se pagarán con la preferencia establecida en el número 4 del artículo 2472 del Código Civil.

\hypertarget{artuxedculo-74.--venta-de-activos-y-contrataciuxf3n-de-pruxe9stamos-durante-la-protecciuxf3n-financiera-concursal.}{%
\paragraph*{Artículo 74.- Venta de activos y contratación de préstamos durante la Protección Financiera Concursal.}\label{artuxedculo-74.--venta-de-activos-y-contrataciuxf3n-de-pruxe9stamos-durante-la-protecciuxf3n-financiera-concursal.}}
\addcontentsline{toc}{paragraph}{Artículo 74.- Venta de activos y contratación de préstamos durante la Protección Financiera Concursal.}

Durante la Protección Financiera Concursal, la Empresa Deudora podrá vender o enajenar activos cuyo valor no exceda el 20\% de su activo fijo contable, y podrá adquirir préstamos para el financiamiento de sus operaciones, siempre que éstos no superen el 20\% de su pasivo señalado en la certificación contable referida en el artículo 55.

La venta, enajenación o contratación de préstamos que excedan los montos señalados en el inciso anterior, así como toda operación con Personas Relacionadas con la Empresa Deudora, requerirá la autorización de los acreedores que representen más del 50\% del pasivo del Deudor.

Los préstamos contratados por la Empresa Deudora en virtud de este artículo no se considerarán en las nóminas de créditos y se pagarán preferentemente en las fechas convenidas, siempre que se utilicen para el financiamiento de sus operaciones, circunstancia que deberá acreditar el Veedor.

En caso de no suscribirse el Acuerdo y, en consecuencia, se dictare la Resolución de Liquidación de la Empresa Deudora, estos préstamos se pagarán con la preferencia establecida en el número 4 del artículo 2472 del Código Civil.

\hypertarget{artuxedculo-75.--venta-de-bienes-otorgados-en-prenda-o-hipoteca-durante-la-protecciuxf3n-financiera-concursal.}{%
\paragraph*{Artículo 75.- Venta de bienes otorgados en prenda o hipoteca durante la Protección Financiera Concursal.}\label{artuxedculo-75.--venta-de-bienes-otorgados-en-prenda-o-hipoteca-durante-la-protecciuxf3n-financiera-concursal.}}
\addcontentsline{toc}{paragraph}{Artículo 75.- Venta de bienes otorgados en prenda o hipoteca durante la Protección Financiera Concursal.}

En caso que no se acuerde la reorganización y se declare la liquidación de la Empresa Deudora, el acreedor prendario o hipotecario que autorice la enajenación de los bienes otorgados en prenda o hipoteca cuyo valor comercial exceda el monto del respectivo crédito garantizado, podrá percibir de la venta el monto de su respectivo crédito. Lo anterior procederá siempre que se garantice el pago de los créditos de primera clase, mediante el otorgamiento de cualquier instrumento de garantía que reconozcan las leyes vigentes o que la Superintendencia autorice mediante una norma de carácter general.

\hypertarget{artuxedculo-76.--valorizaciuxf3n-de-activos-y-fiscalizaciuxf3n-de-recursos.}{%
\paragraph*{Artículo 76.- Valorización de activos y fiscalización de recursos.}\label{artuxedculo-76.--valorizaciuxf3n-de-activos-y-fiscalizaciuxf3n-de-recursos.}}
\addcontentsline{toc}{paragraph}{Artículo 76.- Valorización de activos y fiscalización de recursos.}

Para efectos de determinar el valor de los activos a vender o enajenar, se estará a la valorización que realice el Veedor.

El Veedor verificará que el producto de todos los actos o contratos que se otorguen o suscriban con motivo de las operaciones que se regulan en el presente Párrafo, ingrese efectivamente a la caja de la Empresa Deudora y se destine única y exclusivamente a financiar su giro. A estos actos o contratos no les será aplicable lo dispuesto en el Capítulo VI de esta ley.

\hypertarget{tuxedtulo-2.-de-la-propuesta-de-acuerdo-de-reorganizaciuxf3n-judicial}{%
\subsection*{Título 2. De la propuesta de Acuerdo de Reorganización Judicial}\label{tuxedtulo-2.-de-la-propuesta-de-acuerdo-de-reorganizaciuxf3n-judicial}}
\addcontentsline{toc}{subsection}{Título 2. De la propuesta de Acuerdo de Reorganización Judicial}

\hypertarget{puxe1rrafo-1.-de-las-normas-generales}{%
\subsubsection*{Párrafo 1. De las normas generales}\label{puxe1rrafo-1.-de-las-normas-generales}}
\addcontentsline{toc}{subsubsection}{Párrafo 1. De las normas generales}

\hypertarget{artuxedculo-77.--efectos-del-retiro-del-acuerdo.}{%
\paragraph*{Artículo 77.- Efectos del retiro del Acuerdo.}\label{artuxedculo-77.--efectos-del-retiro-del-acuerdo.}}
\addcontentsline{toc}{paragraph}{Artículo 77.- Efectos del retiro del Acuerdo.}

Una vez notificada la propuesta de Acuerdo, ésta no podrá ser retirada por el Deudor, salvo que cuente con el apoyo de acreedores que representen a lo menos el 75\% del pasivo.

Si la propuesta de Acuerdo es retirada por el Deudor sin contar con el apoyo referido en el inciso anterior, el tribunal competente dictará la Resolución de Liquidación.

\hypertarget{artuxedculo-78.--acreedores-con-derecho-a-voto.}{%
\paragraph*{Artículo 78.- Acreedores con derecho a voto.}\label{artuxedculo-78.--acreedores-con-derecho-a-voto.}}
\addcontentsline{toc}{paragraph}{Artículo 78.- Acreedores con derecho a voto.}

Sólo tienen derecho a concurrir y votar los acreedores cuyos créditos se encuentren en la nómina de créditos reconocidos a que se refiere el artículo 70 y aquellos que figuren en la ampliación de esta nómina, de acuerdo a lo previsto en el artículo 71. En ambos casos deberá darse cumplimiento a lo ordenado en el número 6) del artículo 57, relativo a la acreditación de personerías.

Los acreedores cuyos créditos se encuentren garantizados con prenda o hipoteca votarán de acuerdo al avalúo comercial de los bienes sobre los que recaen las garantías, conforme conste en la nómina de créditos reconocidos y en su ampliación o modificación, en su caso.

Cuando el avalúo comercial de los bienes sobre los que recaen las garantías exceda el valor del crédito que garantizan, el acreedor correspondiente votará de acuerdo al monto de su crédito, según conste en la nómina de créditos reconocidos y en su ampliación o modificación, en su caso.

\hypertarget{artuxedculo-79.--acuerdo-de-la-junta-de-acreedores-llamada-a-conocer-y-pronunciarse-sobre-la-propuesta.}{%
\paragraph*{Artículo 79.- Acuerdo de la Junta de Acreedores llamada a conocer y pronunciarse sobre la propuesta.}\label{artuxedculo-79.--acuerdo-de-la-junta-de-acreedores-llamada-a-conocer-y-pronunciarse-sobre-la-propuesta.}}
\addcontentsline{toc}{paragraph}{Artículo 79.- Acuerdo de la Junta de Acreedores llamada a conocer y pronunciarse sobre la propuesta.}

Cada una de las clases o categorías de propuestas de Acuerdo que establece el artículo 61 será analizada, deliberada y acordada en forma separada en la misma junta, pudiendo proponerse modificaciones, sin perjuicio de lo previsto en el artículo 82.

La propuesta se entenderá acordada cuando cuente con el consentimiento del Deudor y el voto conforme de los dos tercios o más de los acreedores presentes, que representen al menos dos tercios del total del pasivo con derecho a voto correspondiente a su respectiva clase o categoría.

No podrán votar las Personas Relacionadas con el Deudor y sus créditos no se considerarán en el pasivo.

Los cesionarios de créditos adquiridos dentro de los treinta días anteriores a la fecha de inicio del Procedimiento Concursal de Reorganización, conforme se indica en el artículo 54, no podrán concurrir a la Junta de Acreedores para deliberar y votar el Acuerdo y tampoco podrán impugnarlo.

El acuerdo sobre la propuesta de una clase o categoría se adoptará bajo la condición suspensiva de que se acuerde la propuesta de la otra clase o categoría en la misma Junta de Acreedores, o en la que se realice de conformidad a lo previsto en el artículo 82.

\hypertarget{artuxedculo-80.--procedimiento-de-registro-de-firmas.}{%
\paragraph*{Artículo 80.- Procedimiento de registro de firmas.}\label{artuxedculo-80.--procedimiento-de-registro-de-firmas.}}
\addcontentsline{toc}{paragraph}{Artículo 80.- Procedimiento de registro de firmas.}

Para obtener las mayorías que exige el Procedimiento Concursal de Reorganización, el Veedor podrá recabar la votación de cualquier acreedor, mediante la suscripción de uno o más documentos ante un ministro de fe o mediante firma electrónica avanzada, en que conste la aceptación de los acreedores.

Los votos que se obtengan mediante este sistema se considerarán como votos de acreedores presentes en la Junta de Acreedores llamada a conocer y pronunciarse sobre la propuesta de Acuerdo, para los efectos del cómputo de las mayorías.

Los acreedores del Deudor podrán suscribir estos documentos desde la publicación de la propuesta de Acuerdo en el Boletín Concursal, y hasta tres días antes de la fecha fijada para la Junta de Acreedores llamada a conocer y pronunciarse sobre dicha propuesta.

\hypertarget{artuxedculo-81.--ausencia-del-deudor-en-la-junta-de-acreedores.}{%
\paragraph*{Artículo 81.- Ausencia del Deudor en la Junta de Acreedores.}\label{artuxedculo-81.--ausencia-del-deudor-en-la-junta-de-acreedores.}}
\addcontentsline{toc}{paragraph}{Artículo 81.- Ausencia del Deudor en la Junta de Acreedores.}

Si el Deudor no compareciere a la Junta de Acreedores llamada a conocer y pronunciarse sobre la propuesta de Acuerdo, el tribunal competente deberá dictar la Resolución de Liquidación en la misma Junta.

\hypertarget{artuxedculo-82.--suspensiuxf3n-de-la-junta-de-acreedores.}{%
\paragraph*{Artículo 82.- Suspensión de la Junta de Acreedores.}\label{artuxedculo-82.--suspensiuxf3n-de-la-junta-de-acreedores.}}
\addcontentsline{toc}{paragraph}{Artículo 82.- Suspensión de la Junta de Acreedores.}

La Junta de Acreedores llamada a conocer y pronunciarse sobre la propuesta de Acuerdo podrá acordar con, Quórum Calificado, su suspensión por no más diez días, fijando al efecto nuevo día y hora para su reanudación.

El Deudor conservará la Protección Financiera Concursal hasta la celebración de dicha Junta.

\hypertarget{artuxedculo-83.--modificaciuxf3n-del-acuerdo.}{%
\paragraph*{Artículo 83.- Modificación del Acuerdo.}\label{artuxedculo-83.--modificaciuxf3n-del-acuerdo.}}
\addcontentsline{toc}{paragraph}{Artículo 83.- Modificación del Acuerdo.}

Las modificaciones al Acuerdo deberán adoptarse por el Deudor y los acreedores que lo suscribieron agrupados en sus respectivas clases o categorías, conforme al mismo procedimiento y mayorías exigidas en el artículo 79.

No obstante lo anterior, el Acuerdo que establezca la constitución de una Comisión de Acreedores podrá facultarla para modificarlo con el quórum de aprobación que el mismo Acuerdo determine, el que en ningún caso podrá ser inferior al Quórum Simple.

La modificación podrá recaer sobre todo o parte del contenido del Acuerdo, salvo lo referente a la calidad de acreedor, su clase o categoría, diferencias entre acreedores de igual clase o categoría, monto de sus créditos, sus preferencias, y respecto de aquellas materias que el Acuerdo determine como no modificables por la Comisión de Acreedores.

En las Juntas de Acreedores que se celebren con posterioridad a la aprobación del Acuerdo por el tribunal, el derecho a voto se determinará en conformidad al artículo 78. No tendrán derecho a voto los acreedores que tengan la calidad de Personas Relacionadas con el Deudor.

\hypertarget{artuxedculo-84.--notificaciuxf3n-del-acuerdo.}{%
\paragraph*{Artículo 84.- Notificación del Acuerdo.}\label{artuxedculo-84.--notificaciuxf3n-del-acuerdo.}}
\addcontentsline{toc}{paragraph}{Artículo 84.- Notificación del Acuerdo.}

El texto íntegro del Acuerdo con sus modificaciones, en su caso, será notificado por el Veedor en el Boletín Concursal.

\hypertarget{puxe1rrafo-2.-de-la-impugnaciuxf3n-del-acuerdo-de-reorganizaciuxf3n-judicial}{%
\subsubsection*{Párrafo 2. De la impugnación del Acuerdo de Reorganización Judicial}\label{puxe1rrafo-2.-de-la-impugnaciuxf3n-del-acuerdo-de-reorganizaciuxf3n-judicial}}
\addcontentsline{toc}{subsubsection}{Párrafo 2. De la impugnación del Acuerdo de Reorganización Judicial}

\hypertarget{artuxedculo-85.--causales-para-impugnar-el-acuerdo.}{%
\paragraph*{Artículo 85.- Causales para impugnar el Acuerdo.}\label{artuxedculo-85.--causales-para-impugnar-el-acuerdo.}}
\addcontentsline{toc}{paragraph}{Artículo 85.- Causales para impugnar el Acuerdo.}

El Acuerdo podrá ser impugnado por los acreedores a los que les afecte, siempre que se funde en alguna de las siguientes causales:

\begin{enumerate}
\def\labelenumi{\arabic{enumi})}
\item
  Defectos en las formas establecidas para la convocatoria y celebración de la junta de acreedores, que hubieren impedido el ejercicio de los derechos de los acreedores o del deudor.
\item
  El error en el cómputo de las mayorías requeridas en este Capítulo, siempre que incida sustancialmente en el quórum del Acuerdo de Reorganización Judicial.
\item
  Falsedad o exageración del crédito o incapacidad o falta de personería para votar de alguno de los acreedores que hayan concurrido con su voto a formar el quórum necesario para el Acuerdo, si excluido este acreedor o la parte falsa o exagerada del crédito, no se logra el quórum del Acuerdo.
\item
  Acuerdo entre uno o más acreedores y el Deudor para votar a favor, abstenerse de votar o rechazar el Acuerdo, para obtener una ventaja indebida respecto de los demás acreedores.
\item
  Ocultación o exageración del activo o pasivo.
\item
  Por contener una o más estipulaciones contrarias a lo dispuesto en esta ley.
\end{enumerate}

\hypertarget{artuxedculo-86.--plazo-para-impugnar-el-acuerdo.}{%
\paragraph*{Artículo 86.- Plazo para impugnar el Acuerdo.}\label{artuxedculo-86.--plazo-para-impugnar-el-acuerdo.}}
\addcontentsline{toc}{paragraph}{Artículo 86.- Plazo para impugnar el Acuerdo.}

Podrá impugnarse el Acuerdo dentro del plazo de cinco días contado desde su publicación en el Boletín Concursal.

Las impugnaciones presentadas fuera de plazo serán rechazadas de plano.

\hypertarget{artuxedculo-87.--audiencia-uxfanica-de-resoluciuxf3n-de-impugnaciones.}{%
\paragraph*{Artículo 87.- Audiencia única de resolución de impugnaciones.}\label{artuxedculo-87.--audiencia-uxfanica-de-resoluciuxf3n-de-impugnaciones.}}
\addcontentsline{toc}{paragraph}{Artículo 87.- Audiencia única de resolución de impugnaciones.}

Las impugnaciones al Acuerdo se tramitarán como un solo incidente y se fallarán conjuntamente en una audiencia única, que el tribunal competente citará para tal efecto, dentro de los diez días de vencido el plazo para impugnar. Esta audiencia será verbal y se llevará a cabo con los que asistan. En la misma audiencia deberán resolverse las incidencias que promuevan las partes. El tribunal podrá, si así lo estima, suspender y continuar la referida audiencia con posterioridad. La resolución que se pronuncie sobre las impugnaciones al Acuerdo deberá dictarse a más tardar dentro de los treinta días siguientes a la fecha de celebración de la referida audiencia.

La resolución que resuelva las impugnaciones se publicará en el Boletín Concursal. Esta resolución será apelable en el solo efecto devolutivo.

\hypertarget{artuxedculo-88.--nueva-propuesta-de-acuerdo.}{%
\paragraph*{Artículo 88.- Nueva propuesta de Acuerdo.}\label{artuxedculo-88.--nueva-propuesta-de-acuerdo.}}
\addcontentsline{toc}{paragraph}{Artículo 88.- Nueva propuesta de Acuerdo.}

Si se acoge por resolución firme y ejecutoriada la impugnación al Acuerdo por las causales establecidas en los números 1), 2), 3) y 6) del artículo 85, el Deudor podrá presentar una nueva propuesta de Acuerdo, dentro de los diez días siguientes contados desde que se notifique la resolución que tuvo por acogida la impugnación referida, siempre que esta nueva propuesta se presente apoyada por dos o más acreedores que representen, a lo menos, un 66\% del pasivo total con derecho a voto. En este caso, el Deudor gozará de Protección Financiera Concursal hasta la celebración de la Junta llamada a conocer y pronunciarse sobre la nueva propuesta. La resolución que tenga por presentada la nueva propuesta de Acuerdo fijará la fecha de la Junta de Acreedores llamada a conocer y pronunciarse sobre dicha nueva propuesta, la que deberá celebrarse dentro de los diez días siguientes contados desde que el Deudor la presentó.

Si el Deudor no presentare la nueva propuesta de Acuerdo que reúna las condiciones indicadas en el inciso anterior, dentro del plazo antes establecido, el tribunal competente dictará, de oficio y sin más trámite, la Resolución de Liquidación de la Empresa Deudora.

Si se acoge una impugnación al Acuerdo por las causales establecidas en los números 4) y 5) del artículo 85, el tribunal, de oficio y sin más trámite, ordenará el inicio del Procedimiento Concursal de Liquidación en la misma resolución que acoge la impugnación, y el Deudor no podrá presentar nuevamente una propuesta de Acuerdo.

\hypertarget{puxe1rrafo-3.-de-la-aprobaciuxf3n-y-vigencia-del-acuerdo-de-reorganizaciuxf3n-judicial}{%
\subsubsection*{Párrafo 3. De la aprobación y vigencia del Acuerdo de Reorganización Judicial}\label{puxe1rrafo-3.-de-la-aprobaciuxf3n-y-vigencia-del-acuerdo-de-reorganizaciuxf3n-judicial}}
\addcontentsline{toc}{subsubsection}{Párrafo 3. De la aprobación y vigencia del Acuerdo de Reorganización Judicial}

\hypertarget{artuxedculo-89.--aprobaciuxf3n-y-vigencia-del-acuerdo.}{%
\paragraph*{Artículo 89.- Aprobación y vigencia del Acuerdo.}\label{artuxedculo-89.--aprobaciuxf3n-y-vigencia-del-acuerdo.}}
\addcontentsline{toc}{paragraph}{Artículo 89.- Aprobación y vigencia del Acuerdo.}

El Acuerdo se entenderá aprobado y comenzará a regir una vez vencido el plazo para impugnarlo, sin que se hubiere impugnado y el tribunal competente lo declare así de oficio o a petición de cualquier interesado o del Veedor.
Si el Acuerdo fuere impugnado y las impugnaciones fueren desechadas, el tribunal competente lo declarará aprobado en la resolución que deseche la o las impugnaciones, y aquél comenzará a regir desde que dicha resolución cause ejecutoria.

Las resoluciones señaladas en los incisos primero y segundo de este artículo se notificarán en el Boletín Concursal.

El Acuerdo regirá no obstante las impugnaciones que se hubieren interpuesto en su contra. Sin embargo, si éstas fueren interpuestas por acreedores de una determinada clase o categoría, que representen en su conjunto a lo menos el 30\% del pasivo con derecho a voto de su respectiva clase o categoría, el Acuerdo no empezará regir hasta que dichas impugnaciones fueren desestimadas por sentencia firme y ejecutoriada. En este caso y en el del inciso segundo de este artículo, los actos y contratos ejecutados o celebrados por el Deudor en el tiempo que medie entre el Acuerdo y la fecha en que quede ejecutoriada la resolución que acoja las impugnaciones, no podrán dejarse sin efecto.

El recurso de casación deducido en contra de la resolución de segunda instancia que desecha la o las impugnaciones, no suspenderá el cumplimiento de dicha resolución, incluso si la parte vencida solicita que se otorgue fianza de resultas por la parte vencedora.

Si se acogen las impugnaciones al Acuerdo por resolución firme y ejecutoriada, las obligaciones y derechos existentes entre el Deudor y sus acreedores con anterioridad a éste se regirán por sus respectivas convenciones.

\hypertarget{artuxedculo-90.--autorizaciuxf3n-del-acuerdo.}{%
\paragraph*{Artículo 90.- Autorización del Acuerdo.}\label{artuxedculo-90.--autorizaciuxf3n-del-acuerdo.}}
\addcontentsline{toc}{paragraph}{Artículo 90.- Autorización del Acuerdo.}

Una copia del acta de la Junta de Acreedores en la que conste el voto favorable del Acuerdo y su texto íntegro, junto a la copia de la resolución judicial que lo aprueba y su certificado de ejecutoria, podrá ser autorizada por un ministro de fe o protocolizarse ante un notario público. Una vez autorizada o protocolizada, tendrá mérito ejecutivo para todos los efectos legales.

\hypertarget{puxe1rrafo-4.-de-los-efectos-del-acuerdo-de-reorganizaciuxf3n-judicial}{%
\subsubsection*{Párrafo 4. De los efectos del Acuerdo de Reorganización Judicial}\label{puxe1rrafo-4.-de-los-efectos-del-acuerdo-de-reorganizaciuxf3n-judicial}}
\addcontentsline{toc}{subsubsection}{Párrafo 4. De los efectos del Acuerdo de Reorganización Judicial}

\hypertarget{artuxedculo-91.--efectos.}{%
\paragraph*{Artículo 91.- Efectos.}\label{artuxedculo-91.--efectos.}}
\addcontentsline{toc}{paragraph}{Artículo 91.- Efectos.}

El Acuerdo, debidamente aprobado, obliga al Deudor y a todos los acreedores de cada clase o categoría de éste, hayan o no concurrido a la Junta que lo acuerde.

\hypertarget{artuxedculo-92.--cancelaciuxf3n-de-anotaciones-e-inscripciones.}{%
\paragraph*{Artículo 92.- Cancelación de anotaciones e inscripciones.}\label{artuxedculo-92.--cancelaciuxf3n-de-anotaciones-e-inscripciones.}}
\addcontentsline{toc}{paragraph}{Artículo 92.- Cancelación de anotaciones e inscripciones.}

Aprobado el Acuerdo de Reorganización Judicial, se cancelarán las inscripciones previstas en el número 7) del artículo 57.

\hypertarget{artuxedculo-93.--efectos-sobre-los-cruxe9ditos.}{%
\paragraph*{Artículo 93.- Efectos sobre los créditos.}\label{artuxedculo-93.--efectos-sobre-los-cruxe9ditos.}}
\addcontentsline{toc}{paragraph}{Artículo 93.- Efectos sobre los créditos.}

Los créditos que sean parte del Acuerdo de Reorganización Judicial se entenderán remitidos, novados o repactados, según corresponda, para todos los efectos legales.

El acreedor, contribuyente del impuesto de primera categoría de la Ley sobre Impuesto a la Renta, contenida en el artículo 1° del decreto ley N° 824, de 1974, podrá deducir como gasto necesario conforme a lo dispuesto en el número 4° del artículo 31 de dicha ley, las cantidades que correspondan a la condonación o remisión de deudas, intereses, reajustes u otras cantidades que se hayan devengado en su favor, siempre y cuando cumpla con las siguientes condiciones copulativas:

\begin{enumerate}
\def\labelenumi{\arabic{enumi})}
\item
  Que se trate de créditos otorgados o adquiridos con anterioridad al plazo de un año contado desde la celebración del Acuerdo de Reorganización Judicial;
\item
  Que dicha condonación o remisión conste detalladamente en el referido Acuerdo o sus modificaciones, aprobado conforme al artículo 89, y
\item
  Que no correspondan a créditos de Personas Relacionadas con el Deudor ni a créditos de acreedores Personas Relacionadas entre sí, cuando éstos, en su conjunto, representen el 50\% o más del pasivo reconocido con derecho a voto.
\end{enumerate}

Lo anterior, es sin perjuicio de la obligación del Deudor de reconocer como ingreso, para efectos tributarios, aquellas cantidades que se hubieren devengado a favor del acreedor y que se condonen o remitan.

\hypertarget{artuxedculo-94.--de-los-bienes-no-esenciales-para-la-continuidad-del-giro-de-la-empresa-deudora.}{%
\paragraph*{Artículo 94.- De los bienes no esenciales para la continuidad del giro de la Empresa Deudora.}\label{artuxedculo-94.--de-los-bienes-no-esenciales-para-la-continuidad-del-giro-de-la-empresa-deudora.}}
\addcontentsline{toc}{paragraph}{Artículo 94.- De los bienes no esenciales para la continuidad del giro de la Empresa Deudora.}

En el plazo de ocho días siguientes a la publicación de la Resolución de Reorganización referida en el artículo 57, el acreedor cuyo crédito se encuentre garantizado con prenda o hipoteca podrá solicitar fundadamente al tribunal competente que declare que el bien sobre el que recae su garantía no es esencial para el giro de la Empresa Deudora. Para resolver lo anterior, el tribunal podrá solicitar al Veedor un informe que contendrá la calificación de si el bien es o no esencial para el giro de la Empresa Deudora y el avalúo comercial del bien sobre el que recaen las referidas garantías. El tribunal deberá resolver dicha calificación en única instancia, a más tardar el segundo día anterior a la fecha de celebración de la Junta de Acreedores llamada a conocer y pronunciarse sobre las proposiciones de Acuerdo de Reorganización Judicial.

El acreedor cuya garantía recae sobre un bien calificado como no esencial concurrirá y votará en la clase o categoría de acreedores valistas, únicamente por el saldo del crédito no cubierto por la garantía. El saldo cubierto por la garantía no se considerará en el pasivo de la clase o categoría de acreedores garantizados.

El acreedor cuyo crédito no hubiere sido enteramente cubierto por la garantía podrá solicitar, mediante un procedimiento incidental ante el mismo tribunal que conoció y se pronunció sobre el Acuerdo, que dicho Acuerdo se cumpla a su favor, mientras no se encuentren prescritas las acciones que del mismo emanen.

El excedente que resulte de la venta del bien declarado no esencial, una vez pagado el respectivo crédito, se destinará al cumplimiento del Acuerdo.

\hypertarget{artuxedculo-95.--efectos-del-acuerdo-de-reorganizaciuxf3n-judicial-en-las-obligaciones-garantizadas-del-deudor.}{%
\paragraph*{Artículo 95.- Efectos del Acuerdo de Reorganización Judicial en las obligaciones garantizadas del Deudor.}\label{artuxedculo-95.--efectos-del-acuerdo-de-reorganizaciuxf3n-judicial-en-las-obligaciones-garantizadas-del-deudor.}}
\addcontentsline{toc}{paragraph}{Artículo 95.- Efectos del Acuerdo de Reorganización Judicial en las obligaciones garantizadas del Deudor.}

Tales efectos serán los siguientes:

\begin{enumerate}
\def\labelenumi{\arabic{enumi}.}
\item
  Respecto de las obligaciones del Deudor garantizadas con prenda o hipoteca sobre bienes de propiedad del Deudor o de terceros, declarados esenciales para el giro de la Empresa Deudora, de acuerdo a los artículos 56 y 94, se aplicarán los términos y modalidades establecidos en el Acuerdo de Reorganización Judicial.
\item
  Respecto de las obligaciones del Deudor garantizadas con prenda o hipoteca sobre bienes de propiedad del Deudor, declarados no esenciales para el giro de la Empresa Deudora de acuerdo a los artículos 56 y 94, regirá lo establecido en los incisos segundo y tercero del artículo anterior.
\item
  Respecto de las obligaciones del Deudor garantizadas con prenda o hipoteca sobre bienes de propiedad de terceros, declarados no esenciales para el giro de la Empresa Deudora de acuerdo a los artículos 56 y 94, deberá distinguirse:
\end{enumerate}

\begin{enumerate}
\def\labelenumi{\alph{enumi})}
\item
  Si el respectivo acreedor vota a favor del Acuerdo, se sujetará a los términos y modalidades establecidos en el referido acuerdo y no podrá perseguir su crédito en términos distintos a los estipulados.
\item
  Si el respectivo acreedor manifiesta su intención de no votar o no asiste a la Junta de Acreedores llamada a conocer y pronunciarse sobre la propuesta de Acuerdo, su crédito no se considerará en el pasivo con derecho a voto correspondiente a su clase o categoría, y podrá cobrar su crédito respecto de las prendas o hipotecas otorgadas por terceros.
\end{enumerate}

\begin{enumerate}
\def\labelenumi{\arabic{enumi}.}
\setcounter{enumi}{3}
\tightlist
\item
  Respecto de las obligaciones del Deudor garantizadas con cauciones personales, deberá distinguirse:
\end{enumerate}

\begin{enumerate}
\def\labelenumi{\alph{enumi})}
\item
  Si el respectivo acreedor vota en su clase o categoría de valista a favor del Acuerdo, se sujetará a los términos y modalidades establecidos en el referido acuerdo y no podrá cobrar su crédito en términos distintos a los estipulados.
\item
  Si el respectivo acreedor manifiesta su intención de no votar o no asiste a la Junta de Acreedores llamada a conocer y pronunciarse sobre la propuesta de Acuerdo, su crédito no se considerará en el pasivo con derecho a voto correspondiente a su clase o categoría, y podrá cobrar su crédito respecto de los fiadores o codeudores, solidarios o subsidiarios, o avalistas en los términos originalmente pactados.
\end{enumerate}

El fiador, codeudor, solidario o subsidiario, avalista, tercero poseedor de la finca hipotecada o propietario del bien prendado que hubiere pagado, de acuerdo a lo establecido en la letra b) del número 3) o en la letra b) del número 4) anteriores, podrá ejercer, según corresponda, su derecho de subrogación o reembolso, mediante un procedimiento incidental, ante el mismo tribunal que conoció y se pronunció sobre el Acuerdo, solicitando que éste se cumpla a su favor, mientras no se encuentren prescritas las acciones que de él resulten.

\hypertarget{puxe1rrafo-5.-del-rechazo-del-acuerdo-de-reorganizaciuxf3n-judicial}{%
\subsubsection*{Párrafo 5. Del rechazo del Acuerdo de Reorganización Judicial}\label{puxe1rrafo-5.-del-rechazo-del-acuerdo-de-reorganizaciuxf3n-judicial}}
\addcontentsline{toc}{subsubsection}{Párrafo 5. Del rechazo del Acuerdo de Reorganización Judicial}

\hypertarget{artuxedculo-96.--rechazo-del-acuerdo.}{%
\paragraph*{Artículo 96.- Rechazo del Acuerdo.}\label{artuxedculo-96.--rechazo-del-acuerdo.}}
\addcontentsline{toc}{paragraph}{Artículo 96.- Rechazo del Acuerdo.}

Si la propuesta de Acuerdo es rechazada por los acreedores por no haberse obtenido el quórum de aprobación necesario o porque el Deudor no otorga su consentimiento, el tribunal dictará la Resolución de Liquidación, de oficio y sin más trámite, en la misma Junta de Acreedores llamada a conocer y pronunciarse sobre el Acuerdo, salvo que la referida Junta disponga lo contrario por Quórum Especial.

En este caso, el Deudor deberá, a través del Veedor, publicar una nueva propuesta de Acuerdo en el Boletín Concursal y acompañarla al tribunal diez días antes de la Junta de Acreedores que tiene por objeto pronunciarse sobre ésta. El Deudor conservará la Protección Financiera Concursal hasta la celebración de dicha Junta, que deberá llevarse a cabo dentro de los veinte días siguientes a la que rechazó el Acuerdo.

Si el Deudor no presenta la nueva propuesta de Acuerdo dentro del plazo antes establecido, el tribunal dictará la Resolución de Liquidación, de oficio y sin más trámite.

La Junta de Acreedores que rechace la primera o segunda propuesta de Acuerdo, en su caso, deberá nominar a los Liquidadores titular y suplente, a los que el tribunal competente deberá designar con el carácter de definitivos.

\hypertarget{puxe1rrafo-6.-de-la-nulidad-y-declaraciuxf3n-de-incumplimiento-del-acuerdo-de-reorganizaciuxf3n-judicial}{%
\subsubsection*{Párrafo 6. De la nulidad y declaración de incumplimiento del Acuerdo de Reorganización Judicial}\label{puxe1rrafo-6.-de-la-nulidad-y-declaraciuxf3n-de-incumplimiento-del-acuerdo-de-reorganizaciuxf3n-judicial}}
\addcontentsline{toc}{subsubsection}{Párrafo 6. De la nulidad y declaración de incumplimiento del Acuerdo de Reorganización Judicial}

\hypertarget{artuxedculo-97.--nulidad-del-acuerdo.}{%
\paragraph*{Artículo 97.- Nulidad del Acuerdo.}\label{artuxedculo-97.--nulidad-del-acuerdo.}}
\addcontentsline{toc}{paragraph}{Artículo 97.- Nulidad del Acuerdo.}

No se admitirán otras acciones en contra del Acuerdo que las fundadas en la ocultación o exageración del activo o del pasivo y de las que se hubiere tomado conocimiento después de haber vencido el plazo para impugnar el Acuerdo.

La declaración de nulidad del Acuerdo extingue de pleno derecho las cauciones que lo garantizan.

Las acciones de nulidad del Acuerdo podrán interponerse por cualquier interesado y prescribirán en el plazo de un año contado desde la fecha en que aquél comenzó a regir.

\hypertarget{artuxedculo-98.--acciuxf3n-de-incumplimiento.}{%
\paragraph*{Artículo 98.- Acción de incumplimiento.}\label{artuxedculo-98.--acciuxf3n-de-incumplimiento.}}
\addcontentsline{toc}{paragraph}{Artículo 98.- Acción de incumplimiento.}

El Acuerdo podrá declararse incumplido a solicitud de cualquiera de los acreedores a los que les afecte por inobservancia de sus estipulaciones.

Podrá también declararse incumplido si se hubiere agravado el mal estado de los negocios del Deudor de forma que haga temer un perjuicio para dichos acreedores.

Si la acción de incumplimiento se deduce sólo por la inobservancia de las estipulaciones de una de las clases o categorías del Acuerdo, el Deudor podrá enervar la acción cumpliendo dichas estipulaciones dentro del plazo de sesenta días contado desde la notificación de la acción. El Deudor podrá enervarla por una sola vez para cada categoría o clase del Acuerdo.

Las acciones de incumplimiento del Acuerdo prescribirán en el plazo de un año contado desde que se produce el incumplimiento.

La declaración de incumplimiento dejará sin efecto el Acuerdo, pero no extinguirá las cauciones que hubieren garantizado su ejecución total o parcial.

Las personas obligadas por las cauciones señaladas en el inciso anterior y los terceros poseedores de los bienes gravados con las mismas, según sea el caso, serán oídos en el juicio de declaración de incumplimiento y podrán impedir la continuación de éste enervando la acción mediante el cumplimiento del Acuerdo dentro de los tres días siguientes a la citación.

Las cantidades pagadas por el Deudor antes de la declaración de incumplimiento del Acuerdo y el producto obtenido durante el Procedimiento Concursal de Liquidación servirán de abono a la deuda en caso que la caución se extienda a toda la suma estipulada. Pero si comprende únicamente una parte de ella, sólo le servirá para imputarla a la parte que reste de la cuota no caucionada.

\hypertarget{artuxedculo-99.--procedimiento-de-declaraciuxf3n-de-nulidad-e-incumplimiento-del-acuerdo.}{%
\paragraph*{Artículo 99.- Procedimiento de declaración de nulidad e incumplimiento del Acuerdo.}\label{artuxedculo-99.--procedimiento-de-declaraciuxf3n-de-nulidad-e-incumplimiento-del-acuerdo.}}
\addcontentsline{toc}{paragraph}{Artículo 99.- Procedimiento de declaración de nulidad e incumplimiento del Acuerdo.}

La nulidad o incumplimiento del Acuerdo se sujetarán al procedimiento del juicio sumario y será competente para conocer de estas acciones el tribunal ante el cual se tramitó el Acuerdo.

La resolución que acoja las acciones de nulidad o incumplimiento del Acuerdo será apelable en ambos efectos, pero el Deudor quedará de inmediato sujeto a la intervención de un Veedor que tendrá las facultades de interventor contenidas en los números 1), 7), 8) y 9) del artículo 25.

La declaración de nulidad o incumplimiento del Acuerdo no tendrá efecto retroactivo y no afectará la validez de los actos o contratos debidamente celebrados en el tiempo que media entre la resolución que aprueba el Acuerdo y la que declare la nulidad o el incumplimiento.

\hypertarget{artuxedculo-100.--inicio-del-procedimiento-concursal-de-liquidaciuxf3n.}{%
\paragraph*{Artículo 100.- Inicio del Procedimiento Concursal de Liquidación.}\label{artuxedculo-100.--inicio-del-procedimiento-concursal-de-liquidaciuxf3n.}}
\addcontentsline{toc}{paragraph}{Artículo 100.- Inicio del Procedimiento Concursal de Liquidación.}

Una vez firme y ejecutoriada la resolución que declare la nulidad o el incumplimiento del Acuerdo, el mismo tribunal dictará la Resolución de Liquidación de la Empresa Deudora, de oficio y sin más trámite.

\hypertarget{artuxedculo-101.--designaciuxf3n-del-liquidador.}{%
\paragraph*{Artículo 101.- Designación del Liquidador.}\label{artuxedculo-101.--designaciuxf3n-del-liquidador.}}
\addcontentsline{toc}{paragraph}{Artículo 101.- Designación del Liquidador.}

En la demanda de nulidad o de incumplimiento del Acuerdo, el demandante propondrá a un Liquidador titular y a uno suplente de la Nómina de Liquidadores vigente, debiendo el tribunal designarlos en la Resolución de Liquidación.

Si se interpusiere más de una demanda de nulidad o de incumplimiento del Acuerdo, el tribunal competente designará a los Liquidadores titular y suplente nominados en la primera demanda que se acoja.

\hypertarget{tuxedtulo-3.-del-acuerdo-de-reorganizaciuxf3n-extrajudicial-o-simplificado}{%
\subsection*{Título 3. Del Acuerdo de Reorganización Extrajudicial o Simplificado}\label{tuxedtulo-3.-del-acuerdo-de-reorganizaciuxf3n-extrajudicial-o-simplificado}}
\addcontentsline{toc}{subsection}{Título 3. Del Acuerdo de Reorganización Extrajudicial o Simplificado}

\hypertarget{artuxedculo-102.--legitimaciuxf3n.}{%
\paragraph*{Artículo 102.- Legitimación.}\label{artuxedculo-102.--legitimaciuxf3n.}}
\addcontentsline{toc}{paragraph}{Artículo 102.- Legitimación.}

Toda Empresa Deudora podrá celebrar un Acuerdo de Reorganización Extrajudicial o Simplificado con sus acreedores y someterlo a aprobación judicial, conforme a lo establecido en el presente Título. Para los efectos de este Título se denominará indistintamente Empresa Deudora o Deudor.

\hypertarget{artuxedculo-103.--competencia.}{%
\paragraph*{Artículo 103.- Competencia.}\label{artuxedculo-103.--competencia.}}
\addcontentsline{toc}{paragraph}{Artículo 103.- Competencia.}

Será competente para aprobar el Acuerdo Simplificado el tribunal que hubiere sido competente para conocer de un Procedimiento Concursal de Reorganización del Deudor de acuerdo a esta ley.

\hypertarget{artuxedculo-104.--formalidades.}{%
\paragraph*{Artículo 104.- Formalidades.}\label{artuxedculo-104.--formalidades.}}
\addcontentsline{toc}{paragraph}{Artículo 104.- Formalidades.}

El Acuerdo Simplificado deberá ser otorgado ante un ministro de fe o ante un ministro de fe de la Superintendencia, quien certificará, además, la personería de los representantes que concurran al otorgamiento de este instrumento, cuyas copias autorizadas deberán agregarse al Acuerdo respectivo.

\hypertarget{artuxedculo-105.--objeto.}{%
\paragraph*{Artículo 105.- Objeto.}\label{artuxedculo-105.--objeto.}}
\addcontentsline{toc}{paragraph}{Artículo 105.- Objeto.}

El Acuerdo Simplificado podrá versar sobre cualquier objeto tendiente a reestructurar los activos y pasivos del Deudor.

\hypertarget{artuxedculo-106.--normas-aplicables.}{%
\paragraph*{Artículo 106.- Normas aplicables.}\label{artuxedculo-106.--normas-aplicables.}}
\addcontentsline{toc}{paragraph}{Artículo 106.- Normas aplicables.}

Serán aplicables al Acuerdo Simplificado, cuando corresponda y siempre que no contravengan lo dispuesto en el presente Párrafo, los Títulos 1 y 2 de este Capítulo, en lo relativo a los acuerdos por clases o categorías de acreedores, determinación del pasivo, propuestas alternativas, diferencias entre acreedores de igual clase o categoría, condonación o remisión de créditos, constitución de garantías, cláusulas de arbitraje, nombramiento del interventor y designación de la Comisión de Acreedores.

\hypertarget{artuxedculo-107.--requisitos.}{%
\paragraph*{Artículo 107.- Requisitos.}\label{artuxedculo-107.--requisitos.}}
\addcontentsline{toc}{paragraph}{Artículo 107.- Requisitos.}

Para la aprobación judicial del Acuerdo Simplificado, éste deberá presentarse ante el tribunal competente junto con los antecedentes singularizados en el artículo 56, acompañado de un listado de todos los juicios y procesos administrativos seguidos contra el Deudor que tengan efectos patrimoniales, con indicación del tribunal, órgano de la Administración del Estado, rol o número de identificación y materias sobre las que tratan estos procesos.

Conjuntamente con la presentación del Acuerdo Simplificado, deberá presentarse un informe de un Veedor de la Nómina de Veedores, elegido por el Deudor y sus dos principales acreedores, que deberá contener la calificación fundada acerca de:

\begin{enumerate}
\def\labelenumi{\arabic{enumi}.}
\item
  Si la propuesta es susceptible de ser cumplida, habida consideración de las condiciones del Deudor;
\item
  El monto probable de recuperación que le correspondería a cada acreedor en sus respectivas categorías, en caso de un Procedimiento Concursal de Liquidación, y
\item
  Si la determinación de los créditos y su preferencia, cuya propuesta acompañó el Deudor, se ajusta a esta ley.
\end{enumerate}

\hypertarget{artuxedculo-108.--resoluciuxf3n-de-reorganizaciuxf3n-simplificada.}{%
\paragraph*{Artículo 108.- Resolución de Reorganización Simplificada.}\label{artuxedculo-108.--resoluciuxf3n-de-reorganizaciuxf3n-simplificada.}}
\addcontentsline{toc}{paragraph}{Artículo 108.- Resolución de Reorganización Simplificada.}

Presentada la solicitud de aprobación judicial del Acuerdo Simplificado y hasta la aprobación judicial regulada en el artículo 112, el tribunal dispondrá:

\begin{enumerate}
\def\labelenumi{\alph{enumi})}
\item
  La prohibición de solicitar la Liquidación Forzosa del Deudor y de iniciarse en su contra juicios ejecutivos, ejecuciones de cualquier clase o restitución en los juicios de arrendamiento. Lo anterior no se aplicará a los juicios laborales sobre obligaciones que gocen de preferencia de primera clase, suspendiéndose en ese caso sólo la ejecución y realización de bienes del Deudor, excepto los que el Deudor tuviere, en tal carácter, a favor de su cónyuge o de sus parientes o de los gerentes, administradores, apoderados con poder general de administración u otras personas que hayan tenido o tengan injerencia en la administración de sus negocios. Para estos efectos, se entenderá por parientes los ascendientes y descendientes y los colaterales por consanguinidad y afinidad hasta el cuarto grado, inclusive.
\item
  La suspensión de la tramitación de los procedimientos señalados en la letra a) precedente y la suspensión de los plazos de prescripción extintiva.
\item
  La prohibición al Deudor de gravar o enajenar sus bienes, salvo los que resulten estrictamente necesarios para la continuación de su giro.
\end{enumerate}

\hypertarget{artuxedculo-109.--quuxf3rum.}{%
\paragraph*{Artículo 109.- Quórum.}\label{artuxedculo-109.--quuxf3rum.}}
\addcontentsline{toc}{paragraph}{Artículo 109.- Quórum.}

El Deudor deberá presentar el Acuerdo Simplificado suscrito por dos o más acreedores que representen al menos tres cuartas partes del total de su pasivo, correspondiente a su respectiva clase o categoría. Las Personas Relacionadas con el Deudor no podrán suscribir un Acuerdo Simplificado, ni sus créditos se considerarán en el monto del pasivo para los efectos de la determinación del quórum de aprobación del referido Acuerdo.

Los cesionarios de créditos adquiridos dentro de los treinta días anteriores a la fecha de la presentación a aprobación judicial del Acuerdo Simplificado tampoco se considerarán para el quórum señalado en el inciso anterior.

\hypertarget{artuxedculo-110.--publicidad.}{%
\paragraph*{Artículo 110.- Publicidad.}\label{artuxedculo-110.--publicidad.}}
\addcontentsline{toc}{paragraph}{Artículo 110.- Publicidad.}

Junto con presentar al tribunal el Acuerdo Extrajudicial o Simplificado con los antecedentes señalados en el artículo 107, el Deudor deberá acompañar al Veedor copia de éstos para que los publique en el Boletín Concursal y los acompañe a los acreedores por medio de correos electrónicos, si lo tuvieren.

\hypertarget{artuxedculo-111.--impugnaciuxf3n.}{%
\paragraph*{Artículo 111.- Impugnación.}\label{artuxedculo-111.--impugnaciuxf3n.}}
\addcontentsline{toc}{paragraph}{Artículo 111.- Impugnación.}

Podrán impugnar el Acuerdo Simplificado los acreedores disidentes y aquellos que demuestren haber sido omitidos de los antecedentes previstos en el artículo 107, siempre y cuando la impugnación se funde en alguna de las causales establecidas en el artículo 85 respecto de los Acuerdos de Reorganización Judicial, o bien en la existencia, los montos y las preferencias de sus créditos.

La impugnación deberá presentarse ante el tribunal competente dentro de los diez días siguientes a la publicación del Acuerdo Simplificado efectuada conforme al artículo anterior. Una copia de la impugnación señalada y de los antecedentes correspondientes deberán ser publicados en el Boletín Concursal por el Veedor.

Las impugnaciones al Acuerdo Simplificado se tramitarán como incidente y se fallarán conjuntamente en una audiencia única, que el tribunal citará para tal efecto y que se celebrará dentro de los diez días siguientes de vencido el plazo para impugnar. Esta audiencia será verbal y se llevará a cabo con los que asistan. La resolución que se pronuncie sobre las impugnaciones se publicará en el Boletín Concursal y será apelable en el solo efecto devolutivo.

\hypertarget{artuxedculo-112.--aprobaciuxf3n-judicial.}{%
\paragraph*{Artículo 112.- Aprobación judicial.}\label{artuxedculo-112.--aprobaciuxf3n-judicial.}}
\addcontentsline{toc}{paragraph}{Artículo 112.- Aprobación judicial.}

Dentro de los diez días siguientes a la publicación del Acuerdo Simplificado, el tribunal podrá citar a todos los acreedores a quienes les afecte el Acuerdo, para su aceptación ante el tribunal, la cual deberá contar con el quórum señalado en el artículo 109.

Una vez aceptado el Acuerdo Simplificado, o vencido el plazo señalado en el inciso anterior sin que el tribunal hubiere citado, y vencido el plazo para presentar impugnaciones sin que se hayan interpuesto o si, deducidas, se hubieren rechazado por resolución que se encuentre firme y ejecutoriada, el tribunal competente, previa verificación del cumplimiento de los requisitos legales, dictará la correspondiente resolución aprobando el Acuerdo Simplificado, debiendo el Veedor publicarla en el Boletín Concursal.

\hypertarget{artuxedculo-113.--efectos-de-la-aprobaciuxf3n-judicial.}{%
\paragraph*{Artículo 113.- Efectos de la aprobación judicial.}\label{artuxedculo-113.--efectos-de-la-aprobaciuxf3n-judicial.}}
\addcontentsline{toc}{paragraph}{Artículo 113.- Efectos de la aprobación judicial.}

El Acuerdo Simplificado aprobado judicialmente de conformidad a las disposiciones anteriores producirá, cuando corresponda, los efectos previstos en el Párrafo 4 del Título 2 de este Capítulo, siempre que no contravenga lo dispuesto en el presente Párrafo.

\hypertarget{artuxedculo-114.--nulidad-e-incumplimiento-del-acuerdo-simplificado.}{%
\paragraph*{Artículo 114.- Nulidad e Incumplimiento del Acuerdo Simplificado.}\label{artuxedculo-114.--nulidad-e-incumplimiento-del-acuerdo-simplificado.}}
\addcontentsline{toc}{paragraph}{Artículo 114.- Nulidad e Incumplimiento del Acuerdo Simplificado.}

Demandada la nulidad o el incumplimiento del Acuerdo Simplificado, se aplicará lo dispuesto en el Párrafo 6 del Título 2 de este Capítulo.

\hypertarget{capuxedtulo-iv-del-procedimiento-concursal-de-liquidaciuxf3n}{%
\section*{CAPÍTULO IV: DEL PROCEDIMIENTO CONCURSAL DE LIQUIDACIÓN}\label{capuxedtulo-iv-del-procedimiento-concursal-de-liquidaciuxf3n}}
\addcontentsline{toc}{section}{CAPÍTULO IV: DEL PROCEDIMIENTO CONCURSAL DE LIQUIDACIÓN}

\hypertarget{tuxedtulo-1.-del-procedimiento-concursal-propiamente-tal}{%
\subsection*{Título 1. Del Procedimiento Concursal propiamente tal}\label{tuxedtulo-1.-del-procedimiento-concursal-propiamente-tal}}
\addcontentsline{toc}{subsection}{Título 1. Del Procedimiento Concursal propiamente tal}

\hypertarget{puxe1rrafo-1.-de-la-liquidaciuxf3n-voluntaria}{%
\subsubsection*{Párrafo 1. De la Liquidación Voluntaria}\label{puxe1rrafo-1.-de-la-liquidaciuxf3n-voluntaria}}
\addcontentsline{toc}{subsubsection}{Párrafo 1. De la Liquidación Voluntaria}

\hypertarget{artuxedculo-115.--uxe1mbito-de-aplicaciuxf3n-y-requisitos.}{%
\paragraph*{Artículo 115.- Ámbito de aplicación y requisitos.}\label{artuxedculo-115.--uxe1mbito-de-aplicaciuxf3n-y-requisitos.}}
\addcontentsline{toc}{paragraph}{Artículo 115.- Ámbito de aplicación y requisitos.}

La Empresa Deudora podrá solicitar ante el juzgado de letras competente su Liquidación Voluntaria, acompañando los siguientes antecedentes, con copia:

\begin{enumerate}
\def\labelenumi{\arabic{enumi})}
\item
  Lista de sus bienes, lugar en que se encuentran y los gravámenes que les afectan.
\item
  Lista de los bienes legalmente excluidos de la Liquidación.
\item
  Relación de sus juicios pendientes.
\item
  Estado de deudas, con nombre, domicilio y datos de contacto de los acreedores, así como la naturaleza de sus créditos.
\item
  Nómina de los trabajadores, cualquiera sea su situación contractual, con indicación de las prestaciones laborales y previsionales adeudadas y fueros en su caso.
\item
  Si el Deudor llevare contabilidad completa presentará, además, su último balance.
\end{enumerate}

Si se tratare de una persona jurídica, los documentos antes referidos serán firmados por sus representantes legales.

Para los efectos de este Capítulo se denominará indistintamente Empresa Deudora o Deudor.

\hypertarget{artuxedculo-116.--tramitaciuxf3n.}{%
\paragraph*{Artículo 116.- Tramitación.}\label{artuxedculo-116.--tramitaciuxf3n.}}
\addcontentsline{toc}{paragraph}{Artículo 116.- Tramitación.}

El tribunal competente revisará la presentación del Deudor y, si cumple con los requisitos señalados en el Artículo anterior, procederá dentro de tercero día de conformidad a lo dispuesto en los Artículos 37 y 129, aplicándose lo establecido en el Párrafo 4 de este Título.

\hypertarget{puxe1rrafo-2.-de-la-liquidaciuxf3n-forzosa}{%
\subsubsection*{Párrafo 2. De la Liquidación Forzosa}\label{puxe1rrafo-2.-de-la-liquidaciuxf3n-forzosa}}
\addcontentsline{toc}{subsubsection}{Párrafo 2. De la Liquidación Forzosa}

\hypertarget{artuxedculo-117.--uxe1mbito-de-aplicaciuxf3n-y-causales.}{%
\paragraph*{Artículo 117.- Ámbito de aplicación y causales.}\label{artuxedculo-117.--uxe1mbito-de-aplicaciuxf3n-y-causales.}}
\addcontentsline{toc}{paragraph}{Artículo 117.- Ámbito de aplicación y causales.}

Cualquier acreedor podrá demandar el inicio del Procedimiento Concursal de Liquidación de una Empresa Deudora en los siguientes casos:

\begin{enumerate}
\def\labelenumi{\arabic{enumi})}
\item
  Si cesa en el pago de una obligación que conste en Título ejecutivo con el acreedor solicitante. Esta causal no podrá invocarse para solicitar el inicio del Procedimiento Concursal de Liquidación respecto de los fiadores, codeudores solidarios o subsidiarios, o avalistas de la Empresa Deudora que ha cesado en el pago de las obligaciones garantizadas por éstos.
\item
  Si existieren en su contra dos o más Títulos ejecutivos vencidos, provenientes de obligaciones diversas, encontrándose iniciadas a lo menos dos ejecuciones, y no hubiere presentado bienes suficientes para responder a la prestación que adeude y a sus costas, dentro de los cuatro días siguientes a los respectivos requerimientos.
\item
  Cuando la Empresa Deudora o sus administradores no sean habidos, y hayan dejado cerradas sus oficinas o establecimientos sin haber nombrado mandatario con facultades suficientes para dar cumplimiento a sus obligaciones y contestar nuevas demandas. En este caso, el demandante podrá invocar como crédito incluso aquel que se encuentre sujeto a un plazo o a una condición suspensiva.
\end{enumerate}

\hypertarget{artuxedculo-118.--requisitos.}{%
\paragraph*{Artículo 118.- Requisitos.}\label{artuxedculo-118.--requisitos.}}
\addcontentsline{toc}{paragraph}{Artículo 118.- Requisitos.}

La demanda se presentará ante el tribunal competente, señalará la causal invocada y sus hechos justificativos y acompañará los siguientes antecedentes:

\begin{enumerate}
\def\labelenumi{\arabic{enumi})}
\item
  Los documentos o antecedentes escritos que acreditan la causal invocada.
\item
  Vale vista o boleta bancaria expedida a la orden del tribunal por una suma equivalente a 100 unidades de fomento para subvenir los gastos iniciales del Procedimiento Concursal de Liquidación.
\end{enumerate}

En caso que se dicte la correspondiente Resolución de Liquidación, dicha suma será considerada como un crédito del acreedor solicitante, y gozará de la preferencia establecida en el número 4 del Artículo 2472 del Código Civil.

\begin{enumerate}
\def\labelenumi{\arabic{enumi})}
\setcounter{enumi}{2}
\item
  El acreedor peticionario podrá designar a un Veedor vigente de la Nómina de Veedores, que asumirá en caso que el Deudor se oponga a la Liquidación Forzosa. Dicho Veedor supervigilará las actividades del Deudor mientras dure la tramitación del Juicio de Oposición, conforme a lo dispuesto en el Párrafo 3 de este Título, y tendrá las facultades de interventor contenidas en el Artículo 25 de esta ley. Los honorarios del Veedor no podrán ser superiores a 100 unidades de fomento y serán de cargo del acreedor peticionario. Asimismo, el demandante podrá solicitar en su demanda cualquiera de las medidas señaladas en los \#\#\# Títulos IV y V del Libro Segundo del Código de Procedimiento Civil. El Veedor estará facultado para solicitar las medidas cautelares que estime necesarias, con cargo del acreedor peticionario, para garantizar la mantención del activo del Deudor mientras dure el Juicio de Oposición, quedando el Deudor sujeto a las restricciones señaladas en el número 2) del Artículo 57 de esta ley.
\item
  El nombre de los Liquidadores titular y suplente, para el caso que el Deudor no compareciere o no efectuare actuación alguna por escrito en la Audiencia Inicial prevista en el Artículo 120.
\end{enumerate}

El Liquidador o Veedor que hubiera ejercido como tal en algún Procedimiento Concursal, no podrá asumir en otro procedimiento respecto de un mismo Deudor.

\hypertarget{artuxedculo-119.--revisiuxf3n-primera-providencia-y-notificaciuxf3n.}{%
\paragraph*{Artículo 119.- Revisión, primera providencia y notificación.}\label{artuxedculo-119.--revisiuxf3n-primera-providencia-y-notificaciuxf3n.}}
\addcontentsline{toc}{paragraph}{Artículo 119.- Revisión, primera providencia y notificación.}

Presentada la demanda, el tribunal competente examinará en el plazo de tres días el cumplimiento de los requisitos del Artículo precedente. En caso que los considere cumplidos, la tendrá por presentada, ordenará publicarla en el Boletín Concursal y citará a las partes a una audiencia que tendrá lugar al quinto día desde la notificación personal del Deudor o la realizada conforme al Artículo 44 del Código de Procedimiento Civil, aun cuando no se encuentre en el lugar del juicio. En caso contrario, ordenará al demandante la corrección pertinente y fijará un plazo de tres días para que los subsane, bajo apercibimiento de tener por no presentada la demanda.

\hypertarget{artuxedculo-120.--audiencia-inicial.}{%
\paragraph*{Artículo 120.- Audiencia Inicial.}\label{artuxedculo-120.--audiencia-inicial.}}
\addcontentsline{toc}{paragraph}{Artículo 120.- Audiencia Inicial.}

La Audiencia Inicial se desarrollará conforme a las siguientes reglas:

\begin{enumerate}
\def\labelenumi{\arabic{enumi})}
\item
  El tribunal informará al Deudor acerca de la demanda presentada en su contra y de los efectos de un eventual Procedimiento Concursal de Liquidación.
\item
  Acto seguido, el Deudor podrá proponer por escrito o verbalmente alguna de las actuaciones señaladas en los literales siguientes, debiendo siempre señalar el nombre o razón social, domicilio y correo electrónico de sus tres acreedores, o sus representantes legales, que figuren en su contabilidad con los mayores créditos. Si el Deudor no cumple con este requisito el tribunal tendrá por no presentada la actuación y dictará de inmediato la Resolución de Liquidación, nombrando a los Liquidadores titular y suplente, ambos en carácter de provisionales, que el acreedor peticionario hubiere designado en su demanda, conforme a lo dispuesto en el número 4 del Artículo 118. Las referidas actuaciones podrán ser:
\end{enumerate}

\begin{enumerate}
\def\labelenumi{\alph{enumi})}
\item
  Consignar fondos suficientes para el pago del crédito demandado y las costas correspondientes. El tribunal tendrá por efectuada la consignación, ordenará practicar la liquidación del crédito, la regulación y tasación de las costas y señalará el plazo en que el Deudor deberá pagarlos, el que se contará desde que esas actuaciones se encuentren firmes. Si el Deudor no pagare en el plazo fijado, el tribunal dictará la respectiva Resolución de Liquidación.
\item
  Allanarse por escrito o verbalmente a la demanda, dictando en este caso el tribunal la respectiva Resolución de Liquidación.
\item
  Acogerse expresamente al Procedimiento Concursal de Reorganización contemplado en el Capítulo III de esta ley.
\item
  Oponerse a la demanda de Liquidación Forzosa, en cuyo caso se observarán las disposiciones del Párrafo 3 del presente Título. La oposición del Deudor sólo podrá fundarse en las causales previstas en el Artículo 464 del Código de Procedimiento Civil.
\end{enumerate}

\begin{enumerate}
\def\labelenumi{\arabic{enumi})}
\setcounter{enumi}{2}
\tightlist
\item
  Si el Deudor no compareciere a esta audiencia, o compareciendo no efectúa alguna de las actuaciones señaladas en el número 2 anterior, el tribunal dictará la Resolución de Liquidación y nombrará a los Liquidadores titular y suplente que el acreedor peticionario hubiere designado en su demanda, ambos en carácter de provisionales, conforme a lo dispuesto en el número 4 del Artículo 118.
\end{enumerate}

De lo obrado en esta audiencia se levantará acta, la que deberá ser firmada por los comparecientes y el secretario del tribunal.

\hypertarget{puxe1rrafo-3.-del-juicio-de-oposiciuxf3n}{%
\subsubsection*{Párrafo 3. Del Juicio de Oposición}\label{puxe1rrafo-3.-del-juicio-de-oposiciuxf3n}}
\addcontentsline{toc}{subsubsection}{Párrafo 3. Del Juicio de Oposición}

\hypertarget{artuxedculo-121.--de-la-oposiciuxf3n.}{%
\paragraph*{Artículo 121.- De la Oposición.}\label{artuxedculo-121.--de-la-oposiciuxf3n.}}
\addcontentsline{toc}{paragraph}{Artículo 121.- De la Oposición.}

En su escrito de oposición, el Deudor deberá:

\begin{enumerate}
\def\labelenumi{\arabic{enumi})}
\item
  Señalar las excepciones opuestas y defensas invocadas, así como sus fundamentos de hecho y de derecho;
\item
  Ofrecer todos los medios de prueba de que pretenda valerse, de conformidad a lo previsto en el Artículo siguiente, y
\item
  Acompañar toda la prueba documental pertinente.
\end{enumerate}

\hypertarget{artuxedculo-122.--de-las-pruebas.}{%
\paragraph*{Artículo 122.- De las pruebas.}\label{artuxedculo-122.--de-las-pruebas.}}
\addcontentsline{toc}{paragraph}{Artículo 122.- De las pruebas.}

Para acreditar las excepciones y defensas del Deudor se aplicarán a las reglas siguientes:

\begin{enumerate}
\def\labelenumi{\arabic{enumi})}
\item
  Prueba testimonial: el escrito de oposición deberá incluir la completa individualización de los testigos que depondrán, así como las razones que justifican su comparecencia.
\item
  Prueba confesional: el escrito de oposición deberá acompañar el pliego de posiciones. Si el acreedor solicitante fuere una persona jurídica, podrá comparecer cualquier persona habilitada a nombre del representante legal, siempre que exhiba en el día de la diligencia la respectiva delegación, otorgada por escritura pública y en la que conste expresamente la facultad de absolver posiciones a nombre del demandante.
\item
  Prueba pericial: se aplicarán las disposiciones de los Artículos 409, 410 y 411 del Código de Procedimiento Civil en lo referido a la procedencia de este medio de prueba. Tratándose de casos de informe pericial facultativo, el Deudor deberá exponer las razones que justifican decretar dicha diligencia.
\item
  Prueba documental: los documentos sólo podrán acompañarse junto al escrito de oposición. Con todo, el tribunal podrá aceptar la agregación de documentos con posterioridad a dicha actuación siempre que la parte que los presenta acredite que se trata de antecedentes que han surgido después de la Audiencia Inicial o que, siendo anteriores, no pudieron acompañarse oportunamente por razones independientes de su voluntad. El tribunal resolverá esta solicitud de plano, con los antecedentes que le sean proporcionados en la misma petición y contra lo resuelto no procederá recurso alguno.
\end{enumerate}

\hypertarget{artuxedculo-123.--resoluciones-del-tribunal-competente.}{%
\paragraph*{Artículo 123.- Resoluciones del tribunal competente.}\label{artuxedculo-123.--resoluciones-del-tribunal-competente.}}
\addcontentsline{toc}{paragraph}{Artículo 123.- Resoluciones del tribunal competente.}

Deducida la oposición, el tribunal constatará el cumplimiento de los requisitos legales y, si procede, tendrá por opuesto al Deudor a la Liquidación Forzosa y por acompañados los documentos regulados en el Artículo anterior. En caso contrario, se estará a lo dispuesto en el número 3 del Artículo 120.

\hypertarget{artuxedculo-124.--truxe1mites-probatorios.}{%
\paragraph*{Artículo 124.- Trámites probatorios.}\label{artuxedculo-124.--truxe1mites-probatorios.}}
\addcontentsline{toc}{paragraph}{Artículo 124.- Trámites probatorios.}

Una vez decretada la oposición, el tribunal competente:

\begin{enumerate}
\def\labelenumi{\arabic{enumi})}
\item
  Existiendo hechos sustanciales, pertinentes y controvertidos que requieran ser probados para una adecuada resolución de la controversia, recibirá la causa a prueba y fijará los puntos sobre los cuales ésta deberá recaer. Dicha resolución sólo será susceptible de recurso de reposición por las partes, el que deberá interponerse dentro de tercero día. En caso contrario, citará a las partes a la Audiencia de Fallo.
\item
  Una vez recibida la causa a prueba y fijados los puntos sobre los cuales deberá recaer:
\end{enumerate}

\begin{enumerate}
\def\labelenumi{\alph{enumi})}
\item
  Se pronunciará acerca de la admisibilidad y pertinencia de las pruebas ofrecidas;
\item
  Tratándose de prueba pericial, el tribunal determinará la calidad del perito y los puntos sobre los cuales deberá pronunciarse, instando a las partes para que acuerden su nombre. En caso de desacuerdo, el perito deberá ser designado en ese mismo acto por el tribunal, y se fijará un plazo de siete días para que el perito evacue su informe. No será necesario en estos casos practicar la audiencia de reconocimiento.
\item
  Concederá al acreedor demandante la oportunidad de ofrecer prueba, la que deberá ser singularizada y acompañada al día siguiente. La resolución acerca de la admisibilidad y pertinencia de las pruebas del acreedor deberá ser pronunciada antes de la Audiencia de Prueba. Contra lo resuelto, el Deudor podrá interponer un recurso de reposición en la forma prevista en el Artículo 125, tramitándose tal petición como cuestión previa.
\end{enumerate}

\begin{enumerate}
\def\labelenumi{\arabic{enumi})}
\setcounter{enumi}{2}
\tightlist
\item
  Citará a las partes a una Audiencia de Prueba, la que deberá tener lugar al quinto día siguiente, debiendo indicar la fecha y la hora de celebración. Las partes se entenderán notificadas en ese mismo acto.
\end{enumerate}

En caso de que se fijen nuevos puntos de prueba por haberse acogido la reposición señalada en el número 1) anterior, el tribunal deberá resolver la admisibilidad o pertinencia de las nuevas pruebas antes de la Audiencia de Prueba señalada en el Artículo 126.

\hypertarget{artuxedculo-125.--recursos.}{%
\paragraph*{Artículo 125.- Recursos.}\label{artuxedculo-125.--recursos.}}
\addcontentsline{toc}{paragraph}{Artículo 125.- Recursos.}

En contra de las resoluciones que se pronuncien en la Audiencia Inicial acerca de la admisibilidad o procedencia de las pruebas ofrecidas, los puntos de prueba fijados, la forma de hacer valer los medios probatorios o cualquier otra circunstancia que incida en éstos, sólo será procedente el recurso de reposición, que deberá deducirse verbalmente por las partes y será resuelto en la misma Audiencia Inicial.

\hypertarget{artuxedculo-126.--audiencia-de-prueba.}{%
\paragraph*{Artículo 126.- Audiencia de Prueba.}\label{artuxedculo-126.--audiencia-de-prueba.}}
\addcontentsline{toc}{paragraph}{Artículo 126.- Audiencia de Prueba.}

A la hora decretada y con las partes que asistan, se rendirá la prueba declarada admisible en el siguiente orden: confesional y testimonial, iniciándose por la ofrecida por el Deudor.

Sólo se admitirá la declaración de dos testigos por cada parte respecto de cada punto de prueba. Serán aplicables las reglas de los Artículos 356 y siguientes del Código de Procedimiento Civil respecto de la rendición de la prueba testimonial y lo dispuesto en los Artículos 385 y siguientes del mismo Código en relación a la prueba confesional.
Concluida la recepción de la prueba, las partes formularán verbal y brevemente las observaciones que el examen de la misma les sugiera, de un modo preciso y concreto.
La Audiencia de Prueba terminará con la firma de un acta por los asistentes, el juez y el secretario del tribunal. Desde aquel momento, las partes asistentes y las que no hayan asistido se entenderán citadas y notificadas de pleno derecho a la Audiencia de Fallo, la que deberá celebrarse al décimo día contado desde el término de la Audiencia de Prueba, existan o no diligencias pendientes, debiendo el tribunal fijar su hora de inicio.
Las pruebas señaladas se apreciaran por el tribunal de acuerdo a las reglas de la sana crítica.

\hypertarget{artuxedculo-127.--de-la-audiencia-de-fallo.}{%
\paragraph*{Artículo 127.- De la Audiencia de Fallo.}\label{artuxedculo-127.--de-la-audiencia-de-fallo.}}
\addcontentsline{toc}{paragraph}{Artículo 127.- De la Audiencia de Fallo.}

La Audiencia de Fallo se celebrará con las partes que asistan y en ella se dictará la sentencia definitiva de primera instancia, la que será notificada a las partes. El secretario del tribunal certificará el hecho de su pronunciamiento, la asistencia de las partes y la copia autorizada que se les entregará de la sentencia definitiva. La parte inasistente se entenderá notificada de pleno derecho con el solo mérito de la celebración de la audiencia.

\hypertarget{artuxedculo-128.--de-la-sentencia-definitiva.}{%
\paragraph*{Artículo 128.- De la sentencia definitiva.}\label{artuxedculo-128.--de-la-sentencia-definitiva.}}
\addcontentsline{toc}{paragraph}{Artículo 128.- De la sentencia definitiva.}

La sentencia definitiva que acoja la oposición del Deudor deberá cumplir con lo dispuesto en el Artículo 170 del Código de Procedimiento Civil y, con ocasión de ella, cesará en sus funciones el Veedor. Contra esta sentencia procederá únicamente el recurso de apelación, el que se concederá en ambos efectos y gozará de preferencia extraordinaria para su inclusión a la tabla y para su vista y fallo. Contra el fallo de segunda instancia no procederá recurso alguno, sea ordinario o extraordinario.

La sentencia definitiva que rechace la oposición del Deudor ordenará su liquidación en los términos del Artículo 129 y una vez notificada, el Veedor propuesto en conformidad a lo dispuesto en el número 3 del Artículo 118 cesará en su cargo.

Acogida la oposición del Deudor, éste podrá demandar indemnización de perjuicios al demandante, a su representante legal, o al administrador solicitante, si probare que procedió culpable o dolosamente.

\hypertarget{artuxedculo-129.--resoluciuxf3n-de-liquidaciuxf3n.}{%
\paragraph{Artículo 129.- Resolución de Liquidación.}\label{artuxedculo-129.--resoluciuxf3n-de-liquidaciuxf3n.}}

La Resolución de Liquidación contendrá, además de lo establecido en los Artículos 169 y 170 del Código de Procedimiento Civil, lo siguiente:

\begin{enumerate}
\def\labelenumi{\arabic{enumi})}
\item
  En caso de ser procedente, las consideraciones de hecho o de derecho que sirven de fundamento para el rechazo de las excepciones opuestas por el Deudor.
\item
  La determinación de si el Deudor es una Empresa Deudora, individualizándola.
\item
  La designación de un Liquidador titular y de uno suplente, ambos en carácter de provisionales de acuerdo a lo establecido en el Artículo 37 de esta ley, y la orden al Liquidador para que incaute todos los bienes del Deudor, sus libros y documentos bajo inventario, y de que se le preste, para este objeto, el auxilio de la fuerza pública, con la exhibición de la copia autorizada de la Resolución de Liquidación.
\item
  La orden para que las oficinas de correos entreguen al Liquidador la correspondencia cuyo destinatario sea el Deudor.
\item
  La orden de acumular al Procedimiento Concursal de Liquidación todos los juicios pendientes contra el deudor que puedan afectar sus bienes, seguidos ante otros tribunales de cualquier jurisdicción, salvo las excepciones legales.
\item
  La advertencia al público que no pague ni entregue mercaderías al Deudor, bajo pena de nulidad de los pagos y entregas, y la orden a las personas que tengan bienes o documentos pertenecientes al Deudor para que los pongan, dentro de tercero día, a disposición del Liquidador.
\item
  La orden de informar a todos los acreedores residentes en el territorio de la República que tienen el plazo de treinta días contado desde la fecha de la publicación de la Resolución de Liquidación, para que se presenten con los documentos justificativos de sus créditos bajo apercibimiento de ser afectados por los resultados del juicio sin nueva citación.
\item
  La orden de notificar, por el medio más expedito posible, la Resolución de Liquidación a los acreedores que se hallen fuera del territorio de la República.
\item
  La orden de inscribir la Resolución de Liquidación en los conservadores de bienes raíces correspondientes a cada uno de los inmuebles pertenecientes al Deudor, y de anotarla al margen de la inscripción social de la Empresa Deudora en el Registro de Comercio, si fuere procedente.
\item
  La indicación precisa del lugar, día y hora en que se celebrará la primera Junta de Acreedores.
\end{enumerate}

La Resolución de Liquidación se notificará al Deudor, a los acreedores y a terceros por medio de su publicación en el Boletín Concursal y contra ella procederá únicamente el recurso de apelación, el que se concederá en el solo efecto devolutivo y gozará de preferencia para su agregación extraordinaria a la tabla, y para su vista y fallo. Contra el fallo de segunda instancia no procederá recurso alguno, sea ordinario o extraordinario.

\hypertarget{puxe1rrafo-4.-de-los-efectos-de-la-resoluciuxf3n-de-liquidaciuxf3n}{%
\subsubsection*{Párrafo 4. De los efectos de la Resolución de Liquidación}\label{puxe1rrafo-4.-de-los-efectos-de-la-resoluciuxf3n-de-liquidaciuxf3n}}
\addcontentsline{toc}{subsubsection}{Párrafo 4. De los efectos de la Resolución de Liquidación}

\hypertarget{artuxedculo-130.--administraciuxf3n-de-bienes.}{%
\paragraph*{Artículo 130.- Administración de bienes.}\label{artuxedculo-130.--administraciuxf3n-de-bienes.}}
\addcontentsline{toc}{paragraph}{Artículo 130.- Administración de bienes.}

Desde la dictación de la Resolución de Liquidación se producirán los siguientes efectos en relación al Deudor y a sus bienes:

\begin{enumerate}
\def\labelenumi{\arabic{enumi})}
\tightlist
\item
  Quedará inhibido de pleno derecho de la administración de todos sus bienes presentes, esto es, aquellos sujetos al Procedimiento Concursal de Liquidación y existentes en su patrimonio a la época de la dictación de esta resolución, excluidos aquellos que la ley declare inembargables. Su administración pasará de pleno derecho al Liquidador.
\end{enumerate}

En consecuencia, serán nulos los actos y contratos posteriores que el Deudor ejecute o celebre en relación a estos bienes.

\begin{enumerate}
\def\labelenumi{\arabic{enumi})}
\setcounter{enumi}{1}
\item
  No perderá el dominio sobre sus bienes, sino sólo la facultad de disposición sobre ellos y sobre sus frutos.
\item
  No podrá comparecer en juicio como demandante ni como demandado, en lo relativo a los bienes objeto del Procedimiento Concursal de Liquidación, pero podrá actuar como coadyuvante.
\item
  Podrá interponer por sí todas las acciones que se refieran exclusivamente a su persona y que tengan por objeto derechos inherentes a ella. Tampoco será privado del ejercicio de sus derechos civiles, ni se le impondrán inhabilidades especiales sino en los casos expresamente determinados por las leyes.
\item
  En caso de negligencia del Liquidador, podrá solicitar al tribunal que ordene la ejecución de las providencias conservativas que fueren pertinentes.
\end{enumerate}

\hypertarget{artuxedculo-131.--resoluciuxf3n-de-controversias-entre-partes.}{%
\paragraph*{Artículo 131.- Resolución de controversias entre partes.}\label{artuxedculo-131.--resoluciuxf3n-de-controversias-entre-partes.}}
\addcontentsline{toc}{paragraph}{Artículo 131.- Resolución de controversias entre partes.}

Todas las cuestiones que se susciten entre el Deudor, el Liquidador y cualquier otro interesado en relación a la administración de los bienes sujetos al Procedimiento Concursal de Liquidación serán resueltas por el tribunal en audiencias verbales, a solicitud del interesado y conforme a las reglas que siguen:

\begin{enumerate}
\def\labelenumi{\alph{enumi})}
\item
  El solicitante deberá exponer por escrito al tribunal tanto la petición que formula como los antecedentes que le sirven de sustento.
\item
  El tribunal analizará la petición y podrá desecharla de plano si considera que carece de fundamento plausible.
\item
  En caso contrario, citará a las partes a una audiencia verbal que se notificará por el Estado Diario, se publicará por el Liquidador en el Boletín Concursal y se celebrará en el menor tiempo posible.
\item
  El Liquidador podrá comparecer personalmente o a través de su apoderado judicial. La audiencia se celebrará con las partes que asistan y la resolución que adopte el tribunal sólo será susceptible de reposición, la que deberá deducirse y resolverse en la misma audiencia.
\end{enumerate}

\hypertarget{artuxedculo-132.--administraciuxf3n-de-bienes-en-caso-de-usufructo-legal.}{%
\paragraph*{Artículo 132.- Administración de bienes en caso de usufructo legal.}\label{artuxedculo-132.--administraciuxf3n-de-bienes-en-caso-de-usufructo-legal.}}
\addcontentsline{toc}{paragraph}{Artículo 132.- Administración de bienes en caso de usufructo legal.}

La administración que conserva el Deudor sobre los bienes personales de la mujer o hijos de los que tenga el usufructo legal, quedará sujeta a la intervención del Liquidador mientras subsista el derecho del marido, padre o madre sujeto al Procedimiento Concursal de Liquidación.

El Liquidador cuidará que los frutos líquidos que produzcan estos bienes ingresen a la masa, deducidas las cargas legales o convencionales que los graven.

El tribunal, con audiencia del Liquidador y del Deudor, determinará la cuota de los frutos que correspondan a este último para su subsistencia y la de su familia, habida consideración de sus necesidades y la cuantía de los bienes bajo intervención.

El Liquidador podrá comparecer como parte coadyuvante en los juicios de separación de bienes y de divorcio en que el Deudor sea demandado o demandante.

\hypertarget{artuxedculo-133.--situaciuxf3n-de-los-bienes-futuros.}{%
\paragraph*{Artículo 133.- Situación de los bienes futuros.}\label{artuxedculo-133.--situaciuxf3n-de-los-bienes-futuros.}}
\addcontentsline{toc}{paragraph}{Artículo 133.- Situación de los bienes futuros.}

La administración de los bienes que adquiera el Deudor con posterioridad a la Resolución de Liquidación se regirá por las reglas que siguen:

\begin{enumerate}
\def\labelenumi{\alph{enumi})}
\item
  Tratándose de bienes adquiridos a Título gratuito, dicha administración se ejercerá por el Liquidador, manteniéndose la responsabilidad por las cargas con que le hayan sido transferidos o transmitidos y sin perjuicio de los derechos de los acreedores hereditarios.
\item
  Tratándose de bienes adquiridos a Título oneroso, su administración podrá ser sometida a intervención, y los acreedores sólo tendrán derecho a los beneficios líquidos que se obtengan.
\end{enumerate}

\hypertarget{artuxedculo-134.--fijaciuxf3n-de-derechos-de-acreedores.}{%
\paragraph*{Artículo 134.- Fijación de derechos de acreedores.}\label{artuxedculo-134.--fijaciuxf3n-de-derechos-de-acreedores.}}
\addcontentsline{toc}{paragraph}{Artículo 134.- Fijación de derechos de acreedores.}

La Resolución de Liquidación fija irrevocablemente los derechos de todos los acreedores en el estado que tenían al día de su pronunciamiento, salvo las excepciones legales.

\hypertarget{artuxedculo-135.--suspensiuxf3n-de-ejecuciones-individuales.}{%
\paragraph*{Artículo 135.- Suspensión de ejecuciones individuales.}\label{artuxedculo-135.--suspensiuxf3n-de-ejecuciones-individuales.}}
\addcontentsline{toc}{paragraph}{Artículo 135.- Suspensión de ejecuciones individuales.}

La dictación de la Resolución de Liquidación suspende el derecho de los acreedores para ejecutar individualmente al Deudor.

Con todo, los acreedores hipotecarios y prendarios podrán deducir o continuar sus acciones en los bienes gravados con hipoteca o prenda, sin perjuicio de la posibilidad de realizarlos en el Procedimiento Concursal de Liquidación. En ambos casos, para percibir deberán garantizar el pago de los créditos de primera clase que hayan sido verificados ordinariamente o antes de la fecha de liquidación de los bienes afectos a sus respectivas garantías, por los montos que en definitiva resulten reconocidos.

\hypertarget{artuxedculo-136.--exigibilidad-y-reajustabilidad-de-obligaciones.}{%
\paragraph*{Artículo 136.- Exigibilidad y reajustabilidad de obligaciones.}\label{artuxedculo-136.--exigibilidad-y-reajustabilidad-de-obligaciones.}}
\addcontentsline{toc}{paragraph}{Artículo 136.- Exigibilidad y reajustabilidad de obligaciones.}

Una vez dictada la Resolución de Liquidación, todas las obligaciones dinerarias se entenderán vencidas y actualmente exigibles respecto del Deudor, para que los acreedores puedan verificarlas en el Procedimiento Concursal de Liquidación y percibir el pago de sus acreencias. Estas últimas se pagarán según su valor actual más los reajustes e intereses que correspondan, de conformidad a las reglas del Artículo siguiente.

\hypertarget{artuxedculo-137.--determinaciuxf3n-del-valor-actual-de-los-cruxe9ditos.}{%
\paragraph*{Artículo 137.- Determinación del valor actual de los créditos.}\label{artuxedculo-137.--determinaciuxf3n-del-valor-actual-de-los-cruxe9ditos.}}
\addcontentsline{toc}{paragraph}{Artículo 137.- Determinación del valor actual de los créditos.}

Para determinar el valor actual de los créditos se seguirán las siguientes reglas:

\begin{enumerate}
\def\labelenumi{\arabic{enumi})}
\item
  El valor actual de los créditos reajustables en moneda nacional, no vencidos a la fecha de la dictación de la Resolución de Liquidación y que devenguen intereses, será el capital más el reajuste convenido e intereses para operaciones reajustables devengados hasta la fecha de dicha resolución.
\item
  El valor actual de los créditos reajustables en moneda nacional, no vencidos a la fecha de la dictación de la Resolución de Liquidación y que no devenguen intereses, será el capital más el reajuste convenido hasta la fecha de dicha resolución.
\item
  El valor actual de los créditos no reajustables en moneda nacional, no vencidos a la fecha de la dictación de la Resolución de Liquidación y que devenguen intereses, será el capital más los intereses para operaciones no reajustables devengados hasta la fecha de dicha resolución.
\item
  El valor actual de los créditos no reajustables en moneda nacional, no vencidos a la fecha de la dictación de la Resolución de Liquidación y que no devenguen intereses, se determinará descontando del capital los intereses corrientes para operaciones de crédito de dinero no reajustables desde la fecha de la Resolución de Liquidación hasta el día de los respectivos vencimientos. Si no fuere posible determinar el índice de reajustabilidad o si éste hubiere perdido su vigencia, se aplicará lo dispuesto en el número 3) anterior.
\end{enumerate}

\hypertarget{artuxedculo-138.--exigibilidad-de-otros-instrumentos.}{%
\paragraph*{Artículo 138.- Exigibilidad de otros instrumentos.}\label{artuxedculo-138.--exigibilidad-de-otros-instrumentos.}}
\addcontentsline{toc}{paragraph}{Artículo 138.- Exigibilidad de otros instrumentos.}

Si el Deudor fuere aceptante de una letra de cambio, librador de una letra no aceptada o suscriptor de un pagaré, los demás obligados deberán pagar dichos instrumentos inmediatamente.

\hypertarget{artuxedculo-139.--reajuste-y-cuxe1lculo-de-intereses.}{%
\paragraph*{Artículo 139.- Reajuste y cálculo de intereses.}\label{artuxedculo-139.--reajuste-y-cuxe1lculo-de-intereses.}}
\addcontentsline{toc}{paragraph}{Artículo 139.- Reajuste y cálculo de intereses.}

En virtud de la dictación de la Resolución de Liquidación y desde la fecha de ésta, las acreencias del Deudor, vencidas y las actualizadas de conformidad con el Artículo 137:
1) Se reajustarán y devengarán intereses según lo pactado en la convención, en el caso del número 1) del Artículo 137.

\begin{enumerate}
\def\labelenumi{\arabic{enumi})}
\setcounter{enumi}{1}
\item
  Se reajustarán según lo pactado, en el caso del número 2) del mismo Artículo.
\item
  Devengarán intereses corrientes para operaciones de crédito de dinero no reajustables en el caso de los números 3) y 4) del Artículo 137.
\end{enumerate}

El Liquidador podrá impugnar los intereses pactados en caso de estimarlos excesivos.
Las obligaciones contraídas en moneda extranjera se pagarán en la misma moneda establecida en la convención y devengarán el interés pactado en ella.

Los reajustes y los intereses, en su caso, gozarán de iguales preferencias que el respectivo capital al cual acceden.

Sin embargo, los intereses que se devenguen con posterioridad a la dictación de la Resolución de Liquidación quedarán pospuestos para su pago hasta que se pague el capital de los demás créditos en el Procedimiento Concursal de Liquidación.

\hypertarget{artuxedculo-140.--compensaciones.}{%
\paragraph*{Artículo 140.- Compensaciones.}\label{artuxedculo-140.--compensaciones.}}
\addcontentsline{toc}{paragraph}{Artículo 140.- Compensaciones.}

La dictación de la Resolución de Liquidación impide toda compensación que no hubiere operado antes por el ministerio de la ley, entre las obligaciones recíprocas del Deudor y los acreedores, salvo que se trate de obligaciones conexas, derivadas de un mismo contrato o de una misma negociación y aunque sean exigibles en diferentes plazos.

Para estos efectos, se entenderá que revisten el carácter de obligaciones conexas aquellas que, aun siendo en distinta moneda, emanen de operaciones de derivados, tales como futuros, opciones, swaps, forwards u otros instrumentos o contratos de derivados suscritos entre las mismas partes, en una o más oportunidades, bajo ley chilena o extranjera, al amparo de un mismo convenio marco de contratación de los reconocidos por el Banco Central y que incluyan un acuerdo de compensación en caso de Liquidación Voluntaria o de Liquidación Forzosa. El Banco Central de Chile podrá determinar los términos y condiciones generales de los convenios marco de contratación referidos en que sea parte una empresa bancaria o cualquier otro inversionista institucional, considerando para ello los convenios de general aceptación en los mercados internacionales.

Cada una de las obligaciones que emanen de operaciones de derivados efectuadas en la forma antedicha, se entenderá de plazo vencido, líquida y actualmente exigible a la fecha de la dictación de la Resolución de Liquidación y su valor se calculará a dicha fecha de acuerdo a sus términos y condiciones. Luego, las compensaciones que operen por aplicación del inciso precedente serán calculadas y ejecutadas simultáneamente en dicha fecha.

En caso que una de las partes sea un banco establecido en Chile, sólo procederá dicha compensación tratándose de operaciones con productos derivados cuyos términos y condiciones se encuentren autorizados por el Banco Central de Chile.

\hypertarget{artuxedculo-141.--derecho-legal-de-retenciuxf3n-en-el-contrato-de-arrendamiento.}{%
\paragraph*{Artículo 141.- Derecho legal de retención en el contrato de arrendamiento.}\label{artuxedculo-141.--derecho-legal-de-retenciuxf3n-en-el-contrato-de-arrendamiento.}}
\addcontentsline{toc}{paragraph}{Artículo 141.- Derecho legal de retención en el contrato de arrendamiento.}

El derecho legal de retención no podrá ser declarado después de la Resolución de Liquidación.

Durante los treinta días siguientes a la notificación de dicha resolución, el arrendador no podrá perseguir la realización de los bienes muebles destinados a la explotación de los negocios del Deudor por los arrendamientos vencidos, sin perjuicio de su derecho para solicitar providencias conservativas, las que deberán ser resueltas por el tribunal de conformidad al Artículo 131.

Si el arrendamiento ha expirado por alguna causa legal, el arrendador podrá exigir la entrega del inmueble y entablar las acciones correspondientes.

\hypertarget{artuxedculo-142.--regla-general-de-acumulaciuxf3n-al-procedimiento-concursal-de-liquidaciuxf3n.}{%
\paragraph*{Artículo 142.- Regla general de acumulación al Procedimiento Concursal de Liquidación.}\label{artuxedculo-142.--regla-general-de-acumulaciuxf3n-al-procedimiento-concursal-de-liquidaciuxf3n.}}
\addcontentsline{toc}{paragraph}{Artículo 142.- Regla general de acumulación al Procedimiento Concursal de Liquidación.}

Todos los juicios civiles pendientes contra el Deudor ante otros tribunales se acumularán al Procedimiento Concursal de Liquidación. Los que se inicien con posterioridad a la notificación de la Resolución de Liquidación se promoverán ante el tribunal que esté conociendo del Procedimiento Concursal de Liquidación.

Los juicios civiles acumulados al Procedimiento Concursal de Liquidación seguirán tramitándose con arreglo al procedimiento que corresponda según su naturaleza, hasta que quede ejecutoriada la sentencia definitiva.

\hypertarget{artuxedculo-143.--excepciones.}{%
\paragraph*{Artículo 143.- Excepciones.}\label{artuxedculo-143.--excepciones.}}
\addcontentsline{toc}{paragraph}{Artículo 143.- Excepciones.}

La regla de acumulación indicada en el Artículo anterior no se aplicará a los siguientes juicios, que seguirán tramitándose o deberán sustanciarse ante el tribunal competente, respectivamente:

\begin{enumerate}
\def\labelenumi{\arabic{enumi})}
\item
  Los que a la fecha estuvieren siendo conocidos por árbitros.
\item
  Los que fueren materias de arbitraje forzoso.
\item
  Aquellos sometidos por ley a tribunales especiales.
\end{enumerate}

En caso que el Deudor fuere condenado en alguno de los juicios acumulados al Procedimiento Concursal de Liquidación, el Liquidador dará cumplimiento a lo resuelto de conformidad a las disposiciones de esta ley.

\hypertarget{artuxedculo-144.--acumulaciuxf3n-de-juicios-ejecutivos-en-obligaciones-de-dar.}{%
\paragraph*{Artículo 144.- Acumulación de juicios ejecutivos en obligaciones de dar.}\label{artuxedculo-144.--acumulaciuxf3n-de-juicios-ejecutivos-en-obligaciones-de-dar.}}
\addcontentsline{toc}{paragraph}{Artículo 144.- Acumulación de juicios ejecutivos en obligaciones de dar.}

La acumulación al Procedimiento Concursal de Liquidación de esta clase de juicios se sujetará a las reglas siguientes:

\begin{enumerate}
\def\labelenumi{\arabic{enumi})}
\tightlist
\item
  Si no existieren excepciones opuestas, los juicios se suspenderán en el estado en que se encuentren al momento de notificarse la Resolución de Liquidación.
\end{enumerate}

El tribunal de la ejecución pronunciará una resolución que suspenderá la tramitación y ordenará remitir los expedientes al tribunal que esté conociendo del Procedimiento Concursal de Liquidación para que continúe su tramitación. En tal caso, los acreedores ejecutantes verificarán sus créditos conforme a las reglas generales.

\begin{enumerate}
\def\labelenumi{\arabic{enumi})}
\setcounter{enumi}{1}
\tightlist
\item
  Si existieren excepciones opuestas, el tribunal de la ejecución ordenará remitir los expedientes al tribunal que esté conociendo del Procedimiento Concursal de Liquidación y, una vez recibidos, se seguirá adelante en su tramitación particular hasta la resolución de término. En tal caso, el Liquidador asumirá la representación judicial del Deudor y los acreedores ejecutantes podrán verificar sus créditos en forma condicional.
\end{enumerate}

\hypertarget{artuxedculo-145.--acumulaciuxf3n-de-juicios-ejecutivos-en-obligaciones-de-hacer.}{%
\paragraph*{Artículo 145.- Acumulación de juicios ejecutivos en obligaciones de hacer.}\label{artuxedculo-145.--acumulaciuxf3n-de-juicios-ejecutivos-en-obligaciones-de-hacer.}}
\addcontentsline{toc}{paragraph}{Artículo 145.- Acumulación de juicios ejecutivos en obligaciones de hacer.}

La acumulación al Procedimiento Concursal de Liquidación de esta clase de juicios se sujetará a las siguientes reglas:

\begin{enumerate}
\def\labelenumi{\arabic{enumi})}
\item
  Si los fondos para dar cumplimiento al objeto del litigio se encontraren depositados antes de la notificación de la Resolución de Liquidación, el tribunal de la ejecución ordenará remitir los expedientes al tribunal que esté conociendo del Procedimiento Concursal de Liquidación, continuándose la tramitación hasta la inversión total de los fondos o la conclusión de la obra que con ellos deba pagarse.
\item
  En caso contrario, los juicios se acumularán sin importar el estado en que se encuentren y el acreedor sólo podrá verificar el monto de los perjuicios que el tribunal respectivo hubiere declarado o que se declaren con posterioridad por el tribunal que conoce del Procedimiento Concursal de Liquidación.
\end{enumerate}

\hypertarget{artuxedculo-146.--norma-comuxfan-para-juicios-ejecutivos.}{%
\paragraph*{Artículo 146.- Norma común para juicios ejecutivos.}\label{artuxedculo-146.--norma-comuxfan-para-juicios-ejecutivos.}}
\addcontentsline{toc}{paragraph}{Artículo 146.- Norma común para juicios ejecutivos.}

Si entre los ejecutados existieren personas distintas del Deudor, el tribunal de la ejecución deberá:

\begin{enumerate}
\def\labelenumi{\arabic{enumi})}
\item
  Suspender la tramitación sólo respecto del Deudor;
\item
  Remitir al tribunal que esté conociendo del Procedimiento Concursal de Liquidación copias autorizadas del expediente, para que continúe la sustanciación respecto del Deudor, y
\item
  Conservar para sí el expediente original a fin de continuar la ejecución de los restantes demandados.
\end{enumerate}

\hypertarget{artuxedculo-147.--juicios-iniciados-por-el-deudor.}{%
\paragraph*{Artículo 147.- Juicios iniciados por el Deudor.}\label{artuxedculo-147.--juicios-iniciados-por-el-deudor.}}
\addcontentsline{toc}{paragraph}{Artículo 147.- Juicios iniciados por el Deudor.}

Las demandas que se hubieren interpuesto por el Deudor antes de la Resolución de Liquidación, para controvertir la validez, legitimidad o procedencia de los créditos justificativos de la Liquidación Forzosa deberán acumularse al Procedimiento Concursal de Liquidación.

Si en tales juicios las alegaciones del Deudor fueren similares a las de su oposición, planteada de conformidad al Artículo 121, el tribunal que esté conociendo del Procedimiento Concursal de Liquidación deberá resolver ambas controversias en un mismo fallo. En lo meramente procesal, prevalecerán las disposiciones propias del juicio de oposición.

\hypertarget{artuxedculo-148.--principio-general-de-las-medidas-cautelares.}{%
\paragraph*{Artículo 148.- Principio general de las medidas cautelares.}\label{artuxedculo-148.--principio-general-de-las-medidas-cautelares.}}
\addcontentsline{toc}{paragraph}{Artículo 148.- Principio general de las medidas cautelares.}

Los embargos y medidas precautorias decretadas en los juicios sustanciados contra el Deudor y que afecten a bienes que deban realizarse o ingresar al Procedimiento Concursal de Liquidación, quedarán sin efecto desde que se dicte la Resolución de Liquidación.
En caso de acumulación, sólo el Liquidador podrá solicitar el alzamiento respectivo ante el tribunal que lo decretó o ante el tribunal que esté conociendo del Procedimiento Concursal de Liquidación. El tribunal correspondiente decretará el alzamiento sin más trámite, con el sólo mérito de la dictación ya indicada.

\hypertarget{artuxedculo-149.--medidas-cautelares-en-sede-criminal.}{%
\paragraph*{Artículo 149.- Medidas cautelares en sede criminal.}\label{artuxedculo-149.--medidas-cautelares-en-sede-criminal.}}
\addcontentsline{toc}{paragraph}{Artículo 149.- Medidas cautelares en sede criminal.}

Aquellas medidas cautelares concedidas con ocasión de acciones de naturaleza criminal provenientes de los ilícitos contemplados en el Título IX del Libro Segundo del Código Penal, que afecten a bienes del Deudor para responder o garantizar el pago de futuras indemnizaciones civiles, multas o cualquier otra condena en dinero, quedarán sin efecto tan pronto el Liquidador comunique por escrito al Juzgado de Garantía que corresponda que se ha pronunciado la Resolución de Liquidación, adjuntando los documentos que sirvan para acreditarla. Este tribunal entregará los bienes al Liquidador para su administración y proseguirá la tramitación de los respectivos procedimientos, en los cuales el Liquidador actuará como coadyuvante cuando se trate de delitos concursales.

Las multas e indemnizaciones pecuniarias que eventualmente se concedan, cualquiera sea su especie, deberán verificarse en el Procedimiento Concursal de Liquidación conforme a las reglas generales.

\hypertarget{artuxedculo-150.--de-la-reivindicaciuxf3n.}{%
\paragraph*{Artículo 150.- De la Reivindicación.}\label{artuxedculo-150.--de-la-reivindicaciuxf3n.}}
\addcontentsline{toc}{paragraph}{Artículo 150.- De la Reivindicación.}

Fuera de los casos mencionados en los Artículos siguientes, podrán entablarse las acciones reivindicatorias que procedan, en conformidad a las reglas generales.

Las tercerías de dominio que estuvieren iniciadas a la fecha de dictación de la Resolución de Liquidación continuarán tramitándose en conformidad al procedimiento que corresponda.

\hypertarget{artuxedculo-151.--reivindicaciuxf3n-de-efectos-de-comercio.}{%
\paragraph*{Artículo 151.- Reivindicación de efectos de comercio.}\label{artuxedculo-151.--reivindicaciuxf3n-de-efectos-de-comercio.}}
\addcontentsline{toc}{paragraph}{Artículo 151.- Reivindicación de efectos de comercio.}

Podrán ser reivindicados los efectos de comercio y cualquier otro documento de crédito no pagado y existente a la fecha de dictación de la Resolución de Liquidación, en poder del Deudor o de un tercero que los conserve a nombre de éste, y siempre que el propietario los haya entregado o remitido al Deudor por un Título no traslaticio de dominio.

\hypertarget{artuxedculo-152.--reivindicaciuxf3n-de-mercaderuxedas.}{%
\paragraph*{Artículo 152.- Reivindicación de mercaderías.}\label{artuxedculo-152.--reivindicaciuxf3n-de-mercaderuxedas.}}
\addcontentsline{toc}{paragraph}{Artículo 152.- Reivindicación de mercaderías.}

Podrán ser también reivindicadas, en todo o en parte y mientras puedan ser identificadas, las mercaderías consignadas al Deudor a Título de depósito, comisión de venta o a cualquier otro que no transfiera el dominio.

Vendidas las mercaderías, el propietario de ellas podrá reivindicar el precio o la parte del precio que no hubiere sido pagado o compensado entre el Deudor y el comprador a la fecha de la Resolución de Liquidación.

No se entiende pagado el precio por la simple dación de documentos de crédito, firmados o transferidos por el comprador a favor del Deudor. Si existieren tales documentos en poder de éste, el propietario podrá reivindicarlos, siempre que acredite su origen e identidad.

\hypertarget{artuxedculo-153.--derecho-legal-de-retenciuxf3n-del-deudor.}{%
\paragraph*{Artículo 153.- Derecho legal de retención del Deudor.}\label{artuxedculo-153.--derecho-legal-de-retenciuxf3n-del-deudor.}}
\addcontentsline{toc}{paragraph}{Artículo 153.- Derecho legal de retención del Deudor.}

Lo dispuesto en los Artículos 151 y 152 precedentes no obsta al derecho legal de retención o al de prenda que corresponda al Deudor.

\hypertarget{artuxedculo-154.--resoluciuxf3n-de-la-compraventa.}{%
\paragraph*{Artículo 154.- Resolución de la compraventa.}\label{artuxedculo-154.--resoluciuxf3n-de-la-compraventa.}}
\addcontentsline{toc}{paragraph}{Artículo 154.- Resolución de la compraventa.}

El contrato de compraventa podrá resolverse por incumplimiento de las obligaciones del Deudor comprador, salvo cuando se trate de cosas muebles que hayan llegado a poder de éste.

\hypertarget{artuxedculo-155.--definiciuxf3n-de-mercaderuxeda-en-truxe1nsito.}{%
\paragraph*{Artículo 155.- Definición de mercadería en tránsito.}\label{artuxedculo-155.--definiciuxf3n-de-mercaderuxeda-en-truxe1nsito.}}
\addcontentsline{toc}{paragraph}{Artículo 155.- Definición de mercadería en tránsito.}

Para los efectos de lo dispuesto en los Artículos siguientes, se entiende que las cosas muebles están en tránsito desde el momento en que las reciben los agentes encargados de su conducción, hasta que queden en poder del comprador Deudor o de la persona que lo represente.

\hypertarget{artuxedculo-156.--facultades-del-vendedor-respecto-de-las-mercaderuxedas-en-truxe1nsito.}{%
\paragraph*{Artículo 156.- Facultades del vendedor respecto de las mercaderías en tránsito.}\label{artuxedculo-156.--facultades-del-vendedor-respecto-de-las-mercaderuxedas-en-truxe1nsito.}}
\addcontentsline{toc}{paragraph}{Artículo 156.- Facultades del vendedor respecto de las mercaderías en tránsito.}

Mientras estén en tránsito las cosas muebles vendidas y remitidas al Deudor, el vendedor no pagado podrá dejar sin efecto la tradición, recuperar la posesión y pedir la resolución de la compraventa.

El vendedor podrá también retener las cosas vendidas hasta el entero pago de su crédito.

\hypertarget{artuxedculo-157.--mercaderuxedas-en-truxe1nsito-vendidas-a-un-tercero.}{%
\paragraph*{Artículo 157.- Mercaderías en tránsito vendidas a un tercero.}\label{artuxedculo-157.--mercaderuxedas-en-truxe1nsito-vendidas-a-un-tercero.}}
\addcontentsline{toc}{paragraph}{Artículo 157.- Mercaderías en tránsito vendidas a un tercero.}

En caso que las cosas a que se refiere el Artículo anterior hayan sido vendidas durante su tránsito a un tercero de buena fe, a quien se hubiere transferido la factura, conocimiento o carta de porte, el vendedor no podrá ejercer las acciones que le confiere dicho Artículo.

Pero si el nuevo comprador no hubiere pagado el precio antes de la Resolución de Liquidación, el vendedor primitivo podrá demandar su entrega hasta la concurrencia de la cantidad que se le deba.

\hypertarget{artuxedculo-158.--efecto-de-la-resoluciuxf3n-de-la-compraventa.}{%
\paragraph*{Artículo 158.- Efecto de la resolución de la compraventa.}\label{artuxedculo-158.--efecto-de-la-resoluciuxf3n-de-la-compraventa.}}
\addcontentsline{toc}{paragraph}{Artículo 158.- Efecto de la resolución de la compraventa.}

En caso de resolución de la compraventa, el vendedor estará obligado a reembolsar a la masa los abonos a cuenta que hubiere recibido.

\hypertarget{artuxedculo-159.--comisiuxf3n-por-cuenta-propia.}{%
\paragraph*{Artículo 159.- Comisión por cuenta propia.}\label{artuxedculo-159.--comisiuxf3n-por-cuenta-propia.}}
\addcontentsline{toc}{paragraph}{Artículo 159.- Comisión por cuenta propia.}

El comisionista que ha pagado o se ha obligado a pagar con sus propios fondos las mercaderías compradas y remitidas por orden y cuenta del Deudor, podrá ejercitar las mismas acciones concedidas al vendedor por e Artículo 156.

\hypertarget{artuxedculo-160.--procedencia-del-derecho-legal-de-retenciuxf3n.}{%
\paragraph*{Artículo 160.- Procedencia del derecho legal de retención.}\label{artuxedculo-160.--procedencia-del-derecho-legal-de-retenciuxf3n.}}
\addcontentsline{toc}{paragraph}{Artículo 160.- Procedencia del derecho legal de retención.}

Fuera de los casos expresamente señalados por las leyes, el derecho legal de retención tendrá lugar siempre que la persona que ha pagado o que se ha obligado a pagar por el Deudor tenga en su poder mercaderías o valores de crédito que pertenezcan a éste, con tal que la tenencia nazca de un hecho voluntario del Deudor, anterior al pago o a la obligación, y que esos bienes no hayan sido remitidos con un destino determinado.

\hypertarget{artuxedculo-161.--oposiciuxf3n-del-liquidador-a-la-resoluciuxf3n-o-retenciuxf3n.}{%
\paragraph*{Artículo 161.- Oposición del Liquidador a la resolución o retención.}\label{artuxedculo-161.--oposiciuxf3n-del-liquidador-a-la-resoluciuxf3n-o-retenciuxf3n.}}
\addcontentsline{toc}{paragraph}{Artículo 161.- Oposición del Liquidador a la resolución o retención.}

En los casos a que se refieren los Artículos precedentes, el Liquidador podrá oponerse a la resolución o retención y exigir la entrega de las cosas vendidas o retenidas, pagando la deuda, intereses, costas y perjuicios, o dando caución que asegure el pago.

\hypertarget{artuxedculo-162.--razuxf3n-social-del-deudor-sujeto-a-un-procedimiento-concursal-de-liquidaciuxf3n.}{%
\paragraph*{Artículo 162.- Razón social del Deudor sujeto a un Procedimiento Concursal de Liquidación.}\label{artuxedculo-162.--razuxf3n-social-del-deudor-sujeto-a-un-procedimiento-concursal-de-liquidaciuxf3n.}}
\addcontentsline{toc}{paragraph}{Artículo 162.- Razón social del Deudor sujeto a un Procedimiento Concursal de Liquidación.}

El nombre o razón social del Deudor sujeto a un Procedimiento Concursal de Liquidación será complementado con la frase final ``en Procedimiento Concursal de Liquidación'', y su uso deberá ser precedido por la firma del Liquidador y demás habilitados. En caso contrario, serán solidariamente responsables tanto el Liquidador como los que hubieren ejecutado el acto o celebrado el contrato respectivo.

\hypertarget{puxe1rrafo-5.-de-la-incautaciuxf3n-e-inventario-de-bienes}{%
\subsubsection*{Párrafo 5. De la incautación e inventario de bienes}\label{puxe1rrafo-5.-de-la-incautaciuxf3n-e-inventario-de-bienes}}
\addcontentsline{toc}{subsubsection}{Párrafo 5. De la incautación e inventario de bienes}

\hypertarget{artuxedculo-163.--procedimiento.}{%
\paragraph*{Artículo 163.- Procedimiento.}\label{artuxedculo-163.--procedimiento.}}
\addcontentsline{toc}{paragraph}{Artículo 163.- Procedimiento.}

Una vez que haya asumido oficialmente el cargo y en presencia del secretario u otro ministro de fe designado por el tribunal competente, el Liquidador deberá:

\begin{enumerate}
\def\labelenumi{\arabic{enumi})}
\item
  Adoptar de inmediato las medidas conservativas necesarias para proteger y custodiar los bienes del Deudor, si estima que peligran o corren riesgos donde se encuentran.
\item
  Practicar la diligencia de incautación y confección del inventario de los bienes del Deudor.
\end{enumerate}

\hypertarget{artuxedculo-164.--del-acta-de-incautaciuxf3n.}{%
\paragraph*{Artículo 164.- Del acta de incautación.}\label{artuxedculo-164.--del-acta-de-incautaciuxf3n.}}
\addcontentsline{toc}{paragraph}{Artículo 164.- Del acta de incautación.}

De las diligencias de incautación se levantará un acta que deberá incluir, al menos, las siguientes menciones:

\begin{enumerate}
\def\labelenumi{\arabic{enumi})}
\item
  La singularización de cada uno de los domicilios, sucursales o sedes del Deudor en que se hubieren practicado.
\item
  El día, la hora y el nombre de los asistentes a las diligencias practicadas.
\item
  La circunstancia de haber sido necesario o no el auxilio de la fuerza pública.
\item
  La constancia de todo derecho o pretensión formulados por terceros en relación con los bienes del Deudor.
\item
  El inventario de bienes señalado en el Artículo 165.
\item
  El nombre y la firma del Liquidador y del ministro de fe que estuvo presente en la incautación e inventario de bienes.
\end{enumerate}

Si aparecieren nuevos bienes por inventariar, se aplicará en lo pertinente lo dispuesto en este Artículo.

\hypertarget{artuxedculo-165.--del-inventario.}{%
\paragraph*{Artículo 165.- Del inventario.}\label{artuxedculo-165.--del-inventario.}}
\addcontentsline{toc}{paragraph}{Artículo 165.- Del inventario.}

El inventario de los bienes del Deudor que el Liquidador confeccione deberá incluir, al menos, las siguientes menciones:

\begin{enumerate}
\def\labelenumi{\arabic{enumi})}
\item
  Un registro e indicación de los libros, correspondencia y documentos del Deudor, si los hubiere.
\item
  La individualización de los bienes del Deudor, dejando especial constancia acerca del estado de conservación de las maquinarias, útiles y equipos.
\item
  La identificación de los bienes respecto de los cuales el Liquidador constate la existencia de contratos de arrendamiento con opción de compra, y de todos aquellos que se encuentren en poder del Deudor en una calidad distinta a la de dueño.
\end{enumerate}

\hypertarget{artuxedculo-166.--publicidad-del-acta-de-incautaciuxf3n-e-inventario.}{%
\paragraph*{Artículo 166.- Publicidad del acta de incautación e inventario.}\label{artuxedculo-166.--publicidad-del-acta-de-incautaciuxf3n-e-inventario.}}
\addcontentsline{toc}{paragraph}{Artículo 166.- Publicidad del acta de incautación e inventario.}

El Liquidador deberá agregar el acta de incautación e inventario al expediente y publicarla en el Boletín Concursal a más tardar al quinto día contado desde la última diligencia practicada. Igual regla se aplicará a las incautaciones posteriores y a las que excluyan bienes del inventario.

\hypertarget{artuxedculo-167.--asesoruxeda-tuxe9cnica-al-liquidador.}{%
\paragraph*{Artículo 167.- Asesoría técnica al Liquidador.}\label{artuxedculo-167.--asesoruxeda-tuxe9cnica-al-liquidador.}}
\addcontentsline{toc}{paragraph}{Artículo 167.- Asesoría técnica al Liquidador.}

El Liquidador podrá practicar la diligencia de incautación y confección de inventario asesorado por un especialista en el giro del Deudor, cuyos honorarios serán considerados gastos de administración del Procedimiento Concursal de Liquidación. Asimismo, el Liquidador deberá dejar constancia en el acta de la idoneidad técnica del asesor, reseñándose los antecedentes que sirvan para acreditarla.

En todo caso, corresponderá a la Junta de Acreedores inmediatamente posterior aprobar o rechazar en definitiva dicho gasto.

\hypertarget{artuxedculo-168.--asesoruxeda-general-al-liquidador.}{%
\paragraph*{Artículo 168.- Asesoría general al Liquidador.}\label{artuxedculo-168.--asesoruxeda-general-al-liquidador.}}
\addcontentsline{toc}{paragraph}{Artículo 168.- Asesoría general al Liquidador.}

En las diligencias de incautación e inventario también podrán acompañar al Liquidador sus dependientes o asesores de confianza, cuyos honorarios serán exclusivamente de cargo del Liquidador.

\hypertarget{artuxedculo-169.--deber-de-colaboraciuxf3n-del-deudor.}{%
\paragraph*{Artículo 169.- Deber de colaboración del Deudor.}\label{artuxedculo-169.--deber-de-colaboraciuxf3n-del-deudor.}}
\addcontentsline{toc}{paragraph}{Artículo 169.- Deber de colaboración del Deudor.}

El Deudor deberá indicar y poner a disposición del Liquidador todos sus bienes y antecedentes. En caso que el Deudor se negare o no pudiere dar cumplimiento a lo anterior, el deber recaerá en cualquiera de sus administradores, si los hubiera.

Sin perjuicio de lo anterior, el Liquidador podrá solicitar el auxilio de la fuerza pública en caso de oposición del Deudor o de sus administradores, para lo cual bastará la exhibición de copia autorizada de la Resolución de Liquidación al jefe de turno de la respectiva unidad de Carabineros de Chile.

\hypertarget{puxe1rrafo-6.-de-la-determinaciuxf3n-del-pasivo}{%
\subsubsection*{Párrafo 6. De la determinación del pasivo}\label{puxe1rrafo-6.-de-la-determinaciuxf3n-del-pasivo}}
\addcontentsline{toc}{subsubsection}{Párrafo 6. De la determinación del pasivo}

\hypertarget{artuxedculo-170.--verificaciuxf3n-ordinaria-de-cruxe9ditos.}{%
\paragraph*{Artículo 170.- Verificación ordinaria de créditos.}\label{artuxedculo-170.--verificaciuxf3n-ordinaria-de-cruxe9ditos.}}
\addcontentsline{toc}{paragraph}{Artículo 170.- Verificación ordinaria de créditos.}

Los acreedores tendrán un plazo de treinta días contado desde la notificación de la Resolución de Liquidación para verificar sus créditos y alegar su preferencia ante el tribunal que conoce del procedimiento, acompañando los Títulos justificativos del crédito e indicando una dirección válida de correo electrónico para recibir las notificaciones que fueren pertinentes.

Vencido el plazo señalado en el inciso anterior, dentro de los dos días siguientes, el Liquidador publicará en el Boletín Concursal todas las verificaciones presentadas.

\hypertarget{artuxedculo-171.--acreedores-prestadores-de-servicios-de-utilidad-puxfablica.}{%
\paragraph*{Artículo 171.- Acreedores prestadores de Servicios de Utilidad Pública.}\label{artuxedculo-171.--acreedores-prestadores-de-servicios-de-utilidad-puxfablica.}}
\addcontentsline{toc}{paragraph}{Artículo 171.- Acreedores prestadores de Servicios de Utilidad Pública.}

Lo preceptuado en el Artículo precedente también será aplicable a los acreedores que presten Servicios de Utilidad Pública, quienes deberán verificar los créditos correspondientes a suministros anteriores a la Resolución de Liquidación y no podrán, con posterioridad a ella, suspender tales servicios, salvo autorización del tribunal, previa audiencia del Liquidador.

Los créditos correspondientes a Servicios de Utilidad Pública que se suministren con posterioridad a la notificación de la Resolución de Liquidación se considerarán incluidos en el número 4 del Artículo 2472 del Código Civil. La suspensión del servicio en contravención a lo dispuesto en el inciso primero de este Artículo, se sancionará sumariamente por el tribunal con multa de 1 a 200 unidades tributarias mensuales, debiendo restablecerse su suministro tan pronto el tribunal lo ordene.

Si a la fecha de la dictación de la Resolución de Liquidación los suministros se encontraren suspendidos, el Liquidador podrá solicitar al tribunal que ordene su inmediata reposición, solicitud que se deberá resolver a más tardar al día siguiente, sin necesidad de oír al prestador del servicio.

Los créditos que nazcan como resultado del ejercicio de esta facultad, se considerarán incluidos en el número 4 del Artículo 2472 del Código Civil. El costo de reposición será de cargo del respectivo prestador del servicio.

\hypertarget{artuxedculo-172.--tuxe9rmino-del-peruxedodo-de-verificaciuxf3n-ordinaria-de-cruxe9ditos.}{%
\paragraph*{Artículo 172.- Término del período de verificación ordinaria de créditos.}\label{artuxedculo-172.--tuxe9rmino-del-peruxedodo-de-verificaciuxf3n-ordinaria-de-cruxe9ditos.}}
\addcontentsline{toc}{paragraph}{Artículo 172.- Término del período de verificación ordinaria de créditos.}

Vencido el plazo de treinta días indicado en el Artículo 170 se entenderá de pleno derecho cerrado el período ordinario de verificación de créditos, sin necesidad de resolución ni notificación alguna. Sin perjuicio de lo anterior, dentro de los dos días siguientes de vencido el plazo señalado, el Liquidador publicará este cierre en el Boletín Concursal, junto con el listado de todos los créditos verificados con sus montos y preferencias alegadas.

\hypertarget{artuxedculo-173.--estudio-de-cruxe9ditos-y-preferencias.}{%
\paragraph*{Artículo 173.- Estudio de créditos y preferencias.}\label{artuxedculo-173.--estudio-de-cruxe9ditos-y-preferencias.}}
\addcontentsline{toc}{paragraph}{Artículo 173.- Estudio de créditos y preferencias.}

En cumplimiento de sus deberes legales, el Liquidador examinará todos los créditos que se verifiquen y las preferencias que se aleguen, investigando su origen, cuantía y legitimidad por todos los medios a su alcance, especialmente aquellos verificados por las Personas Relacionadas del Deudor. Si no encontrare justificado algún crédito o preferencia, deberá deducir la objeción que corresponda, de conformidad a las disposiciones del Artículo 174.

\hypertarget{artuxedculo-174.--objeciuxf3n-de-cruxe9ditos.}{%
\paragraph*{Artículo 174.- Objeción de créditos.}\label{artuxedculo-174.--objeciuxf3n-de-cruxe9ditos.}}
\addcontentsline{toc}{paragraph}{Artículo 174.- Objeción de créditos.}

Los acreedores, el Liquidador y el Deudor tendrán un plazo de diez días contado desde el vencimiento del período ordinario de verificación para deducir objeción fundada sobre la existencia, montos o preferencias de los créditos que se hayan presentado a verificación.
Las objeciones señaladas anteriormente se presentarán ante el tribunal que conoce del procedimiento. Expirado el plazo de diez días que se indica en el inciso anterior sin que se formulen objeciones, los créditos no objetados quedarán reconocidos. Asimismo, vencido dicho plazo, y dentro de los tres días siguientes, el Liquidador publicará en el Boletín Concursal todas las objeciones presentadas, confeccionará la nómina de los créditos reconocidos, la acompañará al expediente y la publicará en el Boletín Concursal.

\hypertarget{artuxedculo-175.--impugnaciuxf3n-de-cruxe9ditos.}{%
\paragraph*{Artículo 175.- Impugnación de créditos.}\label{artuxedculo-175.--impugnaciuxf3n-de-cruxe9ditos.}}
\addcontentsline{toc}{paragraph}{Artículo 175.- Impugnación de créditos.}

Si se formulan objeciones, el Liquidador arbitrará las medidas necesarias para que se obtenga el debido ajuste entre los acreedores o entre éstos y el Deudor, y se subsanen las objeciones. Si no se subsanan las objeciones deducidas, los créditos objeto de dichas objeciones se considerarán impugnados y el Liquidador los acumulará y emitirá un informe acerca de si existen o no fundamentos plausibles para ser considerados por el tribunal.

El Liquidador acompañará la nómina de créditos impugnados conjuntamente con su informe al tribunal y la publicará en el Boletín Concursal, dentro de los diez días siguientes a la expiración del plazo previsto para objetar señalado en el inciso primero del Artículo anterior.

Agregada al expediente la nómina de créditos impugnados con el informe del Liquidador, el tribunal citará a una audiencia única y verbal para el fallo de las respectivas impugnaciones, dentro de décimo día contado desde la notificación de la resolución que tiene por acompañada la nómina de créditos impugnados. A dicha audiencia podrán concurrir los impugnantes, el Deudor, el Liquidador y los acreedores impugnados en su caso. El tribunal competente podrá, por una sola vez, suspender y continuar la referida audiencia con posterioridad.

La resolución que falle las impugnaciones ordenará la incorporación o modificación de los créditos en la nómina de créditos reconocidos, cuando corresponda. La referida nómina de créditos reconocidos modificada deberá publicarse en el Boletín Concursal dentro los dos días siguientes a la fecha en que se dicte la resolución señalada.

\hypertarget{artuxedculo-176.--de-las-costas.}{%
\paragraph*{Artículo 176.- De las costas.}\label{artuxedculo-176.--de-las-costas.}}
\addcontentsline{toc}{paragraph}{Artículo 176.- De las costas.}

El impugnante vencido será condenado en costas a beneficio del acreedor impugnado, a menos que el tribunal considere que ha tenido motivos plausibles para litigar. Las costas que se determinen serán equivalentes al diez por ciento del crédito impugnado y no podrán exceder de 500 unidades de fomento.

Lo anterior no será procedente en caso que el impugnante sea el Liquidador.

\hypertarget{artuxedculo-177.--de-la-apelaciuxf3n.}{%
\paragraph*{Artículo 177.- De la apelación.}\label{artuxedculo-177.--de-la-apelaciuxf3n.}}
\addcontentsline{toc}{paragraph}{Artículo 177.- De la apelación.}

La resolución que se pronuncie sobre las impugnaciones será apelable en el sólo efecto devolutivo, gozando de preferencia para su inclusión a la tabla y para su vista y fallo.

\hypertarget{artuxedculo-178.--deber-del-liquidador-en-los-procesos-de-verificaciuxf3n-e-impugnaciuxf3n.}{%
\paragraph*{Artículo 178.- Deber del Liquidador en los procesos de verificación e impugnación.}\label{artuxedculo-178.--deber-del-liquidador-en-los-procesos-de-verificaciuxf3n-e-impugnaciuxf3n.}}
\addcontentsline{toc}{paragraph}{Artículo 178.- Deber del Liquidador en los procesos de verificación e impugnación.}

El Liquidador deberá perseguir judicialmente el pago de las costas y multas a beneficio de la masa, pudiendo, al efecto, descontarlas administrativamente de cualquier reparto que deba practicar al acreedor obligado a su pago.

\hypertarget{artuxedculo-179.--de-la-verificaciuxf3n-extraordinaria-de-cruxe9ditos.}{%
\paragraph*{Artículo 179.- De la verificación extraordinaria de créditos.}\label{artuxedculo-179.--de-la-verificaciuxf3n-extraordinaria-de-cruxe9ditos.}}
\addcontentsline{toc}{paragraph}{Artículo 179.- De la verificación extraordinaria de créditos.}

Los acreedores que no hayan verificado sus créditos en el período ordinario, podrán hacerlo mientras no esté firme y ejecutoriada la Cuenta Final de Administración del Liquidador, para ser considerados sólo en los repartos futuros, y deberán aceptar todo lo obrado con anterioridad.

Los créditos verificados extraordinariamente podrán ser objetados o impugnados en conformidad al procedimiento establecido en los Artículos 174 y 175, dentro del plazo de diez días contado desde la notificación de su verificación en el Boletín Concursal.

\hypertarget{puxe1rrafo-7.-de-las-juntas-de-acreedores-en-los-procedimientos-concursales-de-liquidaciuxf3n}{%
\subsubsection*{Párrafo 7. De las Juntas de Acreedores en los Procedimientos Concursales de Liquidación}\label{puxe1rrafo-7.-de-las-juntas-de-acreedores-en-los-procedimientos-concursales-de-liquidaciuxf3n}}
\addcontentsline{toc}{subsubsection}{Párrafo 7. De las Juntas de Acreedores en los Procedimientos Concursales de Liquidación}

\hypertarget{artuxedculo-180.--de-las-juntas-de-acreedores.}{%
\paragraph*{Artículo 180.- De las Juntas de Acreedores.}\label{artuxedculo-180.--de-las-juntas-de-acreedores.}}
\addcontentsline{toc}{paragraph}{Artículo 180.- De las Juntas de Acreedores.}

Los acreedores adoptarán los acuerdos en Juntas de Acreedores celebradas de conformidad a las disposiciones del presente Párrafo, las que se denominarán, según corresponda, Junta Constitutiva, Juntas Ordinarias y Juntas Extraordinarias.

\hypertarget{artuxedculo-181.--del-quuxf3rum-para-sesionar.}{%
\paragraph*{Artículo 181.- Del quórum para sesionar.}\label{artuxedculo-181.--del-quuxf3rum-para-sesionar.}}
\addcontentsline{toc}{paragraph}{Artículo 181.- Del quórum para sesionar.}

Toda Junta de Acreedores se entenderá constituida legalmente para sesionar si cuenta con la concurrencia de uno o más acreedores que representen al menos el 25\% del pasivo con derecho a voto, a menos que esta ley señale expresamente un quórum de constitución distinto. Los acuerdos se adoptarán con Quórum Simple, salvo que esta ley establezca un quórum diferente.

\hypertarget{artuxedculo-182.--asistencia-y-derecho-a-voz.}{%
\paragraph*{Artículo 182.- Asistencia y derecho a voz.}\label{artuxedculo-182.--asistencia-y-derecho-a-voz.}}
\addcontentsline{toc}{paragraph}{Artículo 182.- Asistencia y derecho a voz.}

Las Juntas de Acreedores serán públicas y el Liquidador podrá disponer que, por razones de seguridad y previa autorización judicial, se celebren sesiones con presencia limitada de público general.

Tendrán derecho a voz:

\begin{enumerate}
\def\labelenumi{\arabic{enumi})}
\item
  Todos los acreedores que hayan verificado sus créditos, tengan o no derecho a voto.
\item
  El Liquidador.
\item
  El Deudor.
\item
  El Superintendente de Insolvencia y Emprendimiento, o quien éste designe.
\end{enumerate}

\hypertarget{artuxedculo-183.--nuxf3mina-de-asistencia.}{%
\paragraph*{Artículo 183.- Nómina de asistencia.}\label{artuxedculo-183.--nuxf3mina-de-asistencia.}}
\addcontentsline{toc}{paragraph}{Artículo 183.- Nómina de asistencia.}

Los acreedores que asistan a las Juntas de Acreedores que se celebren con arreglo a este \#\#\#\# Párrafo deberán suscribir la correspondiente nómina de asistencia que al efecto proporcione el Liquidador, indicando su nombre completo o razón social y la individualización del apoderado que asiste en su representación, en su caso. Igual deber pesará sobre el Deudor.

\hypertarget{artuxedculo-184.--del-acta-y-su-publicaciuxf3n.}{%
\paragraph*{Artículo 184.- Del acta y su publicación.}\label{artuxedculo-184.--del-acta-y-su-publicaciuxf3n.}}
\addcontentsline{toc}{paragraph}{Artículo 184.- Del acta y su publicación.}

De todo lo obrado en la Junta de Acreedores, incluyendo acuerdos adoptados y propuestas desestimadas, se levantará un acta, la que deberá ser suscrita por el Liquidador, el Deudor si lo estimare y los acreedores que para ello se designen en la misma Junta de Acreedores. Dicha acta será publicada al día siguiente por el Liquidador en el Boletín Concursal.

\hypertarget{artuxedculo-185.--certificado-de-no-celebraciuxf3n-de-la-junta-de-acreedores.}{%
\paragraph*{Artículo 185.- Certificado de no celebración de la Junta de Acreedores.}\label{artuxedculo-185.--certificado-de-no-celebraciuxf3n-de-la-junta-de-acreedores.}}
\addcontentsline{toc}{paragraph}{Artículo 185.- Certificado de no celebración de la Junta de Acreedores.}

En caso que no se celebrare una Junta de Acreedores por falta de quórum, el Liquidador certificará dicha circunstancia y deberá publicar el correspondiente certificado en el Boletín Concursal al día siguiente de aquel en que la Junta debió celebrarse.

\hypertarget{artuxedculo-186.--suspensiuxf3n-y-reanudaciuxf3n-de-juntas-de-acreedores.}{%
\paragraph*{Artículo 186.- Suspensión y reanudación de Juntas de Acreedores.}\label{artuxedculo-186.--suspensiuxf3n-y-reanudaciuxf3n-de-juntas-de-acreedores.}}
\addcontentsline{toc}{paragraph}{Artículo 186.- Suspensión y reanudación de Juntas de Acreedores.}

En caso que durante cualquier Junta de Acreedores no se adoptasen uno o más acuerdos en razón de las abstenciones de los acreedores presentes con derecho a voto, el Liquidador podrá, a su sólo arbitrio, suspender la Junta de Acreedores una vez tratadas y votadas las respectivas materias, a efectos de lograr los quórum legales para adoptar tales propuestas.
La Junta suspendida se reanudará al segundo día en el mismo lugar y hora, pudiendo en todo caso fijarse otro distinto por Quórum Simple. En caso que el Liquidador haga uso de esta facultad se observarán las reglas que siguen:

\begin{enumerate}
\def\labelenumi{\arabic{enumi})}
\item
  Los acreedores se entenderán legalmente notificados de la fecha, hora, lugar y materias de la Junta que se reanudará, por el sólo ejercicio de la facultad prevista en este Artículo.
\item
  Se levantará acta de lo obrado hasta el momento de la suspensión, según lo previsto en el Artículo 184, dejándose constancia del ejercicio de la facultad de suspensión por parte del Liquidador, así como del porcentaje de votación favorable que hubieren alcanzado el o los acuerdos no adoptados en razón de las abstenciones de los acreedores presentes.
\item
  Los acuerdos que se hubieren adoptado antes de la suspensión no podrán ser modificados o alterados en la Junta de Acreedores reanudada y deberán ejecutarse conforme a las reglas generales, salvo que los mismos acreedores y por las mismas acreencias que concurrieron con su voto consientan en modificarlo o dejarlo sin efecto.
\item
  En la Junta de Acreedores reanudada se presumirá de derecho la mantención del quórum de asistencia existente al momento de la suspensión.
\item
  Si los acreedores que se abstuvieron de votar un determinado acuerdo antes de la suspensión de la Junta de Acreedores no asistieren a la reanudación de la misma o si, asistiendo, se abstuvieren nuevamente de votar, se adicionará de pleno derecho su voto a la mayoría obtenida para ese acuerdo, consignada en el acta a que se refiere el número 2) precedente.
\item
  Se levantará una nueva acta de lo tratado en la Junta de Acreedores reanudada, la que deberá ser suscrita por el Liquidador y los acreedores asistentes, y se estará a lo dispuesto en el Artículo 184.
\end{enumerate}

\hypertarget{artuxedculo-187.--mandato-para-asistir-a-juntas-de-acreedores.}{%
\paragraph*{Artículo 187.- Mandato para asistir a Juntas de Acreedores.}\label{artuxedculo-187.--mandato-para-asistir-a-juntas-de-acreedores.}}
\addcontentsline{toc}{paragraph}{Artículo 187.- Mandato para asistir a Juntas de Acreedores.}

La asistencia de los acreedores y del Deudor a las Juntas de Acreedores que se celebren podrá ser personal o a través de mandatario. A las Juntas de Acreedores que se celebren ante el tribunal, los acreedores deberán comparecer debidamente representados conforme a lo dispuesto en los Artículos 6° y 7° del Código de Procedimiento Civil.

El mandato deberá constar en instrumento público o privado y, en este último caso, la firma del mandante deberá ser autorizada por el secretario del tribunal competente o por un ministro de fe.

Se entenderá que el mandatario tiene idénticas facultades que las de su mandante y se tendrá por no escrita cualquier limitación que hubiere podido establecerse en el mandato. El mandatario podrá votar todos los acuerdos que sean presentados en cada una de las Juntas de Acreedores que se celebren.

Se prohíbe otorgar mandato para asistir a Juntas de Acreedores a más de una persona, salvo para el caso de su reemplazo, pero un mismo mandatario puede serlo de uno o más acreedores.

\hypertarget{artuxedculo-188.--prohibiciuxf3n-de-fraccionar-los-cruxe9ditos.}{%
\paragraph*{Artículo 188.- Prohibición de fraccionar los créditos.}\label{artuxedculo-188.--prohibiciuxf3n-de-fraccionar-los-cruxe9ditos.}}
\addcontentsline{toc}{paragraph}{Artículo 188.- Prohibición de fraccionar los créditos.}

Se prohíbe fraccionar los créditos después de dictada la Resolución de Liquidación y conferir mandato por una parte o fracción de un crédito. El contraventor y los que representen las porciones del crédito perderán el derecho a asistir a las Juntas de Acreedores. Todos los que hagan valer porciones de un crédito fraccionado dentro de los treinta días anteriores al pronunciamiento de la Resolución de Liquidación se contarán como una sola persona y emitirán un solo voto, procediéndose en la forma establecida en el inciso final de este Artículo.

Las disposiciones precedentes no serán aplicables al crédito dividido como consecuencia de la liquidación de una sociedad, o de la partición de una comunidad que no esté exclusivamente formada por dicho crédito.

El crédito perteneciente a una comunidad será representado sólo por uno de los comuneros. Si no se acuerda la designación del representante, cualquiera de ellos podrá solicitar tal designación al tribunal.

\hypertarget{artuxedculo-189.--del-derecho-a-voto.}{%
\paragraph*{Artículo 189.- Del derecho a voto.}\label{artuxedculo-189.--del-derecho-a-voto.}}
\addcontentsline{toc}{paragraph}{Artículo 189.- Del derecho a voto.}

Tendrán derecho a voto aquellos acreedores cuyos créditos estén reconocidos y aquellos a los que se les haya concedido el derecho a votar de conformidad al procedimiento dispuesto en el Artículo siguiente, aunque sus créditos no estén reconocidos, hayan sido o no objetados o impugnados.

\hypertarget{artuxedculo-190.--audiencia-de-determinaciuxf3n-del-derecho-a-voto.}{%
\paragraph*{Artículo 190.- Audiencia de determinación del derecho a voto.}\label{artuxedculo-190.--audiencia-de-determinaciuxf3n-del-derecho-a-voto.}}
\addcontentsline{toc}{paragraph}{Artículo 190.- Audiencia de determinación del derecho a voto.}

Corresponderá al tribunal determinar el derecho a voto respecto de los acreedores indicados en el Artículo anterior cuyos créditos no estén reconocidos, debiendo sujetar su decisión a las reglas siguientes:

\begin{enumerate}
\def\labelenumi{\arabic{enumi})}
\item
  Deberá celebrarse una audiencia el día inmediatamente anterior a la Junta de Acreedores, ante el tribunal y en presencia del secretario, a la que asistirán el Liquidador, el Deudor y los acreedores, estos dos últimos, si lo estiman pertinente.
\item
  La audiencia se celebrará a las 15:00 horas, horario que podrá ser modificado por el tribunal, de oficio o a petición de parte.
\item
  La audiencia comenzará con la entrega de un informe escrito del Liquidador al tribunal acerca de la verosimilitud de la existencia y monto reclamado de los créditos no reconocidos. El informe se deberá referir especialmente a aquellos créditos que estén en alguna de las circunstancias previstas en el Artículo 188. El Informe incluirá todos los créditos no reconocidos que se hubieren verificado hasta el día inmediatamente anterior a dicha audiencia.
\end{enumerate}

Del contenido del referido informe, el Liquidador será responsable de acuerdo a lo señalado en el Artículo 35.

\begin{enumerate}
\def\labelenumi{\arabic{enumi})}
\setcounter{enumi}{3}
\item
  A continuación, el tribunal oirá a aquellos acreedores que soliciten verbalmente argumentar la inclusión o conservación de su propio crédito en el informe o bien la exclusión de otros. No se admitirán presentaciones escritas para sustentar dichos argumentos.
\item
  Acto seguido, el tribunal resolverá en única instancia, con los antecedentes disponibles en dicha audiencia, los que apreciará de acuerdo a las normas de la sana crítica, dejando constancia en el acta respectiva. Contra la resolución del tribunal sólo procederá el recurso de reposición, que deberá ser interpuesto y resuelto en la misma audiencia.
\item
  El acta indicará los acreedores y el monto concreto que gozará de derecho a voto en la Junta a celebrar.
\item
  El reconocimiento de derecho a voto sólo producirá efectos para la Junta de Acreedores en referencia y en nada limitará la libertad del Liquidador y de los acreedores para objetar o impugnar el crédito y sus preferencias de acuerdo a esta ley, ni la del tribunal para resolver la impugnación.
\item
  El Liquidador deberá asistir personalmente a las audiencias de determinación del derecho a voto previas a la Junta Constitutiva y a la primera Junta Ordinaria de Acreedores, pudiendo asistir su apoderado judicial a las restantes.
\end{enumerate}

\hypertarget{artuxedculo-191.--excepciuxf3n-y-limitaciuxf3n-al-ejercicio-del-derecho-a-voto.}{%
\paragraph*{Artículo 191.- Excepción y limitación al ejercicio del derecho a voto.}\label{artuxedculo-191.--excepciuxf3n-y-limitaciuxf3n-al-ejercicio-del-derecho-a-voto.}}
\addcontentsline{toc}{paragraph}{Artículo 191.- Excepción y limitación al ejercicio del derecho a voto.}

Las Personas Relacionadas con el Deudor no gozarán de derecho a voto, ni tampoco se considerarán en el cálculo del respectivo quórum.

El acreedor o su mandatario que tengan un conflicto de interés o un interés distinto del inherente a la calidad de acreedor del Deudor respecto de un determinado acuerdo deberán abstenerse de votar dicho acuerdo y tampoco se considerarán en el cálculo del respectivo quórum.

\hypertarget{artuxedculo-192.--participaciuxf3n-de-cruxe9ditos-pagados.}{%
\paragraph*{Artículo 192.- Participación de créditos pagados.}\label{artuxedculo-192.--participaciuxf3n-de-cruxe9ditos-pagados.}}
\addcontentsline{toc}{paragraph}{Artículo 192.- Participación de créditos pagados.}

Los acreedores no tendrán derecho a voto por los créditos que hubieren sido totalmente pagados a causa de un reparto, de un pago administrativo o por cualquier otra forma, incluso por un tercero. Si el pago del crédito hubiere sido parcial, el acreedor tendrá derecho a voto sólo por el saldo insoluto.

\hypertarget{artuxedculo-193.--de-la-junta-constitutiva.}{%
\paragraph*{Artículo 193.- De la Junta Constitutiva.}\label{artuxedculo-193.--de-la-junta-constitutiva.}}
\addcontentsline{toc}{paragraph}{Artículo 193.- De la Junta Constitutiva.}

Es la primera Junta de Acreedores que se celebra una vez iniciado el Procedimiento Concursal de Liquidación. Tendrá lugar al trigésimo segundo día contado desde la publicación en el Boletín Concursal de la Resolución de Liquidación y se realizará en las dependencias del tribunal o en el lugar específico que éste designe, a la hora que la misma resolución fije.

\hypertarget{artuxedculo-194.--segunda-citaciuxf3n-a-la-junta-constitutiva.}{%
\paragraph*{Artículo 194.- Segunda citación a la Junta Constitutiva.}\label{artuxedculo-194.--segunda-citaciuxf3n-a-la-junta-constitutiva.}}
\addcontentsline{toc}{paragraph}{Artículo 194.- Segunda citación a la Junta Constitutiva.}

En caso de no celebrarse la Junta Constitutiva por falta del quórum necesario para sesionar, ésta deberá efectuarse el segundo día, a la misma hora y en igual lugar. El secretario del tribunal deberá dejar constancia de esta situación en el acta que se levante y desde entonces los acreedores se entenderán legalmente notificados de esa segunda citación. La Junta así convocada se tendrá por constituida y se celebrará con los acreedores que asistan, adoptándose las decisiones con Quórum Simple de los presentes, sin perjuicio de las materias que exijan quórum distintos.

\hypertarget{artuxedculo-195.--inasistencia-de-acreedores-en-segunda-citaciuxf3n.}{%
\paragraph*{Artículo 195.- Inasistencia de acreedores en segunda citación.}\label{artuxedculo-195.--inasistencia-de-acreedores-en-segunda-citaciuxf3n.}}
\addcontentsline{toc}{paragraph}{Artículo 195.- Inasistencia de acreedores en segunda citación.}

Si en la segunda citación no asiste ningún acreedor con derecho a voto, el secretario del tribunal certificará esta circunstancia, produciéndose los siguientes efectos, sin necesidad de declaración judicial:

\begin{enumerate}
\def\labelenumi{\arabic{enumi})}
\item
  Los Liquidadores, titular y suplente provisionales, se entenderán ratificados de pleno derecho en sus cargos, asumiendo ambos la calidad de definitivos, sin perjuicio de la facultad prevista en el Artículo 200 de esta ley.
\item
  El Liquidador publicará en el Boletín Concursal, dentro de tercero día contado desde aquel en que la Junta de Acreedores en segunda citación debió celebrarse, lo siguiente:
\end{enumerate}

\begin{enumerate}
\def\labelenumi{\alph{enumi})}
\item
  Una referencia a la certificación practicada por el secretario del tribunal, indicada en el encabezamiento de este Artículo.
\item
  La cuenta sobre el estado preciso de los negocios del Deudor, de su activo y pasivo y de la labor por él realizada.
\item
  El lugar, día y hora en que se celebrarán las Juntas Ordinarias, que el mismo Liquidador fijará.
\end{enumerate}

\begin{enumerate}
\def\labelenumi{\arabic{enumi})}
\setcounter{enumi}{2}
\tightlist
\item
  El Liquidador dará inicio al procedimiento de liquidación simplificada o sumaria.
\end{enumerate}

\hypertarget{artuxedculo-196.--materias-de-la-junta-constitutiva.}{%
\paragraph*{Artículo 196.- Materias de la Junta Constitutiva.}\label{artuxedculo-196.--materias-de-la-junta-constitutiva.}}
\addcontentsline{toc}{paragraph}{Artículo 196.- Materias de la Junta Constitutiva.}

La Junta Constitutiva tratará las siguientes materias:

\begin{enumerate}
\def\labelenumi{\arabic{enumi})}
\item
  El Liquidador titular provisional deberá presentar una cuenta escrita, la que además expondrá verbal y circunstanciadamente, acerca del estado preciso de los negocios del Deudor, de su activo y pasivo, y de la gestión realizada, incluyendo un desglose de los gastos incurridos a la fecha. Asimismo, deberá informar si los activos del Deudor se encuentran en la situación prevista en la letra b) del Artículo 203.
\item
  La ratificación de los Liquidadores titular y suplente provisionales, o bien, la designación de sus reemplazantes. Los Liquidadores que no hubieren sido ratificados continuarán en sus cargos hasta que asuman sus reemplazantes. Dentro de diez días contados desde la nueva designación deberá suscribirse entre el Liquidador no ratificado y el que lo remplace un acta de traspaso en que conste el estado preciso de los bienes del Deudor y cualquier otro aspecto relevante para una adecuada continuación del Procedimiento Concursal de Liquidación. En el mismo plazo deberán entregarse todos los antecedentes, documentos y otros instrumentos del Deudor que se encuentren en poder del Liquidador no ratificado. Una copia del acta antes indicada deberá ser remitida a la Superintendencia.
\item
  La determinación del día, hora y lugar en que sesionarán las Juntas Ordinarias. Éstas deberán tener lugar al menos semestralmente.
\item
  La designación de un presidente titular y uno suplente y un secretario titular y uno suplente, de entre los acreedores con derecho a voto o sus representantes, para las futuras sesiones.
\item
  Un plan o propuesta circunstanciada de la realización de los bienes del Deudor, la estimación de los principales gastos del Procedimiento Concursal de Liquidación y la continuación de las actividades económicas, de conformidad a lo previsto en el Título 4 de este Capítulo, en los casos que proceda.
\item
  Cualquier otro acuerdo que la Junta estime conducente, con excepción de aquellos que recaigan sobre materias propias de Juntas Extraordinarias.
\end{enumerate}

\hypertarget{artuxedculo-197.--formalidades-de-la-junta-constitutiva.}{%
\paragraph*{Artículo 197.- Formalidades de la Junta Constitutiva.}\label{artuxedculo-197.--formalidades-de-la-junta-constitutiva.}}
\addcontentsline{toc}{paragraph}{Artículo 197.- Formalidades de la Junta Constitutiva.}

La Junta Constitutiva será presidida por el juez que esté conociendo del Procedimiento Concursal de Liquidación y actuará como ministro de fe el secretario del tribunal. De los puntos tratados, los acuerdos adoptados y demás materias que el tribunal estime pertinentes deberá dejarse constancia en un acta que será firmada por el juez, el secretario, el Liquidador, los acreedores que lo soliciten y el Deudor, si así lo decide. Una copia autorizada de dicha acta será agregada al expediente, publicada en el Boletín Concursal dentro del tercer día siguiente de levantada, e incorporada al libro de actas que llevará el Liquidador.

\hypertarget{artuxedculo-198.--de-la-primera-junta-ordinaria.}{%
\paragraph*{Artículo 198.- De la Primera Junta Ordinaria.}\label{artuxedculo-198.--de-la-primera-junta-ordinaria.}}
\addcontentsline{toc}{paragraph}{Artículo 198.- De la Primera Junta Ordinaria.}

Son materias obligatorias a tratar en la Primera Junta Ordinaria, si éstas no se hubieren acordado en la Junta Constitutiva, las siguientes:

\begin{enumerate}
\def\labelenumi{\arabic{enumi})}
\item
  El informe acerca del activo y pasivo del Deudor, especialmente las variaciones que hubieren experimentado desde la Junta Constitutiva, que el Liquidador deberá presentar por escrito y explicar verbalmente;
\item
  El plan o propuesta circunstanciada de realización de los bienes del Deudor, y
\item
  La estimación de los principales gastos del Procedimiento Concursal de Liquidación.
\end{enumerate}

También podrá tratarse y acordarse, a proposición del Liquidador, del Deudor o de cualquier acreedor asistente con derecho a voto, la continuación de actividades económicas, de conformidad a lo previsto en el Título 4 de este Capítulo.

Asimismo, los acreedores podrán acordar, con Quórum Especial, la no celebración de Juntas Ordinarias por un período determinado, o bien, su celebración por citación expresa del Liquidador o de acreedores que representen a lo menos el 25\% del pasivo con derecho a voto. En estos casos, el Liquidador procederá de acuerdo al Artículo 48 y no será necesario otorgar el certificado a que se refiere el Artículo 185.

\hypertarget{artuxedculo-199.--procedencia-de-la-junta-extraordinaria.}{%
\paragraph*{Artículo 199.- Procedencia de la Junta Extraordinaria.}\label{artuxedculo-199.--procedencia-de-la-junta-extraordinaria.}}
\addcontentsline{toc}{paragraph}{Artículo 199.- Procedencia de la Junta Extraordinaria.}

La Junta Extraordinaria tendrá lugar en los casos siguientes:

\begin{enumerate}
\def\labelenumi{\alph{enumi})}
\item
  Cuando fuere ordenada por el tribunal;
\item
  A petición del Liquidador o de la Superintendencia;
\item
  Cuando un acreedor o acreedores que representen a lo menos el 25\% del pasivo con derecho a voto lo soliciten por escrito al Liquidador, quien ejecutará los actos necesarios para su celebración, y
\item
  Cuando así lo hubieren acordado los acreedores en Junta Ordinaria con Quórum Simple.
\end{enumerate}

\hypertarget{artuxedculo-200.--materias-de-juntas-extraordinarias.}{%
\paragraph*{Artículo 200.- Materias de Juntas Extraordinarias.}\label{artuxedculo-200.--materias-de-juntas-extraordinarias.}}
\addcontentsline{toc}{paragraph}{Artículo 200.- Materias de Juntas Extraordinarias.}

Son materias de Juntas Extraordinarias las solicitadas por el o los peticionarios señalados en el Artículo anterior. Además, serán materias exclusivas de Juntas Extraordinarias las siguientes:

\begin{enumerate}
\def\labelenumi{\arabic{enumi})}
\item
  La revocación de los Liquidadores titular y suplente definitivos.
\item
  La presentación de proposiciones de Acuerdos de Reorganización Judicial en los términos del Capítulo III y del Párrafo 5 del Título 5 del Capítulo IV de esta ley.
\item
  Los acuerdos sobre contrataciones especializadas previstas en el Artículo 41 de esta ley.
\item
  Los anticipos de honorarios que solicite el Liquidador durante el Procedimiento Concursal de Liquidación, de acuerdo a lo establecido en el Artículo 39 de esta ley.
\end{enumerate}

\hypertarget{artuxedculo-201.--formalidades-de-la-citaciuxf3n-a-junta-extraordinaria.}{%
\paragraph*{Artículo 201.- Formalidades de la citación a Junta Extraordinaria.}\label{artuxedculo-201.--formalidades-de-la-citaciuxf3n-a-junta-extraordinaria.}}
\addcontentsline{toc}{paragraph}{Artículo 201.- Formalidades de la citación a Junta Extraordinaria.}

El peticionario deberá requerir por escrito al Liquidador la citación a Junta Extraordinaria, acreditando el cumplimiento de los requisitos señalados en el Artículo 199. Si el peticionario es el juez o la Superintendencia, bastará cualquier medio idóneo de comunicación al Liquidador. En el requerimiento que se presente al Liquidador deberá precisarse las materias a tratar en la Junta Extraordinaria y en ésta sólo podrán discutirse y decidirse tales materias. En cuanto a la determinación de día, hora y lugar se seguirán las reglas siguientes:

\begin{enumerate}
\def\labelenumi{\arabic{enumi})}
\item
  Si el requirente es el tribunal o la Superintendencia, se estará a la fecha que éstos fijen, debiendo el Liquidador disponer los medios que permitan su celebración.
\item
  Si el requirente es uno o más acreedores que representen al menos el 25\% del pasivo con derecho a voto, se estará a la fecha que de común acuerdo fijen con el Liquidador. En caso de desacuerdo, se estará a lo señalado por el o los requirentes.
\item
  Si la decisión ha sido adoptada en Junta Ordinaria de Acreedores, el acuerdo deberá indicar la fecha de celebración de la Junta Extraordinaria, debiendo el Liquidador ajustarse a dicha decisión.
\end{enumerate}

El Liquidador deberá publicar la citación a la Junta Extraordinaria de Acreedores en el Boletín Concursal al día siguiente a la solicitud, adjuntando copia de la solicitud que se le haya presentado.

La Junta deberá celebrarse transcurridos a lo menos tres días desde la publicación de la citación por el Liquidador en el Boletín Concursal.

\hypertarget{artuxedculo-202.--comisiuxf3n-de-acreedores.}{%
\paragraph*{Artículo 202.- Comisión de acreedores.}\label{artuxedculo-202.--comisiuxf3n-de-acreedores.}}
\addcontentsline{toc}{paragraph}{Artículo 202.- Comisión de acreedores.}

La Junta de Acreedores podrá acordar, con Quórum Calificado, la constitución de una Comisión de Acreedores, para los efectos de adoptar los acuerdos que se comprendan dentro de la órbita de su competencia con validez general. Su composición, facultades, duración y procedimientos aplicables serán determinados por la propia Junta de Acreedores, con el mismo quórum anterior.

\hypertarget{tuxedtulo-2.-de-la-realizaciuxf3n-simplificada-o-sumaria}{%
\subsection*{Título 2. De la realización simplificada o sumaria}\label{tuxedtulo-2.-de-la-realizaciuxf3n-simplificada-o-sumaria}}
\addcontentsline{toc}{subsection}{Título 2. De la realización simplificada o sumaria}

\hypertarget{puxe1rrafo-1.-del-uxe1mbito-de-aplicaciuxf3n-de-la-realizaciuxf3n-simplificada-o-sumaria}{%
\subsubsection*{Párrafo 1. Del ámbito de aplicación de la realización simplificada o sumaria}\label{puxe1rrafo-1.-del-uxe1mbito-de-aplicaciuxf3n-de-la-realizaciuxf3n-simplificada-o-sumaria}}
\addcontentsline{toc}{subsubsection}{Párrafo 1. Del ámbito de aplicación de la realización simplificada o sumaria}

\hypertarget{artuxedculo-203.--uxe1mbito-de-aplicaciuxf3n.}{%
\paragraph*{Artículo 203.- Ámbito de aplicación.}\label{artuxedculo-203.--uxe1mbito-de-aplicaciuxf3n.}}
\addcontentsline{toc}{paragraph}{Artículo 203.- Ámbito de aplicación.}

La realización simplificada o sumaria prevista en este Título se aplicará en los siguientes casos:

\begin{enumerate}
\def\labelenumi{\alph{enumi})}
\item
  Si el Deudor califica como micro empresa, de conformidad a lo dispuesto en el Artículo segundo de la ley N° 20.416, circunstancia que será acreditada por el Liquidador, para lo cual podrá requerir al Servicio de Impuestos Internos la información relativa al nivel de ventas del Deudor.
\item
  Si el Liquidador informare a los acreedores en la Junta Constitutiva que el producto probable de la realización del activo a liquidar no excederá las 5.000 unidades de fomento. Si el Deudor o cualquier acreedor no estuviere de acuerdo con la estimación efectuada por el Liquidador, deberá formular verbalmente su oposición en la misma Junta Constitutiva. El tribunal, luego de escuchar a los interesados y al Liquidador, deberá resolver la controversia en la misma Junta. Contra la resolución que pronuncie no procederá recurso alguno.
\item
  Si la Junta Constitutiva no se celebrare en segunda citación por falta de quórum.
\item
  Si la Junta Constitutiva se celebrare en segunda citación con asistencia igual o inferior al 20\% del pasivo total con derecho a voto.
\item
  Si la Junta lo acuerda.
\item
  Si fuere procedente la aplicación del Artículo 210 de esta ley.
\end{enumerate}

\hypertarget{puxe1rrafo-2.-de-la-realizaciuxf3n-simplificada-o-sumaria-propiamente-tal}{%
\subsubsection*{Párrafo 2. De la realización simplificada o sumaria propiamente tal}\label{puxe1rrafo-2.-de-la-realizaciuxf3n-simplificada-o-sumaria-propiamente-tal}}
\addcontentsline{toc}{subsubsection}{Párrafo 2. De la realización simplificada o sumaria propiamente tal}

\hypertarget{artuxedculo-204.--reglas-de-realizaciuxf3n-de-los-bienes.}{%
\paragraph*{Artículo 204.- Reglas de realización de los bienes.}\label{artuxedculo-204.--reglas-de-realizaciuxf3n-de-los-bienes.}}
\addcontentsline{toc}{paragraph}{Artículo 204.- Reglas de realización de los bienes.}

Los valores mobiliarios con presencia bursátil se venderán en remate en bolsa. Los demás bienes muebles e inmuebles se liquidarán mediante venta al martillo, conforme a las siguientes reglas:

\begin{enumerate}
\def\labelenumi{\alph{enumi})}
\item
  El Liquidador designará a un Martillero Concursal.
\item
  Las bases y demás condiciones de venta serán confeccionadas por el Liquidador, presentadas al tribunal y publicadas en el Boletín Concursal. Los acreedores y el Deudor podrán, dentro de segundo día, objetar las bases. En tal caso, el tribunal citará a las partes a una única audiencia verbal, que se celebrará a más tardar al quinto día desde el vencimiento del plazo para objetar, con las partes que asistan. La citación a audiencia se notificará por el Estado Diario.
\end{enumerate}

El tribunal resolverá las objeciones deducidas en la audiencia y contra su resolución sólo podrá deducirse verbalmente reposición, la que deberá ser resuelta en la misma oportunidad.
El costo de la redacción de las bases será del Liquidador, con cargo al honorario único que perciba en conformidad a lo dispuesto en el Artículo 40 de esta ley.

\begin{enumerate}
\def\labelenumi{\alph{enumi})}
\setcounter{enumi}{2}
\item
  Una vez resueltas las objeciones, las bases y las demás condiciones se publicarán en el Boletín Concursal, con a lo menos cinco días de anticipación a la fecha del remate y sin perjuicio de las restantes formas de publicidad que prevean las mismas bases.
\item
  En el caso de bienes inmuebles, las bases deberán considerar el otorgamiento de una garantía de seriedad exigible a todo postor de, a lo menos, el 10\% del mínimo por cada bien raíz a rematar. Dicha garantía subsistirá hasta que se otorgue la escritura definitiva de compraventa y se inscriba el dominio del comprador en el conservador de bienes raíces respectivo, libre de todos los gravámenes cuya cancelación y/o alzamiento se hubiese comprometido en las bases.
\item
  El mínimo del remate de bienes inmuebles o de derechos sobre ellos corresponderá al fijado por la Junta Constitutiva de Acreedores o, en su defecto, al Avalúo Fiscal vigente al semestre en que ésta se efectúe, o a la proporción que corresponda según dicho avalúo, respectivamente. En caso que no se presentaren postores, se deberá efectuar un nuevo remate en un plazo máximo de veinte días, y el mínimo corresponderá al 50\% del fijado originalmente. Si tampoco se presentaren postores en este segundo llamado, se deberá efectuar un nuevo remate en un plazo máximo de veinte días, sin mínimo.
\item
  El mínimo del remate de bienes muebles corresponderá al fijado por la Junta Constitutiva de Acreedores o, en su defecto, se subastarán sin mínimo.
\item
  El Martillero Concursal deberá rendir cuenta de su gestión en los términos del Artículo 216.
\item
  Los bienes deberán venderse dentro de los cuatro meses siguientes a la fecha de celebración de la Junta Constitutiva o desde que ésta debió celebrarse en segunda citación. Tratándose de bienes incautados con posterioridad a aquélla, el término se contará desde el día de la diligencia de incautación.
\end{enumerate}

\hypertarget{artuxedculo-205.--deber-de-informaciuxf3n-y-cumplimiento-de-plazos.}{%
\paragraph*{Artículo 205.- Deber de información y cumplimiento de plazos.}\label{artuxedculo-205.--deber-de-informaciuxf3n-y-cumplimiento-de-plazos.}}
\addcontentsline{toc}{paragraph}{Artículo 205.- Deber de información y cumplimiento de plazos.}

En el caso que no sea posible cumplir con los plazos de realización fijados en la letra h) del Artículo anterior, el Liquidador deberá informar dicha circunstancia a la Superintendencia con a lo menos quince días de anticipación al vencimiento, explicando las razones del retraso. Lo anterior no lo exime de perseverar en la venta de los bienes, debiendo justificar su demora cada treinta días. En caso que el retraso fuere imputable al Liquidador, la Superintendencia podrá hacer uso de sus potestades sancionadoras, de conformidad a esta ley.

\hypertarget{artuxedculo-206.--acuerdos-de-la-junta-constitutiva-sobre-la-realizaciuxf3n-sumaria.}{%
\paragraph*{Artículo 206.- Acuerdos de la Junta Constitutiva sobre la realización sumaria.}\label{artuxedculo-206.--acuerdos-de-la-junta-constitutiva-sobre-la-realizaciuxf3n-sumaria.}}
\addcontentsline{toc}{paragraph}{Artículo 206.- Acuerdos de la Junta Constitutiva sobre la realización sumaria.}

Los acreedores podrán acordar, en la Junta Constitutiva y con Quórum Calificado, una fórmula de realización diferente a las señaladas en este Párrafo. Cualquiera sea la modalidad que se acuerde, ésta deberá ejecutarse dentro de los plazos indicados en la letra h) del Artículo 204.

\hypertarget{tuxedtulo-3.-de-la-realizaciuxf3n-ordinaria-de-bienes}{%
\subsection*{Título 3. De la realización ordinaria de bienes}\label{tuxedtulo-3.-de-la-realizaciuxf3n-ordinaria-de-bienes}}
\addcontentsline{toc}{subsection}{Título 3. De la realización ordinaria de bienes}

\hypertarget{puxe1rrafo-1.-de-las-normas-generales-1}{%
\subsubsection*{Párrafo 1. De las normas generales}\label{puxe1rrafo-1.-de-las-normas-generales-1}}
\addcontentsline{toc}{subsubsection}{Párrafo 1. De las normas generales}

\hypertarget{artuxedculo-207.--principio-general-de-realizaciuxf3n-ordinaria.}{%
\paragraph*{Artículo 207.- Principio general de realización ordinaria.}\label{artuxedculo-207.--principio-general-de-realizaciuxf3n-ordinaria.}}
\addcontentsline{toc}{paragraph}{Artículo 207.- Principio general de realización ordinaria.}

La determinación de la forma de realización de los bienes del deudor, sus plazos, condiciones y demás características, corresponderá a la Junta de Acreedores.

\hypertarget{artuxedculo-208.--fuxf3rmulas-de-realizaciuxf3n-ordinaria.}{%
\paragraph*{Artículo 208.- Fórmulas de realización ordinaria.}\label{artuxedculo-208.--fuxf3rmulas-de-realizaciuxf3n-ordinaria.}}
\addcontentsline{toc}{paragraph}{Artículo 208.- Fórmulas de realización ordinaria.}

Los bienes del deudor podrán realizarse mediante:

\begin{enumerate}
\def\labelenumi{\arabic{enumi})}
\item
  La venta al martillo de bienes muebles e inmuebles.
\item
  La venta por medio de remate en bolsa de valores si se trata de valores mobiliarios con presencia bursátil.
\item
  Otra forma distinta de realización de bienes, incluyendo entre ellas la venta como unidad económica establecida en el Artículo 217 y las ofertas de compra directa previstas en el Párrafo 4 de este Título.
\end{enumerate}

\hypertarget{artuxedculo-209.--plazos-para-la-realizaciuxf3n-ordinaria.}{%
\paragraph*{Artículo 209.- Plazos para la realización ordinaria.}\label{artuxedculo-209.--plazos-para-la-realizaciuxf3n-ordinaria.}}
\addcontentsline{toc}{paragraph}{Artículo 209.- Plazos para la realización ordinaria.}

Cualquiera sea la forma de realización de los activos, ésta deberá efectuarse en el menor tiempo posible, el que no podrá exceder de cuatro meses para los bienes muebles, y de siete para los inmuebles, ambos contados desde la fecha de celebración de la Junta Constitutiva o desde que ésta debió haberse celebrado en segunda citación.

Con todo, los acreedores podrán acordar, con Quórum Calificado y antes del vencimiento de los plazos señalados, su extensión fundada hasta por cuatro meses más. Podrá procederse al otorgamiento de nuevas prórrogas, las que deberán acordarse con el mismo quórum indicado anteriormente y contar con la autorización fundada de la Superintendencia.

La extensión del plazo podrá referirse a bienes específicos o, en general, a todos los bienes cuya realización esté pendiente.

\hypertarget{artuxedculo-210.--silencio-de-los-acreedores.}{%
\paragraph*{Artículo 210.- Silencio de los acreedores.}\label{artuxedculo-210.--silencio-de-los-acreedores.}}
\addcontentsline{toc}{paragraph}{Artículo 210.- Silencio de los acreedores.}

Los bienes cuya forma de enajenación no hubiere sido acordada por los acreedores dentro de los sesenta días contados desde la fecha de la Junta Constitutiva o desde la notificación del acta de incautación del activo correspondiente en caso que ésta se practicare con posterioridad, se enajenarán necesariamente de acuerdo a las reglas de la realización sumaria o simplificada. El Liquidador deberá dejar constancia de esta circunstancia en el expediente y, desde la fecha en que el tribunal lo tenga presente, se contará el plazo para enajenar previsto en la letra h) del Artículo 204.

\hypertarget{artuxedculo-211.--deber-de-informaciuxf3n-del-liquidador-y-fiscalizaciuxf3n-de-plazos.}{%
\paragraph*{Artículo 211.- Deber de información del Liquidador y fiscalización de plazos.}\label{artuxedculo-211.--deber-de-informaciuxf3n-del-liquidador-y-fiscalizaciuxf3n-de-plazos.}}
\addcontentsline{toc}{paragraph}{Artículo 211.- Deber de información del Liquidador y fiscalización de plazos.}

Si el Liquidador estima que no se podrá dar cumplimiento a los plazos de realización establecidos en el Artículo 209 deberá comunicarlo a la Superintendencia, explicando las razones del retraso. Esta comunicación deberá efectuarse a lo menos quince días antes del vencimiento del plazo de realización ordinaria. El incumplimiento de este deber de información será considerado falta grave para los efectos del número 2) del Artículo 338.

\hypertarget{artuxedculo-212.--regla-especial-para-realizaciones-impostergables.}{%
\paragraph*{Artículo 212.- Regla especial para realizaciones impostergables.}\label{artuxedculo-212.--regla-especial-para-realizaciones-impostergables.}}
\addcontentsline{toc}{paragraph}{Artículo 212.- Regla especial para realizaciones impostergables.}

El Liquidador podrá realizar en cualquier momento, al martillo o en venta directa, los bienes muebles del Deudor que considere que estén expuestos a próximo deterioro o desvalorización inminente o exijan una conservación dispendiosa. En la Junta inmediatamente posterior, el Liquidador deberá informar a los acreedores sobre los bienes realizados, su forma de enajenación y los recursos obtenidos de ella. Si no hubiere Juntas posteriores, cumplirá informando en tal sentido a la Superintendencia y consignándolo en las cuentas provisorias que deba rendir.

\hypertarget{puxe1rrafo-2.-de-las-ventas-al-martillo}{%
\subsubsection*{Párrafo 2. De las ventas al martillo}\label{puxe1rrafo-2.-de-las-ventas-al-martillo}}
\addcontentsline{toc}{subsubsection}{Párrafo 2. De las ventas al martillo}

\hypertarget{artuxedculo-213.--del-martillero-concursal.}{%
\paragraph*{Artículo 213.- Del Martillero Concursal.}\label{artuxedculo-213.--del-martillero-concursal.}}
\addcontentsline{toc}{paragraph}{Artículo 213.- Del Martillero Concursal.}

Sin perjuicio de las disposiciones contenidas en la ley N° 18.118, sobre ejercicio de la actividad de martillero público, se entenderán como martilleros habilitados para rematar bienes de un Procedimiento Concursal sólo aquellos incluidos en una nómina que al efecto confeccionará y llevará la Superintendencia.

Cualquier martillero que cumpla con los requisitos establecidos en el Artículo 14, en lo que les sean aplicables, y que se someta voluntariamente a las disposiciones de esta ley y a la fiscalización de la Superintendencia exclusivamente respecto de los Procedimientos Concursales en los que participe, podrá solicitar su inclusión en la Nómina de Martilleros Concursales.

\hypertarget{artuxedculo-214.--adopciuxf3n-del-acuerdo-y-formalidades-buxe1sicas.}{%
\paragraph*{Artículo 214.- Adopción del acuerdo y formalidades básicas.}\label{artuxedculo-214.--adopciuxf3n-del-acuerdo-y-formalidades-buxe1sicas.}}
\addcontentsline{toc}{paragraph}{Artículo 214.- Adopción del acuerdo y formalidades básicas.}

El acuerdo de venta al martillo podrá versar tanto sobre bienes muebles como inmuebles del Deudor. El acuerdo deberá designar al Martillero Concursal, elegido de una terna propuesta por el Liquidador y confeccionada sólo con aquellos Martilleros Concursales incluidos en la nómina llevada por la Superintendencia. Las demás condiciones de la venta deberán constar en las bases que proponga el Liquidador en la misma Junta, para la aprobación de los acreedores.

Con a lo menos cinco días de anticipación a la fecha del remate, el Liquidador deberá publicar en el Boletín Concursal las bases aprobadas por la Junta de Acreedores, sin perjuicio de otros medios adicionales de publicidad que las mismas bases puedan consignar.

\hypertarget{artuxedculo-215.--comisiuxf3n-del-martillero-concursal.}{%
\paragraph*{Artículo 215.- Comisión del Martillero Concursal.}\label{artuxedculo-215.--comisiuxf3n-del-martillero-concursal.}}
\addcontentsline{toc}{paragraph}{Artículo 215.- Comisión del Martillero Concursal.}

El Martillero Concursal percibirá una comisión única por el ejercicio de sus funciones, equivalente a un porcentaje sobre el monto total de realización de los bienes encargados rematar. Esta comisión será de cargo del adjudicatario.

La comisión señalada no podrá exceder de un 2\% sobre el monto total de realización de bienes inmuebles y de un 7\% sobre el monto total de realización de bienes muebles.

La Junta de Acreedores, con Quórum Calificado, podrá acordar aumentar la comisión correspondiente a un Martillero Concursal, en cuyo caso el aumento será de cargo del acreedor o acreedores que expresamente lo consientan. El señalado aumento de comisión deberá consignarse en el acuerdo de venta al martillo.

Cualquier contravención a este Artículo será sancionada conforme al Artículo 27 de esta ley.

A los Martilleros Concursales no les serán aplicables las comisiones reguladas en la ley N° 18.118.

\hypertarget{artuxedculo-216.--rendiciuxf3n-de-cuenta.}{%
\paragraph*{Artículo 216.- Rendición de cuenta.}\label{artuxedculo-216.--rendiciuxf3n-de-cuenta.}}
\addcontentsline{toc}{paragraph}{Artículo 216.- Rendición de cuenta.}

Dentro del quinto día siguiente a la fecha del remate, el Martillero Concursal deberá rendir ante la Superintendencia una cuenta detallada y desglosada de los bienes rematados, así como de los ingresos, gastos y resultado final del remate o subasta, y publicarla en el Boletín Concursal. La Superintendencia podrá objetar u observar su contenido, conforme a lo previsto en el número 5) del Artículo 337.

Asimismo, el Liquidador, el Deudor y los acreedores podrán objetar la cuenta presentada por los Martilleros Concursales, siendo aplicable lo dispuesto en los Artículos 49 y siguientes de esta ley en cuanto sea procedente.

\hypertarget{puxe1rrafo-3.-de-la-venta-como-unidad-econuxf3mica}{%
\subsubsection*{Párrafo 3. De la venta como unidad económica}\label{puxe1rrafo-3.-de-la-venta-como-unidad-econuxf3mica}}
\addcontentsline{toc}{subsubsection}{Párrafo 3. De la venta como unidad económica}

\hypertarget{artuxedculo-217.--acuerdo.}{%
\paragraph*{Artículo 217.- Acuerdo.}\label{artuxedculo-217.--acuerdo.}}
\addcontentsline{toc}{paragraph}{Artículo 217.- Acuerdo.}

La Junta de Acreedores podrá acordar vender un conjunto de bienes bajo la modalidad de venta como unidad económica. Esta modalidad se regirá por las siguientes reglas:

\begin{enumerate}
\def\labelenumi{\arabic{enumi})}
\item
  El acuerdo deberá incluir los bienes sujetos a la venta, cualquiera sea su naturaleza. En el evento de que se enajenare un conjunto de bienes ubicados en un bien raíz que no sea de propiedad del Deudor, se incluirán en la venta los derechos que en dicho inmueble le correspondan, cualquiera sea el tenor de la convención o la naturaleza de los hechos en que se funda la posesión, uso o mera tenencia del inmueble.
\item
  Asimismo, el acuerdo deberá señalar el precio mínimo de la venta del conjunto de bienes, forma de pago y garantías, sin perjuicio de las demás modalidades y condiciones de la enajenación que se puedan acordar.
\end{enumerate}

\hypertarget{artuxedculo-218.--efectos-del-acuerdo-de-venta-como-unidad-econuxf3mica.}{%
\paragraph*{Artículo 218.- Efectos del acuerdo de venta como unidad económica.}\label{artuxedculo-218.--efectos-del-acuerdo-de-venta-como-unidad-econuxf3mica.}}
\addcontentsline{toc}{paragraph}{Artículo 218.- Efectos del acuerdo de venta como unidad económica.}

Acordada la enajenación como unidad económica, se suspende el derecho de los acreedores hipotecarios, prendarios y retencionarios para iniciar o proseguir en forma separada las acciones dirigidas a obtener la realización de los bienes que garantizan sus respectivos créditos y que se encuentren comprendidos dentro de la unidad económica. La aprobación de las bases se entenderá como suficiente autorización para los efectos contemplados en los números 3 y 4 del Artículo 1464 del Código Civil.

\hypertarget{artuxedculo-219.--determinaciuxf3n-del-monto-de-realizaciuxf3n-de-los-bienes-hipotecados-prendados-o-retenidos.}{%
\paragraph*{Artículo 219.- Determinación del monto de realización de los bienes hipotecados, prendados o retenidos.}\label{artuxedculo-219.--determinaciuxf3n-del-monto-de-realizaciuxf3n-de-los-bienes-hipotecados-prendados-o-retenidos.}}
\addcontentsline{toc}{paragraph}{Artículo 219.- Determinación del monto de realización de los bienes hipotecados, prendados o retenidos.}

Cuando en el conjunto de bienes hubiere activos afectos a hipoteca, prenda o retención, la Junta de Acreedores podrá acordar que se indique específicamente en las bases la parte del precio de venta de la unidad económica que corresponderá a cada activo en garantía, tanto respecto del precio mínimo como de un eventual sobreprecio en caso de remate, para el sólo efecto de que dichos acreedores puedan hacer valer los derechos que procedan de acuerdo a esta ley. La parte del precio asignada al bien gravado con hipoteca, prenda o retenido no podrá ser inferior al Avalúo Fiscal o a la valorización que efectúe el Liquidador del bien gravado con prenda, salvo aceptación expresa del acreedor hipotecario, prendario o retencionario.

Los acreedores hipotecarios, prendarios o retencionarios que hubieren votado en contra de la valoración asignada por la Junta de Acreedores podrán solicitar al tribunal su rectificación, dentro de tercero día desde la adopción del respectivo acuerdo. En tal caso, el acreedor hipotecario, prendario o retencionario podrá acompañar siempre un informe pericial de tasación del respectivo bien, el cual tendrá presente el tribunal para la determinación final del valor.

En virtud de lo anterior, el tribunal citará a una audiencia verbal, que se celebrará a más tardar al quinto día con las partes que asistan. La citación a audiencia se notificará por el Estado Diario. El tribunal resolverá las objeciones deducidas en la audiencia y contra esa resolución sólo podrá deducirse reposición verbal, la que deberá ser resuelta en la misma oportunidad.

La tramitación de la rectificación solicitada no suspenderá la ejecución del acuerdo adoptado por la Junta de Acreedores.

\hypertarget{artuxedculo-220.--calificaciuxf3n-de-la-venta-de-los-bienes-como-unidad-econuxf3mica.}{%
\paragraph*{Artículo 220.- Calificación de la venta de los bienes como unidad económica.}\label{artuxedculo-220.--calificaciuxf3n-de-la-venta-de-los-bienes-como-unidad-econuxf3mica.}}
\addcontentsline{toc}{paragraph}{Artículo 220.- Calificación de la venta de los bienes como unidad económica.}

La venta de los bienes como unidad económica no calificará como venta de establecimiento comercial.

\hypertarget{artuxedculo-221.--truxe1mites-posteriores.}{%
\paragraph*{Artículo 221.- Trámites posteriores.}\label{artuxedculo-221.--truxe1mites-posteriores.}}
\addcontentsline{toc}{paragraph}{Artículo 221.- Trámites posteriores.}

La venta como unidad económica deberá constar en escritura pública en la que se indicarán los hechos y/o requisitos que acrediten el cumplimiento de las disposiciones anteriores. Dicha escritura será aprobada por el tribunal, el cual ordenará el alzamiento y cancelación de todos los gravámenes y prohibiciones que pesen sobre los bienes que integran la unidad económica.

Los bienes que integran la unidad económica se entenderán constituidos en hipoteca o prenda sin desplazamiento, según su naturaleza, por el sólo ministerio de la ley, para caucionar los saldos insolutos de precio y cualquiera otra obligación que el adquirente haya asumido como consecuencia de la adquisición, salvo que la Junta de Acreedores, al pronunciarse sobre las bases respectivas, hubiese excluido expresamente determinados bienes de tales gravámenes.

\hypertarget{puxe1rrafo-4.-de-la-oferta-de-compra-directa}{%
\subsubsection*{Párrafo 4. De la oferta de compra directa}\label{puxe1rrafo-4.-de-la-oferta-de-compra-directa}}
\addcontentsline{toc}{subsubsection}{Párrafo 4. De la oferta de compra directa}

\hypertarget{artuxedculo-222.--deber-de-informaciuxf3n-del-liquidador.}{%
\paragraph*{Artículo 222.- Deber de información del Liquidador.}\label{artuxedculo-222.--deber-de-informaciuxf3n-del-liquidador.}}
\addcontentsline{toc}{paragraph}{Artículo 222.- Deber de información del Liquidador.}

Todas las ofertas de compra directa que se formulen deberán dirigirse por escrito al Liquidador, quien las expondrá a los acreedores en la Junta de Acreedores inmediatamente siguiente.

\hypertarget{artuxedculo-223.--quuxf3rum-y-acuerdos.}{%
\paragraph*{Artículo 223.- Quórum y acuerdos.}\label{artuxedculo-223.--quuxf3rum-y-acuerdos.}}
\addcontentsline{toc}{paragraph}{Artículo 223.- Quórum y acuerdos.}

La aceptación por parte de la Junta de Acreedores de una oferta de compra directa requerirá de Quórum Especial. Tratándose de ofertas cuya venta no se pudo perfeccionar por no haberse logrado acuerdo con el quórum exigido, la Junta podrá acordar, por Quórum Calificado y con el conocimiento del oferente, que los bienes incluidos en la oferta de compra directa sean previamente ofrecidos en remate al martillo a cualquier interesado.

Las condiciones del remate deberán ser incluidas en las bases que se confeccionen y, en ellas, el precio mínimo de los bienes a rematar deberá ser igual al monto ofrecido por el oferente. Si no se presentaren postores en esa oportunidad, se llevará a cabo la venta propuesta por el oferente, en sus términos originales.

\hypertarget{puxe1rrafo-5.-del-leasing-o-arrendamiento-con-opciuxf3n-de-compra}{%
\subsubsection*{Párrafo 5. Del leasing o arrendamiento con opción de compra}\label{puxe1rrafo-5.-del-leasing-o-arrendamiento-con-opciuxf3n-de-compra}}
\addcontentsline{toc}{subsubsection}{Párrafo 5. Del leasing o arrendamiento con opción de compra}

\hypertarget{artuxedculo-224.--de-la-incautaciuxf3n.}{%
\paragraph*{Artículo 224.- De la incautación.}\label{artuxedculo-224.--de-la-incautaciuxf3n.}}
\addcontentsline{toc}{paragraph}{Artículo 224.- De la incautación.}

Los bienes que el Deudor tenga en su poder en virtud de un contrato de arrendamiento con opción de compra deberán ser incautados por el Liquidador en la forma dispuesta en los \#\#\#\#\# Artículos 163 y 164 de esta ley, debiendo dejar constancia en el acta que levante que se trata de bienes objeto de un contrato de arrendamiento con opción de compra.

Los gastos que irroguen la conservación, custodia y/o bodegaje de dichos bienes deberán ser asumidos por la masa. En caso de desacuerdo en el monto correspondiente, resolverá incidentalmente el juez competente, sin ulterior recurso.

\hypertarget{artuxedculo-225.--efecto-de-la-resoluciuxf3n-de-liquidaciuxf3n-en-los-contratos-de-arrendamiento-con-opciuxf3n-de-compra.}{%
\paragraph*{Artículo 225.- Efecto de la Resolución de Liquidación en los contratos de arrendamiento con opción de compra.}\label{artuxedculo-225.--efecto-de-la-resoluciuxf3n-de-liquidaciuxf3n-en-los-contratos-de-arrendamiento-con-opciuxf3n-de-compra.}}
\addcontentsline{toc}{paragraph}{Artículo 225.- Efecto de la Resolución de Liquidación en los contratos de arrendamiento con opción de compra.}

La dictación de la Resolución de Liquidación no constituirá causal de terminación inmediata del contrato de arrendamiento con opción de compra.

La Junta Constitutiva de Acreedores deberá pronunciarse y acordar al respecto alguna de las siguientes alternativas:

1.- Continuar con el cumplimiento del contrato de arrendamiento con opción de compra, en los términos originalmente pactados.

2.- Ejercer anticipadamente la opción de compra, en los términos establecidos en el respectivo contrato de arrendamiento con opción de compra.

3.- Terminar anticipadamente el contrato de arrendamiento con opción de compra, restituyendo el bien.

Para el caso en que no se celebrare la referida Junta, o ésta no se pronunciare al respecto, se entenderá que se opta por la alternativa regulada en el número 1 precedente.
Se tendrá por no escrita cualquier cláusula pactada en el contrato de arrendamiento con opción de compra, en contrario a lo regulado en este Artículo.

\hypertarget{artuxedculo-226.--de-la-verificaciuxf3n.}{%
\paragraph*{Artículo 226.- De la verificación.}\label{artuxedculo-226.--de-la-verificaciuxf3n.}}
\addcontentsline{toc}{paragraph}{Artículo 226.- De la verificación.}

El arrendador podrá verificar siempre en el Procedimiento Concursal de Liquidación del Deudor arrendatario aquellas cuotas devengadas e impagas hasta la fecha de la Resolución de Liquidación. Las cuotas que se devenguen con posterioridad a la Resolución de Liquidación y hasta la Junta Constitutiva serán siempre de cargo de la masa.

Respecto de las obligaciones que nazcan en virtud del ejercicio de las opciones reguladas en el Artículo anterior, se estará a lo siguiente:

\begin{enumerate}
\def\labelenumi{\alph{enumi})}
\item
  Si la Junta Constitutiva de Acreedores acordare continuar con el contrato de arrendamiento con opción de compra vigente en los términos originalmente pactados, las rentas que se devenguen con posterioridad a la fecha de la Resolución de Liquidación serán de cargo de la masa, y se pagarán en los términos y condiciones originalmente estipulados en el referido contrato.
\item
  Si la Junta Constitutiva de Acreedores acordare el ejercicio anticipado de la opción de compra en los términos originalmente pactados, su pago será de cargo de la masa. El Liquidador deberá efectuarlo dentro de los treinta días siguientes a la fecha en que se adoptó el acuerdo, prorrogables por igual período, previa autorización del tribunal.
\end{enumerate}

Si el pago no se hiciere efectivo dentro del plazo señalado, el acreedor arrendador podrá poner término al contrato de arrendamiento con opción de compra, debiendo el Liquidador restituir el bien al arrendador.

\begin{enumerate}
\def\labelenumi{\alph{enumi})}
\setcounter{enumi}{2}
\tightlist
\item
  Si la Junta Constitutiva de Acreedores acordare el término anticipado del contrato de arrendamiento con opción de compra, se deberá restituir al arrendador el bien objeto del referido contrato dentro de los 30 días siguientes a la fecha en que se adoptó el acuerdo, prorrogables por igual período, previa autorización del tribunal competente.
\end{enumerate}

Si el contrato incluyese multas, ellas podrán ser verificadas únicamente con el mérito de una sentencia definitiva firme o ejecutoriada que declare su procedencia y que conceda las cantidades reclamadas, procedimiento que se sustanciará mediante las reglas del juicio sumario.

\hypertarget{artuxedculo-227.--realizaciuxf3n-de-bienes-sujetos-a-un-contrato-de-arrendamiento-con-opciuxf3n-de-compra.}{%
\paragraph*{Artículo 227.- Realización de bienes sujetos a un contrato de arrendamiento con opción de compra.}\label{artuxedculo-227.--realizaciuxf3n-de-bienes-sujetos-a-un-contrato-de-arrendamiento-con-opciuxf3n-de-compra.}}
\addcontentsline{toc}{paragraph}{Artículo 227.- Realización de bienes sujetos a un contrato de arrendamiento con opción de compra.}

Sin perjuicio de lo señalado anteriormente, la Junta Constitutiva de Acreedores, con Quórum Calificado, podrá acordar con el arrendador una fórmula de realización que incluya los bienes objeto del contrato de arrendamiento con opción de compra, en cuyo caso se estará a las estipulaciones pactadas, las que deberán constar en el acta respectiva, la cual incluirá el valor que se asigna a dichos bienes.

La parte del crédito verificado con ocasión del contrato de arrendamiento con opción de compra que no alcance a ser cubierta con el producto de la realización del bien objeto del referido contrato, se considerará incobrable para todos los efectos legales a que hubiere lugar.

\hypertarget{puxe1rrafo-6.-de-las-reglas-complementarias-a-la-realizaciuxf3n}{%
\subsubsection*{Párrafo 6. De las reglas complementarias a la realización}\label{puxe1rrafo-6.-de-las-reglas-complementarias-a-la-realizaciuxf3n}}
\addcontentsline{toc}{subsubsection}{Párrafo 6. De las reglas complementarias a la realización}

\hypertarget{artuxedculo-228.--cruxe9ditos-morosos-y-activos-muebles-de-difuxedcil-realizaciuxf3n.}{%
\paragraph*{Artículo 228.- Créditos morosos y activos muebles de difícil realización.}\label{artuxedculo-228.--cruxe9ditos-morosos-y-activos-muebles-de-difuxedcil-realizaciuxf3n.}}
\addcontentsline{toc}{paragraph}{Artículo 228.- Créditos morosos y activos muebles de difícil realización.}

La Junta de Acreedores tendrá la facultad de vender, en la forma y al precio que estime convenientes, los créditos morosos y activos muebles de difícil realización, cumpliendo los requisitos que siguen:

\begin{enumerate}
\def\labelenumi{\arabic{enumi})}
\item
  Acuerdo de la Junta de Acreedores, adoptado por Quórum Calificado;
\item
  Que no se haya efectuado postura alguna respecto del bien, habiéndose ofertado al martillo y sin precio mínimo, o
\item
  Si el Liquidador ha efectuado las gestiones para realizarlo al martillo y al menos tres Martilleros Concursales hayan rechazado el encargo ofrecido por el bajo monto esperado de realización.
\end{enumerate}

\hypertarget{artuxedculo-229.--decisiuxf3n-de-no-perseverar-en-la-persecuciuxf3n-de-bienes.}{%
\paragraph*{Artículo 229.- Decisión de no perseverar en la persecución de bienes.}\label{artuxedculo-229.--decisiuxf3n-de-no-perseverar-en-la-persecuciuxf3n-de-bienes.}}
\addcontentsline{toc}{paragraph}{Artículo 229.- Decisión de no perseverar en la persecución de bienes.}

La Junta de Acreedores podrá acordar, con Quórum Calificado, la no persecución de uno o más bienes determinados del Deudor, en atención a que el costo estimado para recuperarlos es superior al beneficio esperado de su realización. Asimismo, el Liquidador podrá hacer uso de esta facultad si no se hubiese adoptado el acuerdo respectivo en dos Juntas de Acreedores ordinarias consecutivas por falta de quórum de asistencia, siempre que dicho asunto haya estado incluido en la tabla de ambas sesiones.

\hypertarget{tuxedtulo-4.-de-la-continuaciuxf3n-de-actividades-econuxf3micas}{%
\subsection*{Título 4. De la continuación de actividades económicas}\label{tuxedtulo-4.-de-la-continuaciuxf3n-de-actividades-econuxf3micas}}
\addcontentsline{toc}{subsection}{Título 4. De la continuación de actividades económicas}

\hypertarget{artuxedculo-230.--principio-general.}{%
\paragraph*{Artículo 230.- Principio general.}\label{artuxedculo-230.--principio-general.}}
\addcontentsline{toc}{paragraph}{Artículo 230.- Principio general.}

Se podrán desarrollar actividades económicas con los activos del Deudor con sujeción a las normas de este Título.

\hypertarget{artuxedculo-231.--tipos-o-clases.}{%
\paragraph*{Artículo 231.- Tipos o clases.}\label{artuxedculo-231.--tipos-o-clases.}}
\addcontentsline{toc}{paragraph}{Artículo 231.- Tipos o clases.}

La continuación de actividades económicas podrá ser:

\begin{enumerate}
\def\labelenumi{\arabic{enumi})}
\tightlist
\item
  Provisional: aquella que es decidida por el Liquidador con miras a:
\end{enumerate}

\begin{enumerate}
\def\labelenumi{\alph{enumi})}
\item
  Aumentar el porcentaje de recuperación por parte de los acreedores del Deudor;
\item
  Facilitar la ejecución de prestaciones que se encontraren pendientes y de las cuales se derive un beneficio para la masa, y
\item
  Propender a la realización de los activos del Deudor como unidad económica.
\end{enumerate}

El ejercicio de esta facultad sólo podrá tener lugar desde que el Liquidador asuma su cargo y se extenderá hasta la celebración de la Junta de Acreedores Constitutiva.

\begin{enumerate}
\def\labelenumi{\arabic{enumi})}
\setcounter{enumi}{1}
\tightlist
\item
  Definitiva: aquella que es acordada con Quórum Especial por la Junta de Acreedores Constitutiva u otra posterior, y a proposición del Liquidador o de cualquier acreedor.
\end{enumerate}

\hypertarget{artuxedculo-232.--continuaciuxf3n-provisional-de-actividades-econuxf3micas.}{%
\paragraph*{Artículo 232.- Continuación provisional de actividades económicas.}\label{artuxedculo-232.--continuaciuxf3n-provisional-de-actividades-econuxf3micas.}}
\addcontentsline{toc}{paragraph}{Artículo 232.- Continuación provisional de actividades económicas.}

La continuación provisional de actividades económicas del Deudor se regirá por las siguientes disposiciones:

\begin{enumerate}
\def\labelenumi{\arabic{enumi})}
\item
  El Liquidador deberá informar al tribunal y a la Superintendencia las razones que justifiquen su decisión, los bienes adscritos a la continuación provisional y la fecha exacta de su inicio. Estas comunicaciones deberán efectuarse al día siguiente de aquél en que el Liquidador disponga la continuación.
\item
  La administración de la continuación provisional de actividades económicas recaerá exclusivamente en el Liquidador, quien tendrá derecho a percibir un honorario adicional por esa gestión. El monto a percibir será determinado en la Junta de Acreedores Constitutiva y, en caso de desacuerdo, por el tribunal, en la misma Junta y sin ulterior recurso.
\item
  En la Junta de Acreedores Constitutiva el Liquidador deberá presentar a los acreedores un informe pormenorizado acerca de todas las operaciones ejecutadas en el desarrollo de la continuación provisional de actividades económicas, conjuntamente con un detalle de los ingresos y egresos del período y un resumen sobre la situación tributaria de la continuación referida.
\end{enumerate}

Una vez recibido el informe del Liquidador la Junta de Acreedores podrá acordar la continuación definitiva de dichas actividades, en cuyo caso regirán las disposiciones del Artículo siguiente.

\hypertarget{artuxedculo-233.--continuaciuxf3n-definitiva-de-actividades-econuxf3micas.}{%
\paragraph*{Artículo 233.- Continuación definitiva de actividades económicas.}\label{artuxedculo-233.--continuaciuxf3n-definitiva-de-actividades-econuxf3micas.}}
\addcontentsline{toc}{paragraph}{Artículo 233.- Continuación definitiva de actividades económicas.}

El acta de la Junta de Acreedores en que conste la continuación definitiva deberá contener, a lo menos, los siguientes puntos:

\begin{enumerate}
\def\labelenumi{\arabic{enumi})}
\item
  Actividades específicas a continuar.
\item
  Bienes adscritos. Si la continuación incluyese bienes hipotecados, prendados o sujetos al derecho legal de retención se suspenderá el derecho de los acreedores respectivos para ejercer sus acciones en tales bienes, siempre que hubieren votado a favor de dicha continuación.
\item
  Identificación del administrador siempre que fuere distinto del Liquidador y sus facultades. El acuerdo de nombramiento del Liquidador requerirá de Quórum Especial.
\item
  Honorarios totales o fórmula de cálculo correspondiente al plazo que se acuerde o resultados que se proyecten. Tratándose de pagos periódicos se aplicará al administrador el deber de retención previsto en el número 6) del Artículo 39 de esta ley.
\item
  Plazo. No podrá ser superior a un año contado desde el acuerdo respectivo. Será prorrogable por una sola vez, con Quórum Especial, mediante acuerdo obtenido en Junta de Acreedores Ordinaria o Extraordinaria celebrada al menos diez días antes del vencimiento.
\end{enumerate}

En caso de prórroga, la Junta deberá designar un administrador de la continuación de las actividades económicas, nombramiento que no podrá recaer en el Liquidador.

Si la Junta acordare la venta de los activos del Deudor como unidad económica, podrá también acordar, con Quórum Especial, proseguir la continuación por el tiempo indispensable para la concreción de ese acuerdo, aun cuando se exceda el plazo máximo ya indicado.

\hypertarget{artuxedculo-234.--administraciuxf3n-separada.}{%
\paragraph*{Artículo 234.- Administración separada.}\label{artuxedculo-234.--administraciuxf3n-separada.}}
\addcontentsline{toc}{paragraph}{Artículo 234.- Administración separada.}

Si la administración de la continuación definitiva de actividades económicas recayere en una persona distinta del Liquidador, se observarán las disposiciones siguientes:

\begin{enumerate}
\def\labelenumi{\arabic{enumi})}
\item
  Respecto de aquellos bienes no adscritos a dicha continuación, el Liquidador mantendrá su administración y procederá de conformidad a las reglas generales.
\item
  Respecto de los bienes adscritos a dicha continuación, el Liquidador tendrá las facultades del Artículo 294 del Código de Procedimiento Civil, reportando a la Junta de Acreedores Ordinaria las circunstancias que considere oportunas para el resguardo de los intereses de los acreedores y el Deudor.
\item
  Cualquier controversia que se suscite entre el administrador de la continuación definitiva de las actividades económicas y el Liquidador será resuelta por el tribunal en una audiencia verbal citada al efecto, para lo cual podrá solicitar informe a la Superintendencia.
\item
  La Superintendencia tendrá sobre el administrador de la continuación definitiva de las actividades económicas iguales potestades que sobre los Liquidadores.
\end{enumerate}

\hypertarget{artuxedculo-235.--informe-periuxf3dico.}{%
\paragraph*{Artículo 235.- Informe periódico.}\label{artuxedculo-235.--informe-periuxf3dico.}}
\addcontentsline{toc}{paragraph}{Artículo 235.- Informe periódico.}

El administrador deberá presentar en cada Junta un informe pormenorizado acerca de todas las actividades ejecutadas, y un detalle de los ingresos, egresos y utilidades o pérdidas del período.

\hypertarget{artuxedculo-236.--identificaciuxf3n-y-responsabilidad.}{%
\paragraph*{Artículo 236.- Identificación y responsabilidad.}\label{artuxedculo-236.--identificaciuxf3n-y-responsabilidad.}}
\addcontentsline{toc}{paragraph}{Artículo 236.- Identificación y responsabilidad.}

Tratándose de continuaciones definitivas de actividades económicas, el nombre o razón social del Deudor será complementado con la frase final ``en continuación de actividades económicas'', y su uso deberá ser precedido por la firma del administrador, en su caso, y de los demás habilitados. En caso contrario, serán solidariamente responsables de esas obligaciones tanto el administrador como los que hubieren ejecutado el acto o celebrado el contrato respectivo.

\hypertarget{artuxedculo-237.--tuxe9rmino-anticipado.}{%
\paragraph*{Artículo 237.- Término anticipado.}\label{artuxedculo-237.--tuxe9rmino-anticipado.}}
\addcontentsline{toc}{paragraph}{Artículo 237.- Término anticipado.}

La Junta, con Quórum Especial, podrá decidir el fin de la continuación definitiva de actividades económicas antes del término previsto, lo que será comunicado de inmediato al administrador.

Los honorarios pactados podrán reducirse proporcionalmente, previo acuerdo de las partes, resolviendo el juez en caso contrario, sin ulterior recurso y en el menor tiempo posible.

\hypertarget{artuxedculo-238.--responsabilidad-del-administrador.}{%
\paragraph*{Artículo 238.- Responsabilidad del administrador.}\label{artuxedculo-238.--responsabilidad-del-administrador.}}
\addcontentsline{toc}{paragraph}{Artículo 238.- Responsabilidad del administrador.}

La responsabilidad civil del administrador de la continuación de actividades económicas alcanzará hasta la culpa levísima y subsistirá hasta la aprobación de su cuenta definitiva de gestión. Dicha responsabilidad podrá perseguirse en juicio sumario una vez presentada la referida cuenta, conforme a lo dispuesto en los Artículos 49 y siguientes de esta ley, y sin perjuicio de la responsabilidad legal en que pudiere incurrir.

No obstante lo anterior, si el administrador de la continuación de actividades económicas no rindiere su cuenta definitiva de gestión dentro del plazo de treinta días contado desde el término de dicha continuación, su responsabilidad civil también podrá perseguirse desde el vencimiento de dicho plazo.

\hypertarget{artuxedculo-239.--cruxe9ditos-provenientes-de-la-continuaciuxf3n-de-actividades-econuxf3micas-del-deudor.}{%
\paragraph*{Artículo 239.- Créditos provenientes de la continuación de actividades económicas del Deudor.}\label{artuxedculo-239.--cruxe9ditos-provenientes-de-la-continuaciuxf3n-de-actividades-econuxf3micas-del-deudor.}}
\addcontentsline{toc}{paragraph}{Artículo 239.- Créditos provenientes de la continuación de actividades económicas del Deudor.}

Los créditos provenientes de la continuación de actividades económicas del Deudor podrán perseguirse solamente en los bienes comprendidos en ella y gozarán de la preferencia establecida en el número 4 del Artículo 2472 del Código Civil para el pago respecto de los demás acreedores del Deudor.

Los créditos de la continuación de actividades económicas del Deudor preferirán a los de los acreedores hipotecarios, prendarios y retencionarios que hubieren dado su aprobación a dicha continuación, sólo en el caso que los bienes no gravados comprendidos en ella fueren insuficientes para el pago. La diferencia, si la hubiere, será soportada por los señalados acreedores a prorrata del monto de sus respectivos créditos en el Procedimiento Concursal de Liquidación y hasta la concurrencia del valor de liquidación de los bienes dados en garantía de sus respectivos créditos.

El acreedor hipotecario, prendario o retencionario que pague más del porcentaje que le correspondiere de conformidad al inciso anterior, se subrogará por el exceso en los derechos de los acreedores de la continuación de actividades económicas, en conformidad a las normas del Párrafo 8 del Título XIV del Libro IV del Código Civil.

En el evento que en la continuación de actividades económicas se obtengan excedentes, éstos corresponderán a los acreedores del Deudor hasta la concurrencia del monto de sus créditos, reajustes e intereses, que corresponda pagar en el Procedimiento Concursal de Liquidación, deducidos los gastos. El remanente, si lo hubiere, pertenecerá al Deudor.

\hypertarget{artuxedculo-240.--cuenta-final-de-administraciuxf3n.}{%
\paragraph*{Artículo 240.- Cuenta Final de Administración.}\label{artuxedculo-240.--cuenta-final-de-administraciuxf3n.}}
\addcontentsline{toc}{paragraph}{Artículo 240.- Cuenta Final de Administración.}

Se aplicarán al administrador de la continuación definitiva de actividades económicas las disposiciones sobre Cuenta Final de Administración del Liquidador, sin entorpecer el Procedimiento Concursal de Liquidación ni la realización de los bienes del Deudor. Los honorarios que correspondan y la participación en las utilidades o el remanente retenido sólo podrán ser percibidos una vez que la referida cuenta se encuentre firme o ejecutoriada.

\hypertarget{tuxedtulo-5.-del-pago-del-pasivo}{%
\subsection*{Título 5. Del pago del pasivo}\label{tuxedtulo-5.-del-pago-del-pasivo}}
\addcontentsline{toc}{subsection}{Título 5. Del pago del pasivo}

\hypertarget{puxe1rrafo-1.-de-los-principios-generales}{%
\subsubsection*{Párrafo 1. De los principios generales}\label{puxe1rrafo-1.-de-los-principios-generales}}
\addcontentsline{toc}{subsubsection}{Párrafo 1. De los principios generales}

\hypertarget{artuxedculo-241.--orden-de-prelaciuxf3n.}{%
\paragraph*{Artículo 241.- Orden de prelación.}\label{artuxedculo-241.--orden-de-prelaciuxf3n.}}
\addcontentsline{toc}{paragraph}{Artículo 241.- Orden de prelación.}

Los acreedores serán pagados de conformidad a lo dispuesto en el Título XLI del Libro IV del Código Civil y, en el caso de los acreedores valistas, con pleno respeto a la subordinación de créditos establecida en la referida normativa. Para su eficacia, la subordinación deberá ser alegada al momento de la verificación del crédito por parte del acreedor beneficiario o bien notificarse al Liquidador, si se establece en una fecha posterior.

Los créditos de la primera clase señalados en el Artículo 2472 del Código Civil preferirán a todo otro crédito con privilegio establecido por leyes especiales.

Los acreedores Personas Relacionadas del Deudor, cuyos créditos no se encuentren debidamente documentados 90 días antes de la Resolución de Liquidación, serán pospuestos en el pago de sus créditos aun después de los acreedores valistas.

\hypertarget{artuxedculo-242.--acreedores-prendarios-y-retencionarios.}{%
\paragraph*{Artículo 242.- Acreedores prendarios y retencionarios.}\label{artuxedculo-242.--acreedores-prendarios-y-retencionarios.}}
\addcontentsline{toc}{paragraph}{Artículo 242.- Acreedores prendarios y retencionarios.}

Los acreedores de la segunda clase y aquellos que gocen del derecho de retención judicialmente declarado podrán optar por ejecutar individualmente los bienes gravados, en cuyo caso deberán iniciar ante el tribunal que conoce del Procedimiento Concursal de Liquidación, los procedimientos que correspondan, o continuarlos en él previa acumulación, debiendo siempre asegurar los créditos de mejor derecho.

El Liquidador podrá, si lo considera conveniente para la masa, exigir la entrega de la cosa dada en prenda o retenida, siempre que pague la deuda o deposite, a la orden del tribunal, su valor estimativo en dinero, sobre el cual se hará efectiva la preferencia.

\hypertarget{artuxedculo-243.--acreedores-hipotecarios.}{%
\paragraph*{Artículo 243.- Acreedores hipotecarios.}\label{artuxedculo-243.--acreedores-hipotecarios.}}
\addcontentsline{toc}{paragraph}{Artículo 243.- Acreedores hipotecarios.}

Los acreedores hipotecarios se pagarán en la forma que determinan los Artículos 2477, 2478, 2479 y 2480 del Código Civil.

\hypertarget{puxe1rrafo-2.-de-los-pagos-administrativos}{%
\subsubsection*{Párrafo 2. De los pagos administrativos}\label{puxe1rrafo-2.-de-los-pagos-administrativos}}
\addcontentsline{toc}{subsubsection}{Párrafo 2. De los pagos administrativos}

\hypertarget{artuxedculo-244.--procedencia-y-tramitaciuxf3n.}{%
\paragraph*{Artículo 244.- Procedencia y tramitación.}\label{artuxedculo-244.--procedencia-y-tramitaciuxf3n.}}
\addcontentsline{toc}{paragraph}{Artículo 244.- Procedencia y tramitación.}

Tan pronto existan fondos suficientes para ello y precaviendo que el activo remanente sea suficiente para asegurar los gastos del Procedimiento Concursal de Liquidación y el pago de los créditos de mejor derecho, podrán pagarse por el Liquidador los créditos contenidos en el Artículo 2472 del Código Civil, según las reglas que siguen:

\begin{enumerate}
\def\labelenumi{\arabic{enumi})}
\item
  Los descritos en los números 1 y 4 podrán pagarse sin necesidad de verificación.
\item
  Los incluidos en el número 5 podrán pagarse previa revisión y convicción del Liquidador sobre la suficiencia de los documentos que les sirven de fundamento, sin necesidad de verificación ni de acuerdo de Junta que apruebe el pago.
\item
  Los establecidos en el número 8 se pagarán en los mismos términos del número precedente, hasta el límite del equivalente a un mes de remuneración por cada año de servicio y fracción superior a seis meses por indemnizaciones convencionales de origen laboral y por las indemnizaciones legales del mismo origen que sean consecuencia de la aplicación de la causal señalada en el Artículo 163 bis del Código del Trabajo.
\end{enumerate}

Las restantes indemnizaciones de origen laboral, así como la que sea consecuencia del reclamo del trabajador de conformidad al Artículo 168 del Código del Trabajo, se pagarán con el sólo mérito de la sentencia definitiva firme o ejecutoriada que así lo ordene.

\begin{enumerate}
\def\labelenumi{\arabic{enumi})}
\setcounter{enumi}{3}
\tightlist
\item
  Con todo, podrán verificarse condicionalmente los créditos que gocen de las preferencias de los números 5 y 8, con el sólo mérito de la demanda interpuesta con anterioridad al inicio del Procedimiento Concursal de Liquidación o con la notificación al Liquidador de la demanda interpuesta con posterioridad al referido inicio.
\end{enumerate}

El Liquidador deberá reservar fondos suficientes para el evento que se acoja la demanda, sin perjuicio de los pagos administrativos que procedan, de conformidad a los números precedentes.

\hypertarget{artuxedculo-245.--costas.}{%
\paragraph*{Artículo 245.- Costas.}\label{artuxedculo-245.--costas.}}
\addcontentsline{toc}{paragraph}{Artículo 245.- Costas.}

Para efectos de lo dispuesto en el Artículo anterior, el pago de las costas personales se sujetará a las disposiciones siguientes:

\begin{enumerate}
\def\labelenumi{\arabic{enumi})}
\item
  En caso de Liquidación Forzosa, sólo procederán las correspondientes al acreedor peticionario, las que gozarán de la preferencia del número 1 del Artículo 2472 del Código Civil.
\item
  En caso de Liquidación Voluntaria, las costas personales del solicitante gozarán de la preferencia establecida en el número 4 del Artículo 2472 del Código Civil.
\item
  En ambos casos se aplicarán los siguientes límites al cálculo de costas:
\end{enumerate}

\begin{enumerate}
\def\labelenumi{\alph{enumi})}
\item
  El 2\% del crédito invocado, si éste no excede de 10.000 unidades de fomento, y
\item
  El 1\% en lo que exceda del valor señalado en la letra anterior.
\end{enumerate}

Para estos efectos, en casos de Liquidación Voluntaria, y siempre que el Deudor invocare más de un crédito, se estará a aquél en cuyo pago hubiere cesado en primer lugar. El saldo, si lo hubiere, se considerará valista.

\hypertarget{artuxedculo-246.--renunciabilidad-de-cruxe9ditos-de-origen-laboral.}{%
\paragraph*{Artículo 246.- Renunciabilidad de créditos de origen laboral.}\label{artuxedculo-246.--renunciabilidad-de-cruxe9ditos-de-origen-laboral.}}
\addcontentsline{toc}{paragraph}{Artículo 246.- Renunciabilidad de créditos de origen laboral.}

No podrán renunciarse los montos y preferencias de los créditos previstos en los números 5 y 8 del Artículo 2472 del Código Civil, salvo en la forma y casos que siguen:

\begin{enumerate}
\def\labelenumi{\arabic{enumi})}
\item
  Mediante conciliación celebrada ante un Juzgado de Letras del Trabajo, la que podrá tener lugar en la audiencia preparatoria o de juicio y deberá contar con la expresa aprobación del juez, y
\item
  En virtud de transacción judicial o extrajudicial que se celebre con posterioridad a la notificación de la sentencia definitiva de primera instancia del juicio laboral respectivo.
\end{enumerate}

\hypertarget{puxe1rrafo-3.-de-los-repartos-de-fondos}{%
\subsubsection*{Párrafo 3. De los repartos de fondos}\label{puxe1rrafo-3.-de-los-repartos-de-fondos}}
\addcontentsline{toc}{subsubsection}{Párrafo 3. De los repartos de fondos}

\hypertarget{artuxedculo-247.--propuesta-de-reparto-de-fondos.}{%
\paragraph*{Artículo 247.- Propuesta de reparto de fondos.}\label{artuxedculo-247.--propuesta-de-reparto-de-fondos.}}
\addcontentsline{toc}{paragraph}{Artículo 247.- Propuesta de reparto de fondos.}

El Liquidador deberá proponer a los acreedores un reparto de fondos siempre que se reúnan los siguientes requisitos copulativos:

\begin{enumerate}
\def\labelenumi{\arabic{enumi})}
\item
  Disponibilidad de fondos para abonar a los acreedores reconocidos una cantidad no inferior al cinco por ciento de sus acreencias.
\item
  Reserva previa de los dineros suficientes para solventar los gastos del Procedimiento Concursal de Liquidación y los créditos de igual o mejor derecho cuya impugnación se encuentre pendiente.
\item
  Reserva para responder a los acreedores residentes en el extranjero que no hayan alcanzado a comparecer, de conformidad a los plazos previstos en el Artículo 252.
\item
  Sujeción al procedimiento establecido en Artículo siguiente.
\end{enumerate}

\hypertarget{artuxedculo-248.--procedimiento-de-reparto-de-fondos.}{%
\paragraph*{Artículo 248.- Procedimiento de reparto de fondos.}\label{artuxedculo-248.--procedimiento-de-reparto-de-fondos.}}
\addcontentsline{toc}{paragraph}{Artículo 248.- Procedimiento de reparto de fondos.}

El Liquidador observará las disposiciones siguientes:

\begin{enumerate}
\def\labelenumi{\arabic{enumi})}
\item
  La proposición será presentada al tribunal conjuntamente con un detalle completo del reparto que se pretende efectuar, sus montos, fórmula de cálculo utilizada y acreedores a pagar.
\item
  El tribunal, al día siguiente de su proposición, tendrá por propuesto el reparto y ordenará al Liquidador publicarlo en el Boletín Concursal.
\item
  Los acreedores que conjunta o separadamente representen al menos el 20\% del pasivo con derecho a voto podrán objetar el reparto propuesto dentro del plazo de tres días contado desde la notificación.
\end{enumerate}

Si la objeción deducida afecta la totalidad del reparto, éste no podrá llevarse a cabo mientras la oposición no sea resuelta en primera instancia. Si la objeción deducida es parcial, el reparto podrá ejecutarse en la parte no disputada.

\begin{enumerate}
\def\labelenumi{\arabic{enumi})}
\setcounter{enumi}{3}
\item
  El tribunal conferirá traslado al Liquidador de todas las objeciones deducidas, el que deberá ser evacuado dentro de tercero día.
\item
  Transcurrido el término anterior, haya o no evacuado el Liquidador el traslado conferido, el tribunal resolverá sin más trámite la objeción. La resolución que se dicte no será susceptible de recurso alguno.
\item
  El objetante vencido será condenado al pago de costas, las que se calcularán sobre la base del monto objetado, salvo que haya tenido motivo plausible para litigar. Si la objeción hubiere sido deducida conjuntamente por dos o más acreedores, y fuere rechazada, todos ellos serán solidariamente responsables del pago de las costas.
\end{enumerate}

El Liquidador deberá perseguir en beneficio de la masa el cobro de las costas por cuerda separada ante el mismo tribunal, pudiendo solicitar que las fijadas sean descontadas del reparto presente o futuro que les correspondería al o los objetantes vencidos.

\begin{enumerate}
\def\labelenumi{\arabic{enumi})}
\setcounter{enumi}{6}
\item
  La resolución que acoja una impugnación deberá ordenar la confección de una nueva proposición de reparto.
\item
  No habiéndose deducido objeciones, rechazadas las interpuestas o modificado el reparto en la forma decretada por el tribunal, éste ordenará al Liquidador la distribución del reparto dentro del plazo de tres días contado desde que expire el término para objetar.
\item
  La resolución que ordene la distribución del reparto se notificará en el Boletín Concursal y desde entonces los acreedores incluidos en el reparto podrán reclamar al Liquidador el pago de las sumas correspondientes. En el caso de créditos afectos a subordinación, el o los acreedores subordinados contribuirán al pago de sus respectivos acreedores beneficiarios, a prorrata, con lo que les correspondiere en dicho reparto de su crédito subordinado.
\end{enumerate}

\hypertarget{artuxedculo-249.--acreedor-condicional.}{%
\paragraph*{Artículo 249.- Acreedor condicional.}\label{artuxedculo-249.--acreedor-condicional.}}
\addcontentsline{toc}{paragraph}{Artículo 249.- Acreedor condicional.}

El acreedor condicional podrá solicitar al tribunal que ordene la reserva de los fondos que le corresponderían cumplida la condición, o su entrega bajo caución suficiente de restituirlos a la masa, con el interés corriente para operaciones reajustables, para el caso de que la condición no se verifique. La caución señalada deberá constar en boleta de garantía bancaria o póliza de seguro, debiendo ser reemplazada o renovada sucesivamente hasta que se cumpla la respectiva condición.

\hypertarget{artuxedculo-250.--deudas-y-cruxe9ditos-recuxedprocos.}{%
\paragraph*{Artículo 250.- Deudas y créditos recíprocos.}\label{artuxedculo-250.--deudas-y-cruxe9ditos-recuxedprocos.}}
\addcontentsline{toc}{paragraph}{Artículo 250.- Deudas y créditos recíprocos.}

Cuando un acreedor fuere a la vez Deudor de quien está sujeto a un Procedimiento Concursal de Liquidación, sin que hubiere operado la compensación, las sumas que le correspondan a dicho acreedor se aplicarán al pago de su deuda, aunque no estuviere vencida.

\hypertarget{artuxedculo-251.--acreedores-que-verifican-su-cruxe9dito-extraordinariamente.}{%
\paragraph*{Artículo 251.- Acreedores que verifican su crédito extraordinariamente.}\label{artuxedculo-251.--acreedores-que-verifican-su-cruxe9dito-extraordinariamente.}}
\addcontentsline{toc}{paragraph}{Artículo 251.- Acreedores que verifican su crédito extraordinariamente.}

La verificación de los créditos de los acreedores realizada extraordinariamente no suspenderá la realización de los repartos, pero si encontrándose pendiente el reconocimiento de estos nuevos créditos se ordenare otro reparto, dichos acreedores serán comprendidos en él, por la suma que corresponda, en conformidad al siguiente inciso, manteniéndose en depósito las sumas que invocan hasta que sus créditos queden reconocidos.
Reconocidos sus créditos, los reclamantes tendrán derecho a exigir que los fondos materia de reparto que les hubieren correspondido en las distribuciones precedentes sean de preferencia cubiertos con los fondos no repartidos, pero no podrán demandar a los acreedores pagados en los anteriores repartos la devolución de cantidad alguna, aun cuando los bienes sujetos al Procedimiento Concursal de Liquidación no alcancen a cubrir íntegramente sus dividendos insolutos.

\hypertarget{artuxedculo-252.--situaciuxf3n-de-acreedores-fuera-del-territorio-de-la-repuxfablica.}{%
\paragraph*{Artículo 252.- Situación de acreedores fuera del territorio de la República.}\label{artuxedculo-252.--situaciuxf3n-de-acreedores-fuera-del-territorio-de-la-repuxfablica.}}
\addcontentsline{toc}{paragraph}{Artículo 252.- Situación de acreedores fuera del territorio de la República.}

La cantidad reservada para los acreedores residentes fuera del territorio de la República permanecerá en depósito hasta el vencimiento del duplo del término de emplazamiento que les corresponda. Vencido este plazo, se aplicará al pago de los créditos reconocidos.

\hypertarget{artuxedculo-253.--destino-de-los-fondos-en-caso-de-no-comparecencia.}{%
\paragraph*{Artículo 253.- Destino de los fondos en caso de no comparecencia.}\label{artuxedculo-253.--destino-de-los-fondos-en-caso-de-no-comparecencia.}}
\addcontentsline{toc}{paragraph}{Artículo 253.- Destino de los fondos en caso de no comparecencia.}

Si algún acreedor comprendido en la nómina de reparto no compareciere a recibir lo que le corresponda tres meses después de la notificación del reparto, el Liquidador depositará su importe en arcas fiscales a la orden del acreedor. Transcurridos tres años desde dicho depósito sin que se haya cobrado su monto, la Tesorería General de la República lo destinará en su integridad al Cuerpo de Bomberos.

\hypertarget{puxe1rrafo-4.-del-tuxe9rmino-del-procedimiento-concursal-de-liquidaciuxf3n}{%
\subsubsection*{Párrafo 4. Del término del Procedimiento Concursal de Liquidación}\label{puxe1rrafo-4.-del-tuxe9rmino-del-procedimiento-concursal-de-liquidaciuxf3n}}
\addcontentsline{toc}{subsubsection}{Párrafo 4. Del término del Procedimiento Concursal de Liquidación}

\hypertarget{artuxedculo-254.--resoluciuxf3n-de-tuxe9rmino.}{%
\paragraph*{Artículo 254.- Resolución de término.}\label{artuxedculo-254.--resoluciuxf3n-de-tuxe9rmino.}}
\addcontentsline{toc}{paragraph}{Artículo 254.- Resolución de término.}

Una vez publicada la resolución que tuvo por aprobada la Cuenta Final de Administración en los términos descritos en los Artículos 49 y siguientes, el tribunal, de oficio, a petición de parte o de la Superintendencia, dictará una resolución declarando terminado el Procedimiento Concursal de Liquidación.

Con la resolución de término del Procedimiento Concursal de Liquidación, el Deudor recuperará la libre administración de sus bienes.

\hypertarget{artuxedculo-255.--efectos-de-la-resoluciuxf3n-de-tuxe9rmino.}{%
\paragraph*{Artículo 255.- Efectos de la Resolución de Término.}\label{artuxedculo-255.--efectos-de-la-resoluciuxf3n-de-tuxe9rmino.}}
\addcontentsline{toc}{paragraph}{Artículo 255.- Efectos de la Resolución de Término.}

Una vez que se encuentre firme o ejecutoriada la resolución que declara el término del Procedimiento Concursal de Liquidación, se entenderán extinguidos por el solo ministerio de la ley y para todos los efectos legales los saldos insolutos de las obligaciones contraídas por el Deudor con anterioridad al inicio del Procedimiento Concursal de Liquidación.

Extinguidas las obligaciones conforme al inciso anterior, el Deudor se entenderá rehabilitado para todos los efectos legales, salvo que la resolución señalada en el Artículo precedente establezca algo distinto.

\hypertarget{artuxedculo-256.--recursos-contra-la-resoluciuxf3n-de-tuxe9rmino.}{%
\paragraph*{Artículo 256.- Recursos contra la resolución de término.}\label{artuxedculo-256.--recursos-contra-la-resoluciuxf3n-de-tuxe9rmino.}}
\addcontentsline{toc}{paragraph}{Artículo 256.- Recursos contra la resolución de término.}

La resolución que declare terminado el Procedimiento Concursal de Liquidación será susceptible de recurso de apelación, el que se concederá en el solo efecto devolutivo, conservando en el intertanto el Deudor la libre administración de sus bienes.

\hypertarget{puxe1rrafo-5.-del-tuxe9rmino-del-procedimiento-concursal-de-liquidaciuxf3n-por-acuerdo-de-reorganizaciuxf3n-judicial}{%
\subsubsection*{Párrafo 5. Del término del Procedimiento Concursal de Liquidación por Acuerdo de Reorganización Judicial}\label{puxe1rrafo-5.-del-tuxe9rmino-del-procedimiento-concursal-de-liquidaciuxf3n-por-acuerdo-de-reorganizaciuxf3n-judicial}}
\addcontentsline{toc}{subsubsection}{Párrafo 5. Del término del Procedimiento Concursal de Liquidación por Acuerdo de Reorganización Judicial}

\hypertarget{artuxedculo-257.--tuxe9rmino-del-procedimiento-concursal-de-liquidaciuxf3n-por-acuerdo-de-reorganizaciuxf3n-judicial.}{%
\paragraph*{Artículo 257.- Término del Procedimiento Concursal de Liquidación por Acuerdo de Reorganización Judicial.}\label{artuxedculo-257.--tuxe9rmino-del-procedimiento-concursal-de-liquidaciuxf3n-por-acuerdo-de-reorganizaciuxf3n-judicial.}}
\addcontentsline{toc}{paragraph}{Artículo 257.- Término del Procedimiento Concursal de Liquidación por Acuerdo de Reorganización Judicial.}

Durante el Procedimiento Concursal de Liquidación, una vez notificada la nómina de créditos reconocidos, el Deudor podrá acompañar al tribunal competente una propuesta de Acuerdo de Reorganización Judicial y le serán aplicables las disposiciones contenidas en el Capítulo III de esta ley, en lo que fuere procedente y en todo lo que no se regule en las disposiciones siguientes.

Presentada una propuesta de Acuerdo de Reorganización Judicial, el tribunal dictará una resolución que la tendrá por presentada. Una copia de la referida propuesta deberá ser publicada por el Liquidador en el Boletín Concursal.

En la misma resolución el tribunal competente fijará la fecha, lugar y hora en que deberá efectuarse la Junta de Acreedores llamada a conocer y pronunciarse sobre la propuesta de Acuerdo de Reorganización Judicial que presente el Deudor.

\hypertarget{artuxedculo-258.--acuerdo-de-la-junta-de-acreedores.}{%
\paragraph*{Artículo 258.- Acuerdo de la Junta de Acreedores.}\label{artuxedculo-258.--acuerdo-de-la-junta-de-acreedores.}}
\addcontentsline{toc}{paragraph}{Artículo 258.- Acuerdo de la Junta de Acreedores.}

Cada una de las clases o categorías de propuestas de Acuerdo de Reorganización Judicial acompañado por el Deudor será analizada, deliberada y acordada en forma separada en la misma Junta, sin perjuicio de lo previsto en el Artículo 82.

La propuesta de Acuerdo de Reorganización Judicial se entenderá acordada cuando cuente con el consentimiento del Deudor y el voto conforme de los dos tercios o más de los acreedores presentes, que representen tres cuartas partes del total del pasivo con derecho a voto, correspondiente a su respectiva clase o categoría. Las Personas Relacionadas con el Deudor no podrán votar, ni sus créditos se considerarán en el monto del pasivo.

\hypertarget{artuxedculo-259.--vigencia-del-acuerdo-de-reorganizaciuxf3n-judicial.}{%
\paragraph*{Artículo 259.- Vigencia del Acuerdo de Reorganización Judicial.}\label{artuxedculo-259.--vigencia-del-acuerdo-de-reorganizaciuxf3n-judicial.}}
\addcontentsline{toc}{paragraph}{Artículo 259.- Vigencia del Acuerdo de Reorganización Judicial.}

El Acuerdo del Reorganización Judicial regirá una vez vencido el plazo para impugnarlo, sin que se hubiere impugnado. En este caso se entenderá aprobado y el tribunal competente lo declarará así de oficio o a petición de cualquier interesado o del Veedor. En la misma resolución declarará el término legal del Procedimiento Concursal de Liquidación.

Si el Acuerdo de Reorganización Judicial fuere impugnado, regirá desde que cause ejecutoria la resolución que deseche la o las impugnaciones y lo declare aprobado.

Las resoluciones a que se refieren los incisos primero y segundo de este Artículo se notificarán en el Boletín Concursal.

El Acuerdo de Reorganización Judicial regirá no obstante las impugnaciones que se hubieren interpuesto en su contra. Sin embargo, si las impugnaciones fueren interpuestas por acreedores de una determinada clase o categoría, que representen en su conjunto a lo menos el 30\% del pasivo con derecho a voto de su respectiva clase o categoría, el Acuerdo de Reorganización Judicial no empezará a regir hasta que dichas impugnaciones fueren desestimadas por sentencia firme y ejecutoriada. En este caso, y en el del inciso segundo de este Artículo, los actos y contratos ejecutados o celebrados por el Deudor en el tiempo que medie entre el Acuerdo de Reorganización Judicial y la fecha en que quede ejecutoriada la resolución que acoja las impugnaciones, no podrán dejarse sin efecto.

El recurso de casación deducido en contra de la resolución de segunda instancia que deseche la o las impugnaciones no suspende el cumplimiento de dicha resolución, incluso si la parte vencida solicita se otorgue fianza de resultas por la parte vencedora.
Si se acogen las impugnaciones al Acuerdo de Reorganización Judicial por resolución firme o ejecutoriada, las obligaciones y derechos existentes entre el Deudor y sus acreedores con anterioridad a dicho Acuerdo, volverán al estado en que se encontraban en el Procedimiento Concursal de Liquidación.

  \bibliography{book.bib,packages.bib}

\end{document}
